%% Realease 1.0 dieser Software wurde am Institut f�r Intelligente Systeme der
%% Universit�t Stuttgart (http://www.informatik.uni-stuttgart.de/ifi/is/) unter
%% Leitung von Dietmar Lippold (dietmar.lippold@informatik.uni-stuttgart.de)
%% entwickelt.


\chapter{Performance}

Die folgenden Werte beziehen sich auf die Version 0.9 des Systems und sind
f�r die aktuelle Version nur noch begrenzt aussagekr�ftig.

\section{Einf�hrung}
In diesem Kapitel sind die Ergebnisse einiger Untersuchungen festgehalten, denen
das Compute-System unterzogen wurde, um herauszufinden, wie hoch der
Geschwindigkeitszuwachs bei mehreren Operatives ist, aber auch, wie hoch
der Zeitverlust ist, der sich durch das Verteilen der Probleme ergibt.

\section{Testkonfiguration}
Die Tests wurden in einem Computer-Pool durchgef�hrt, der w�hrend der
Durchf�hrung der Test auch von anderen Leuten benutzt wurde. Alle genannten
Zeiten verstehen sich, falls nicht anders bezeichnet, in Sekunden und sind
gerundete Mittelwerte mehrerer Messreihen eines bereits benutzten
Compute-Systems (siehe dazu ``Der erste Start'').

\subsection{Dispatcher}
  \begin{itemize}
  \item Dual-Athlon 1800+
  \item 1 GB RAM
  \item Red Hat Linux 9
  \item Sun JRE Version 1.4.2
  \end{itemize}
\subsection{Operatives}
  \begin{itemize}
  \item Pentium III 600 MHz
  \item 256 MB RAM
  \item Red Hat Linux 7.3
  \item Sun JRE Version 1.4.0
  \end{itemize}

\subsection{Testprobleme}
\subsubsection{Prim-Problem}
F�r das Prim-Problem ist die Aufgabenstellung: Berechne die 2000000te Primzahl.
Verwendet wurde die Klasse {\tt PrimTest} aus dem Package\\
{\tt de.unistuttgart.architeuthis.testenvironment.prim}.

\subsubsection{Caching-Problem}
Hier ist die Aufgabe: Berechne die 2000000te Primzahl. Au�erdem werden jedoch
drei Klassen zus�tzlich vom Webserver geladen, um zu testen, wie stark sich
das Caching der Klassen auf die Performance auswirkt. Verwendet wurde die Klasse
{\tt CachingTestProblem} in\\
{\tt de.unistuttgart.architeuthis.testenvironment.caching}.

\subsection{Referenzwerte}
Folgender Wert ist als Referenzwert zu betrachten: Die lokale Berechnung des
Prim-Problems auf einem Operative-Rechner ben�tigt 307,6s. Dies definiert
einem Speed-Up von 1. Wenn man das Prim-Problem auf einem Rechner verteilt (also
mit einem Dispatcher und einem Operative) berechnet, dann ergibt sich
eine Dauer von 413,8s, also ein ``Speed-Up'' von 0,74.
\begin{table*}[htbp]
  \centering
  \begin{tabular}{l l l}
  Beschreibung & Dauer in Sekunden & Speed-Up\\ \hline
  Berechnung mit lokaler Klasse & 307,6 & 1\\
  Verteilte Berechnung auf nur einem Rechner & 391,2 & 0,79\\
  \end{tabular}
  \caption{Referenzwerte}
\end{table*}

\section{Testergebnisse}
\subsection{Verschiedene Anzahlen von Operatives}
\subsubsection{Aufgabenstellung}
Dieser Test soll Aufschluss �ber die effektive Geschwindigkeitssteigerung geben,
die bei Verwendung mehrerer Operatives zu erwarten ist. Gestellt wurde das
Prim-Problem.

\subsubsection{Ergebnisse}

\begin{table*}[htbp]
  \centering
  \begin{tabular}{l l l}
  Anzahl Operatives & Gesamtdauer & Speed-Up\\
  \hline
  25 & 20,9 & 14,7 \\
  20 & 25,0 & 12,3\\
  15 & 33,2 & 9,3\\
  10 & 44,7 & 6,9\\
  5  & 84,2 & 3,6\\
  3  &133,6 & 2,3\\
  1  &349,2 & 0,88\\
  \end{tabular}
  \caption{Prim-Problem ohne Caching auf Operativeseite}
  
\end{table*}

\begin{table*}[htbp]
  \centering
  \begin{tabular}{l l l}
  Anzahl Operatives & Gesamtdauer & Speed-Up\\
  \hline
  25 & 20,9 & 14,7 \\
  20 & 24,8 & 12,4\\
  15 & 32,6 & 9,4\\
  10 & 46,6 & 6,6\\
  5  & 86,4 & 3,6\\
  3  &138,2 & 2,2\\
  1  &360,9 & 0,85\\
  \end{tabular}
  \caption{Prim-Problem mit aktiviertem Caching auf Operativeseite}
\end{table*}

Aus den Messwerten ist zu entnehmen, dass die Geschwindigkeit nahezu linear mit
der Anzahl der Operatives skaliert.

\subsection{Einfluss des Caching}

\subsubsection{Aufgabenstellung}
Wie oben bereits ersichtlich ist, beeinflusst das Caching bei gro�en
Rechenzeiten pro Teilproblem und wenig zu ladenden Klassen die Geschwindikeit
kaum. Um einen Unterschied zu messen, wurde das Caching-Problem gestellt. Bei
diesem Test wurden 15 Operatives verwendet.

\subsubsection{Ergebnisse}
\begin{table*}[htbp]
  \centering
  \begin{tabular}{l l}
  Status & Dauer in Millisekunden\\ \hline
  Caching aktiviert & 1182\\
  Caching deaktiviert & 2239\\
  \end{tabular}
  \caption{Berechnung des Caching-Problems mit 15 Operatives}
\end{table*}

Man sieht, dass erst bei sehr kurzen Berechnungszeiten der Teilprobleme das
Caching einen Geschwindigkeitsvorteil bringt, da dann das Laden der Klassen
�berproportional viel Zeit ben�tigt.

\subsubsection{Aufgabenstellung}
Ein weiterer Test zum Caching bestand darin, 3 Probleme, die ebenfalls die 
2000000te Primzahl berechnen, zu vergleichen. Zus�tzlich zum Prim- und zum
Caching-Problem wurde ein MyInteger-Problem geschrieben, dass die L�sung in
eine eigene Klasse, die ebenfalls vom Webserver geladen werden muss, schreibt.

\subsubsection{Ergebnisse}
\begin{table*}[htbp]
\centering
\begin{tabular}{l l l l}
Anzahl der Operatives & Prim-Problem & Caching-Problem & MyInteger-Problem \\
\hline
25 & 17  & 18  & 17  \\
20 & 21  & 22  & 23  \\
15 & 27  & 27  & 27  \\
10 & 41  & 41  & 42  \\
5  & 82  & 82  & 82  \\
3  & 133 & 132 & 131 \\
1  & 346 & 346 & 346 \\
\end{tabular}
\caption{Unterschiede zwischen den Problemen mit Cache}
\end{table*}

\begin{table*}[htbp]
\centering
\begin{tabular}{l l l l}
Anzahl der Operatives & Prim-Problem & Caching-Problem & MyInteger-Problem \\
\hline
25 & 22  & 25  & 22  \\
20 & 23  & 26  & 27  \\
15 & 31  & 32  & 32  \\
10 & 43  & 43  & 43  \\
5  & 85  & 82  & 81  \\
3  & 133 & 133 & 133 \\
1  & 350 & 345 & 346 \\
\end{tabular}
\caption{Unterschiede zwischen den Problemen ohne Cache}
\end{table*}

\subsubsection{Speedup}
Als Referenzwerte dienten:
\begin{table*}[htbp]
\centering
\begin{tabular}{l | l}
Prim-Problem & 304,6 \\
Caching-Problem & 305,2 \\
MyInteger-Problem & 310,3 \\
\end{tabular}
\end{table*}

Das ergibt einen Speedup von:
\begin{table*}[htbp]
\centering
\begin{tabular}{l l l}
Anzahl der Operatives & Prim-Problem & Prim-Problem \\
& mit Cache & ohne Cache \\ \hline
25 & 17,92 & 13,85 \\
20 & 14,51 & 13,24 \\
15 & 11,28 & 9,83  \\
10 & 7,43  & 7,08  \\
5  & 3,72  & 3,58  \\
3  & 2,29  & 2,29  \\
1  & 0,88  & 0,87  \\
\end{tabular}
\end{table*}

\begin{table*}[htbp]
\centering
\begin{tabular}{l l l}
Anzahl der Operatives& Caching-Problem & 
Caching-Problem \\
& mit Cache & ohne Cache \\ \hline
25 & 17,24 & 12,41\\
20 & 14,10 & 11,93\\
15 & 11,49 & 9,70 \\
10 & 7,57  & 7,22 \\
5  & 3,78  & 3,78 \\
3  & 2,35  & 2,33 \\
1  & 0,90  & 0,90 \\
\end{tabular}
\end{table*}

\begin{table*}[htbp]
\centering
\begin{tabular}{l l l}
Anzahl der Operatives & MyInteger-Problem & MyInteger-Problem \\
& mit Cache & ohne Cache\\ \hline
25 & 17,95 & 13,87 \\
20 & 13,27 & 11,30 \\
15 & 11,30 & 9,54  \\
10 & 7,27  & 7,10  \\
5  & 3,72  & 3,77  \\
3  & 2,33  & 2,29  \\
1  & 0,88  & 0,88  \\
\end{tabular}
\end{table*}

\subsection{Der erste Start}

\subsubsection{Aufgabenstellung}
W�hrend der Tests stellten wir fest, dass aufgrund einiger Eigenarten des
RMI-Systems die erste Eingabe eines Problems in das Compute-Systems stets l�nger
dauert als die folgenden Eingaben (selbst bei abgeschaltetem Caching). Dies
h�ngt vor allem mit der Benutzung der Socket-Verbindungen zusammen, da diese
unter RMI wiederverwendet werden, sobald einmal eine Verbindung hergestellt
wurde. Um diesen Einfluss zu testen, wurden die folgenden Testreihen
durchgef�hrt.

\subsubsection{Ergebnisse}
\begin{table*}[htbp]
  \centering
  \begin{tabular}{l l}
  Status & Dauer in Millisekunden\\ \hline
  Erster Start, Caching aktiviert & 7197\\
  Weitere Starts, Caching aktiviert & 1166\\ \hline
  Erster Start, Caching deaktiviert & 7457\\
  Weitere Starts, Caching deaktiviert & 2239\\
  \end{tabular}
  \caption{Berechnung des Caching-Problems mit 15 Operatives}
\end{table*}

\begin{table*}[htbp]
  \centering
  \begin{tabular}{l l}
  Status & Dauer in Millisekunden\\ \hline
  Erster Start, Caching aktiviert & 31,0\\
  Weitere Starts, Caching aktiviert & 20,4\\ \hline
  Erster Start, Caching deaktiviert & 29,3\\
  Weitere Starts, Caching deaktiviert & 21,2\\
  \end{tabular}
  \caption{Berechnung des Prim-Problems mit 25 Operatives}
\end{table*}

