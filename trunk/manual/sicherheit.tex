%% Realease 1.0 dieser Software wurde am Institut f�r Intelligente Systeme der
%% Universit�t Stuttgart (http://www.informatik.uni-stuttgart.de/ifi/is/) unter
%% Leitung von Dietmar Lippold (dietmar.lippold@informatik.uni-stuttgart.de)
%% entwickelt.


\chapter{Sicherheit}
\label{sicherh}

\section{Wichtiger Hinweis}

Bei Benutzung des Compute-Systems muss man sich dringend vor Augen f�hren,
dass man als Administrator eines Dispatchers oder Operatives explizit
anderen Benutzern erlaubt, Programme auf dem eigenen  Dispatcher-Rechner
oder Operative-Rechnern auszuf�hren. Mit der folgenden  Massnahme ist es
jedoch anderen Benutzern nicht m�glich, sch�dlichen Code auszuf�hren. Dies
gilt nat�rlich nur dann, wenn die policy-Dateien und die Security-Manager
korrekt eingebunden werden.


\section{Benutzung der policy-Dateien}

Durch die Angabe von {\tt -Djava.security.policy=<policy-Datei>} wird die 
Java Virtual Machine dazu veranlasst, den verwendeten Security-Manager 
entsprechend den Regeln in der policy-Datei zu konfigurieren. Die 
policy-Datei muss dazu im Dateisystem liegen, darf sich also nicht in einer
jar-Datei mit den Klassen befinden. Deshalb ist der Pfad relativ oder
absolut anzugeben.


\section{Bedeutung der policy-Dateien}

Um eine m�glichst gro�e Sicherheit zu gew�hrleisten, werden in allen Teilen
des Systems Security-Manager verwendet (f�r n�here Informationen siehe bitte
Java-API-Dokumentation). Damit die einzelnen Programmteile nur die
n�tigsten Rechte auf den ausf�hrenden Computern erhalten, gibt es die
policy-Dateien zur Konfiguration der Security-Manager. Dieses Konzept bietet
den Vorteil, dass jeder, der ein Java-Programm ausf�hrt, in Klartext
(zumindest beinahe) die Rechte einsehen und vorgeben kann, die das
Java-Programm hat, indem er direkt die policy-Dateien anschaut bzw.\ �ndert.

Welche policy-Dateien f�r das Compute-System im Einzelnen zu verwenden
sind, ist in den Kapiteln zu den jeweiligen Programmteilen angegeben.

Wichtige Einschr�nkungen durch die policy-Dateien f�r das Compute-System
sind:

\begin{itemize}
\item Kein Zugriff auf das Dateisystem (au�er durch den Problem-�bermittler).
\item Kein Zugriff auf die Systemeigenschaften.
\item Kein Zugriff auf die Security-Manager.
\end{itemize}

Erlaubt ist standardm��ig jedoch:

\begin{itemize}
\item Jeglicher Netzwerkverkehr.
\end{itemize}

F�r das Benutzen eines neuen RMIClassLoaderSpi f�r den Dispatcher und
eventuell f�r den Operative wird zus�tzlich ben�tigt:

\begin{itemize}
\item Erstellen und Benutzen von eigenen ClassLoadern.
\item Zugriff auf die RMI-Codebase.
\item Zugriff auf Sockets, um den Socket zum Laden von Klassen zu setzen.
\end{itemize}

Die weiteren Einstellungen k�nnen direkt aus den policy-Dateien abgelesen
werden. Die oben genannten Einschr�nkungen k�nnen nat�rlich aufgehoben
werden, falls dies in einer Anwendungssituation notwendig ist. Es ist
prinzipiell auch m�glich, nur gewissen signierten Java-Klassen weitere
Rechte zu geben. Dazu sei jedoch auf die Dokumentation zu Java verwiesen.


\section{H�here Sicherheit}

Die policy-Dateien k�nnen an einigen Stellen weiter eingeschr�nkt werden. So
kann auf dem Operative beispielsweise nur Netzwerk-Verkehr zu einem
bestimmten Dispatcher erlaubt werden. Um diese weiteren Einstellungen
vorzunehmen wird  jedoch empfohlen, die policy-Dateien direkt zu editieren,
und die dortigen  Kommentare zu lesen.


\section{Sicherheit f�r den Benutzer}

Da sich beliebige Operatives an das Compute-System zur Berechnung von
Teilproblemen anmelden d�rfen, w�re es auch denkbar, dass die Ergebnisse
dadurch verf�lscht werden, dass einzelne Operatives mutwillig falsche
Ergebnisse zur�ckliefern. Dies l��t sich beispielsweise dadurch verhindern,
dass man in der policy-Datei des Dispatchers nur bestimmte zuverl�ssige
Rechner als Operatives erlaubt, indem man zu anderen Rechnern die
Netzwerkverbindungen untersagt. Beispiele dazu finden sich direkt in der
Datei {\tt dispatcher.pol} oder in der Java API Dokumentation unter
{\tt java.net.Socket\-Permission}.

