%% Realease 1.0 dieser Software wurde am Institut f�r Intelligente Systeme der
%% Universit�t Stuttgart (http://www.informatik.uni-stuttgart.de/ifi/is/) unter
%% Leitung von Dietmar Lippold (dietmar.lippold@informatik.uni-stuttgart.de)
%% entwickelt.


\chapter[Beispiele zur Parallelisierung]{Der Weg zum parallelisierten Programm
         anhand eines Beispiels}

Dieses Kapitel beschreibt die einfache Parallelisierung eines vorgegebenen
Programms.


\section{Aufgabenstellung}

Man betrachte zun�chst die Methode {\tt primzahlTeilbereich} aus
\texttt{de.unistuttgart.architeuthis.testenvironment.PrimeNumbers}:

{\scriptsize \verbatiminput{java/PrimzahlTeilbereich.jav}}

Ermittelt werden sollen also alle Primzahlen, die in einem Intervall von
zwei Zahlen liegen. Da dies so geschieht, da� die Zahlen einzeln �berpr�ft
werden, ob sie jeweils eine Primzahl sind, sind die �berpr�fungen
unabh�ngig voneinander. Die Gesamtaufgabe kann also recht einfach in
Teilaufgaben zerlegt werden, indem das Gesamtintervall in Teilintervalle
zerlegt wird. Jedes Teilintervall bildet dann ein Teilproblem und eine
Teill�sung ist eine Liste der Primzahlen aus dem Teilintervall. Die
Gesamtl�sung ergibt sich dann durch hintereinander h�ngen der Listen
der Teill�sungen. Dabei ist zu beachten, da� die Primzahlen in der
Gesamtliste aufsteigend sortiert sein m�ssen.


\subsection{Implementierung}

Die Sortierung der Primzahlen in der Gesamtliste kann dadurch implizit
erfolgen, da� den Teilproblemen die Teilintervalle aufsteigend zugeordnet
werden (wie es naheliegend ist) und die Teill�sungen (die Listen der
Primzahlen) in der entsprechenden Reihenfolge aneinander geh�ngt werden.
Da die Berechnung der Teilprobleme unabh�ngig voneinander ist, k�nnen sie
au�erdem alle geleichzeitig erzeugt werden.

Die R�ckgabe der Teill�sungen in der Reihenfolge der Ausgabe der Teilprobleme
wird durch die zur Verf�gung stehenden abstrakten Hilfsklassen
(s.\ Kapitel \ref{abstrakt}) erledigt. Da die Teilprobleme au�erdem alle
gleichzeitig erzeugt werden k�nnen, kann die Klasse
\texttt{AbstractFixedSizeProblem} verwendet werden. F�r diese m�ssen
die folgenden beiden Methoden implementiert werden:

\begin{itemize}
\item protected PartialProblem[] createPartialProblems(int suggestedParProbs)
\item protected Serializable createSolution(PartialSolution[] partialSolutions)
\end{itemize}

Deren Implementierung erfolgt wie schon beschrieben. In
\texttt{createPartialProblems} werden die Teilprobleme dadurch erzeugt,
da� jedem Teilproblem ein Teilintervall zugeordnet wird. In
\texttt{createSolution} werden dann alle Teill�sungen ihrer Reihenfolge
im Array aneinandergeh�ngt (zum konkreten Typ der Elemente von Array s.u.).
Das Gesamtintervall wird die Unter- und Obergrenze dem Konsruktor �bergeben.
Nachfolgend die vollst�ndige Implementierung der Problem-Klasse.

{\scriptsize \verbatiminput{java/basic/PrimeRangeProblemImpl.jav}}

Die Implementierung des Teilproblems erfolgt einfach dadurch, da� dem
Konstruktor wird das zu durchsuchende Teilintervall �bergeben wird und
in der Methode \texttt{compute()} zur Berechnung der Primzahlen die
vorgegebene Methode \texttt{primzahlTeilbereich} aufgerufen wird. Um die
Liste als Teill�sung zur�ckzugeben, wird die zur Verf�gung stehende
Hilfsklasse \texttt{ContainerPartialSolution} verwandt. Aus dieser mu�
die Liste dann in der schon oben genannten Methode \texttt{createSolution}
abgerufen werden. Nachfolgend die vollst�ndige Implementierung der
Teilproblem-Klasse.

{\scriptsize \verbatiminput{java/basic/PrimePartialProblemImpl.jav}}


\section{Weitere Beispiele}

Weiter Beispiele kann man Package \texttt{de.unis.architeuthis.testumgebung}
finden. Hier eine kurze Auflistung der Probleme und jeweils eine kurze
Beschreibung dazu.

\begin{enumerate}
\item \texttt{de...testenvironment.prime.basic.PrimRangeProblemImpl}\\
	Dies ist das obige Beispiel dieses Kapitels mittels der abstrakten
        Klasse \texttt{AbstractFixedSizeProblem}.

\item \texttt{de...testenvironment.prime.PrimRangeProblemImpl}\\
        Dieses Problem macht das gleiche wie das obige Problem, nur da�
        die Aufteilung in Teilprobleme geschickter so erfolgt, da� der
        Rechenaufwand f�r die einzelnen Teilprobleme noch gleicher ist.

\item \texttt{de...testenvironment.prime.PrimSequenceProblemImpl}\\
	Dies ist das Beispiel f"ur Fortgeschrittene im n"achsten Kapitel.

\item \texttt{de...testenvironment.montecarlo.MonteCarloProblemImpl}\\
        Hier wird das MonteCarlo-Verfahren zur Bestimmung der Zahl $Pi$
        parallelisert. Man bekommt somit bei einer festgelegten Rechenzeit
        eine bessere Genauigkeit f"ur $Pi$.

\item \texttt{de...testenvironment.random.RandomProblemImpl}\\
        Hier wartet jedes Teilproblem einfach eine zuf"allige Anzahl von
        Sekunden. Dies ist f"ur Testzwecke recht interessant.

\item \texttt{de...testenvironment.fail.FailProblemImpl}\\
        Ein Problem f�r Testzwecke, bei dem die Abfrage nach einer L�sung
        immer den Wert \texttt{null} liefert.

\item \texttt{de...testenvironment.caching.CachingTestProblem}\\
        Hier werden drei Dummy-Klassen in den Teilproblemen geladen. Mit
        diesem Problem wurde verglichen, inwiefern sich das Laden von
        Klassen zeitlich bemerkbar macht. Siehe dazu auch im Kapitel
        Performance.

\item \texttt{de...testenvironment.hashstore.HashStoreProblemImpl}\\
        Bei diesem Problem benutzen die Teilprobleme das Package
        \texttt{de...remotestore.hashmap} als gemeinsamen Speicher.
\end{enumerate}

