%% Realease 1.0 dieser Software wurde am Institut f�r Intelligente Systeme der
%% Universit�t Stuttgart (http://www.iis.uni-stuttgart.de/) unter Leitung
%% von Dietmar Lippold (dietmar.lippold@informatik.uni-stuttgart.de)
%% entwickelt.


\documentclass[a4,10pt]{report}
\usepackage[german]{babel} % Einbindung des deutschen Sprachpaketes
\usepackage[ansinew]{inputenc}
\usepackage{verbatim}
\usepackage[T1]{fontenc}
\usepackage{palatino}
\pagestyle{headings}
\usepackage{parskip}

\usepackage[pdftex=true,
            pdftitle={Benutzerhandbuch f�r Architeuthis},
            pdfauthor={J�rgen Heit, Andreas Heydlauff, Ralf Kible,
                       Achim Linke, Dietmar Lippold},
            pdfsubject={Beschreibung der Benutzung des Systems Architeuthis},
%           pdfkeywords={keyword 1, keyword 2, keyword 3, keyword 4},
%           plainpages=false,
%           hypertexnames=false,
            pdfpagelabels=true,
            pdfpagemode=UseOutlines,
            hyperindex=true,
            colorlinks=true]{hyperref} 

% Hier wird die Farbe der Links ein- und ausgeschaltet.
\usepackage{color}
\definecolor{darkred}{rgb}{0.5,0,0}
\definecolor{darkgreen}{rgb}{0,0.5,0}
\definecolor{darkblue}{rgb}{0,0,0.8}
\hypersetup{colorlinks,
            linkcolor=darkblue,
            filecolor=darkgreen,
            urlcolor=darkred,
            citecolor=darkblue}

%% M�gliche Werte f�r secnumdepth:
%% -1
%% keine �berschrift wird numeriert 
%% 0
%% Kapitel�berschriften werden numeriert 
%% 1
%% Kapitel- und Abschnitts�berschriften werden numeriert 
%% 2
%% Kapitel- bis Unterabschnitts�berschriften werden numeriert 
%% 3
%% Kapitel- bis Unterunterabschnitts�berschriften werden numeriert 
%% 4
%% Kapitel- bis Paragraphs�berschriften werden numeriert 
%% 5
%% alle �berschriften werden numeriert 
\setcounter{secnumdepth}{5}

\setcounter{tocdepth}{2}

\begin{document}
\title{Architeuthis - Benutzerhandbuch}
\author{J�rgen Heit, juergen.heit@gmx.de\\
        Andreas Heydlauff, andiheydlauff@gmx.de\\
        Ralf Kible, ralf\_kible@gmx.de\\
        Achim Linke, achim81@gmx.net\\
        Dietmar Lippold, dietmar.lippold@informatik.uni-stuttgart.de}

\date{Software Version: 2.0\\Stand: \today}
\maketitle
\tableofcontents
%\listoffigures
%\listoftables

\chapter{Allgemeine Hinweise}
Das Compute--System soll einem Benutzer die M"oglichkeit geben, ein von
ihm in Java implementiertes Problem auf mehreren Rechnern verteilt berechnen
zu lassen. Es besteht aus mehreren Komponenten, die hier erl"autert werden.
\\
\\
Da die Packages recht tief geschachtelt sind, werden deren Namen bei der
Angabe von Klassen teilweise gek�rzt. Anstatt\\
\texttt{de.unistuttgart.architeuthis.Klasse} wird dann nur
\texttt{de...Klasse} geschrieben. Wenn die Klasse kurz davor schon genannt
wurde, wird teilweise auf die Angabe vom Package komplett verzichtet. 

    %% Realease 1.0 dieser Software wurde am Institut f�r Intelligente Systeme der
%% Universit�t Stuttgart (http://www.informatik.uni-stuttgart.de/ifi/is/) unter
%% Leitung von Dietmar Lippold (dietmar.lippold@informatik.uni-stuttgart.de)
%% entwickelt.


\chapter{Funktionen der Komponenten des Systems}

\section{Vorwort}

Dieses Kapitel basiert auf der Spezifikation der hier dokumentierten
Software. Jedoch scheint es sich ebenfalls sehr gut als Kurz�bersicht �ber
die Funktionen und die Arbeitsweise des Compute-Systems zu eignen.

\section{Grunds�tzliches:}

\begin{itemize}
   \item Der Name des Systems ist Architeuthis.

   \item Zu erstellen ist ein Compute-System, das die Verwaltung f�r die
         parallele Berechnung einer beliebigen parallelisierbaren Aufgabe
         (im folgenden {\em Problem} genannt, siehe auch Glossar)
         �bernimmt.

   \item Die Implementierung erfolgt platformunabh�ngig in Java mit Hilfe von
         RMI.

   \item Die Implementierung besteht aus mehreren Programmteilen:
         Operative, Dispatcher, Problem-�bermittler,
           sowie einer Testumgebung (mit Beispielen von Problemen).
         F�r die korrekte Funktionalit�t muss eine RMI-f�hige Verbindung
         (Netzwerkverbindung) zwischen Problem-�bermittler und Dispatcher
         und ebenso zwischen Operative und Dispatcher bereitstehen.
\end{itemize}


\section{Dispatcher:}

Der Dispatcher ist ein ausf�hrbares Java-Programm, das pro Compute-System
genau einmal zur Verf�gung steht. Er besteht aus drei Komponenten:

\begin{enumerate}
    \item Dem Problem-Manager 
    \item Dem Compute-Manager
    \item Einer Statistik-Komponente
\end{enumerate}

\subsection{Problem-Manager}
\begin{itemize}
\item Die ben"otigten Klassen des Problems werden vom Problem-�bermittler an
      den Problem-Manager ``�bertragen''. Der Problem-�bermittler "ubergibt
      ihm die URL, an der das Problem-Paket inklusive ihrer
      Hauptproblemklasse liegt und entweder den Namen und die
      serialisierbaren Parameter des Konstruktors oder eine serialisierbare
      Instanz der Problemklasse. Bei der �bergabe eines Namens wird die
      Klasse vom Webserver geladen und das Problem wird initialisiert.

\item Nach der Problem�bergabe weist er das Problem an, sich in Teilprobleme
      zu unterteilen. Dabei wird vom Dispatcher eine Anzahl von Teilen 
      vorgeschlagen, die sich nach der Anzahl der momentan verf�gbaren
      Operatives richtet. Diese Teilprobleme werden einzeln von der
      Problemklasse geholt und �ber den Compute-Manager an Operatives
      verteilt.

\item Gibt die vom Compute-Manager erhaltene Teill�sung an das Problem
      weiter. Falls die Gesamtl�sung durch diese Teill�sung erzeugt werden
      kann, wird diese an den Problem-�bermittler gesendet.
\end{itemize}


\subsection{Compute-Manager}
\begin{itemize}
\item Operatives k�nnen sich bei ihm durch ein Interface registieren
      und abmelden.

\item Er erledigt die Verteilung der Teilprobleme eines oder mehrerer
      Probleme an die Operatives.

\item Falls Teilprobleme vorhanden sind, bekommen anmeldende 
      Operatives eines zum Berechnen.

\item Teilprobleme k�nnen auch mehrfach vergeben werden, sobald keine neuen
      Teilprobleme mehr vorhanden sind, aber noch welche in Berechnung sind.

\item Empf"angt die Teill"osungen der Operatives. Diese werden dann
      an den Problem-Manager weitergeleitet. Dadurch frei werdende Operatives
      bekommen ein neues Teilproblem zum Berechnen, falls eines vorhanden ist.

\item Wenn der Dispatcher keine Teilprobleme mehr hat und auch keine 
      mehr in Berechnung sind, wird jeder Operative, der eine Teill�sung 
      zur�ck gibt, als verf�gbar markiert, so dass ihm ein neues Teilproblem 
      geschickt wird, sobald welche vorhanden sind.

\item "Uberwacht die Operatives auf Erreichbarkeit.
\end{itemize}

\subsection{Statistik}
Es werden verschiedene Statistiken f�r den Benutzer gef�hrt. 
Sie enthalten Informationen �ber den aktuellen Zustand des 
Dispatchers oder �ber einzelne Probleme.

\begin{itemize}
\item  Eine allgemeine Statistik des Dispatchers enth�lt folgende Punkte:
       \begin{enumerate}
        \item Anzahl der angemeldeten Operatives
        \item Anzahl der Operatives, die gerade kein Teilproblem berechnen
        \item Anzahl der noch nicht fertig berechneten Probleme
        \item Anzahl der insgesamt erhaltenen Probleme
        \item Anzahl der berechnete Teilprobleme
        \item Anzahl der gerade in Berechnung befindlichen Teilprobleme
        \item Durchschnittliche Berechungsdauer der fertig berechneten
              Teilprobleme
        \item Berechungsdauer aller fertig berechneten Teilprobleme
        \item Alter des Dispatcher
       \end{enumerate}

\item  Folgende Punkte werden in der problemspezifischen Statistik f�r jedes
       einzelne Problem gef�hrt:
       \begin{enumerate}
        \item Anzahl der fertigen Teilprobleme
        \item Anzahl der gerade in Berechnung befindlichen Teilprobleme
        \item Durchschnittliche Berechungsdauer der fertig berechneten
              Teilprobleme
        \item Berechnungsdauer der fertig berechneten Teilprobleme
        \item Anzahl der vorgeschlagenen Teilprobleme vom Compute-Manager
              (entspricht der Anzahl der freien Operatives beim
              Initialisieren des Problems)   
        \item Alter des Problems auf dem Dispatcher
     \end{enumerate}
\end{itemize}

W�hrend des Betriebs kann eine Momentaufnahme jeder Statistik vom Benutzer 
in extra Programmen abgefragt werden und dort grafisch aufbereitet 
angezeigt werden. Eine problemspezifische Endstatistik wird mit der
Gesamtl�sung an  den Problem-�bermittler gesandt.

\section{Operative:}

\begin{itemize}

   \item Der Operative ist ein ausf�hrbares Java-Programm, das beliebig
         oft pro Compute-System existiert (Gr��enordung 10 -- 100 mal). Er
         berechnet ein Teilproblem.

   \item Ein Operative verbindet sich mit einem anzugebenden
         Dispatcher �ber RMI und registriert sich �ber das Interface,
         das der Dispatcher anbietet. Informationen �ber die Adresse des
         Dispatchers werden dem Operative �ber die Kommando-Zeile
         �bergeben.

   \item Der Verf�gbarkeitsstatus eines Operatives f�r neue Teilprobleme
         wird vom Compute-Manager verwaltet. Bei der Anmeldung des Operatives
           am Dispatcher berechnet der Operative noch keine Teilaufgabe.
         Sofern der Dispatcher noch Teilprobleme hat, wird dem
         Operative unmittelbar nach der Anmeldung ein Teilproblem
         "ubertragen.

   \item Sobald der Operative die Berechnung einer Teilaufgabe 
         abgeschlossen hat, schickt er seine Teill�sung an den Dispatcher. 
         Der Dispatcher wei� jetzt, dass der Operative wieder verf�gbar
         ist und kann ihm eine neue Teilaufgabe zuweisen.

   \item Vor Beenden meldet sich der Operative beim Dispatcher ab.

\end{itemize}


\section{Problem-�bermittler:}

\begin{itemize}

   \item Er ist die Schnittstelle zwischen Benutzer und Compute-Systems. "Uber
         ihn l"auft die Kommunikation.

   \item Der Problem-�bermittler ist ein lauff�higes Java-Programm, das URL
         und entweder ``Hauptklassenname'' und serislisierbare Parameter
         oder eine serialisierbare Instanz eines speziellen Problems an den
         Dispatcher, genauer an den Problem-Manager, �bermittelt.

   \item Er bekommt die L"osung des Problems vom Dispatcher, damit sie f"ur
         den Benutzer verf"ugbar ist. Der Problem-�bermittler wartet, bis der
	 Problem-Manager das Ergebnis zur�ckgeliefert hat.
\end{itemize}

\section{Problem}

\subsection{Anforderungen an das Problem:}

Das Problem muss ein Interface implementieren, das folgende Funktionen
bietet:

\begin{itemize}
   \item Teilprobleme bereitstellen, solange keine L�sung f�r das
	 Problem erstellt werden kann

   \item Teill�sung entgegen nehmen

   \item L�sung in endlicher Zeit erstellen
\end{itemize}

\subsection{Anforderungen an die L�sung:}

Eine L�sung wird von einem Problem in endlicher Zeit erzeugt und muss an
den Problem-"Ubermittler �bermittelbar (serialisierbar) sein.

\subsection{Anforderungen an das Teilproblem:}

Teilprobleme m�ssen von der Problemklasse bereitgestellt werden und vom
Dispatcher zu den Operatives �bermittelbar (serialisierbar) sein. Sie m�ssen
ein Interface implementieren, das die Funktion

\begin{itemize}
    \item Berechne das Teilproblem und erzeuge Teill�sung
\end{itemize}

bietet.

\subsection{Anforderungen an die Teill�sungen:}

Eine Teill�sung wird von einem Teilproblem in endlicher Zeit erzeugt und muss
von Operative �ber Dispatcher zur Problemklasse �bermittelbar sein.

\section{Laufzeit-Vergleich}

Programm dient zum Vergleich eines implementierten Problems zwischen der
Berechnung auf einem einzelnen Computer und der Berechnung auf dem
Com\-pute-System.

\begin{itemize}
    \item Bei der Berechnung auf einem einzelnen Computer wird \textit{ein}
          Teilproblem erzeugt und dieses berechnet.
    \item Bei der Berechnung auf dem Compute-System wird dem
          Problem-"Uber\-mitt\-ler die Adresse des Webservers, auf dem das
          Problem liegt, der Name des Problems und die Adresse des
          Compute-Systems �bergeben.
    \item Es wird die gebrauchte Zeit, sowie die L"osung von beiden
          Berechnungen ausgegeben.
\end{itemize}


    %% Realease 1.0 dieser Software wurde am Institut f�r Intelligente Systeme der
%% Universit�t Stuttgart (http://www.informatik.uni-stuttgart.de/ifi/is/) unter
%% Leitung von Dietmar Lippold (dietmar.lippold@informatik.uni-stuttgart.de)
%% entwickelt.


\chapter{�berblick �ber das Systems, Begriffskl�rung}

\section{Name: Architeuthis}

Der d�nische Wissenschaftler Japetus Steenstrup erhielt Schnabel, Schulp und
einige Saugn�pfe eines Riesenkalmars, der ein Jahr zuvor an der d�nischen
K�ste an Land gesp�lt worden war. Er verglich das Material mit
entsprechenden Organen bekannter kleinerer Kalmararten und schloss daraus,
dass es zu einem riesigen Kalmar geh�ren m�sse, den er Architeuthis, den
ersten oder gr��ten Kalmar, nannte. Die Gattung \emph{Architeuthis}
(Steenstrup 1856) bezeichnet heute noch die Atlantischen Riesenkalmare.

Riesenkalmare geh�ren zu den Kopff��ern (Cephalopoda). Sie haben im Ganzen
zehn Arme, davon zwei lange Tentakel mit keulenf�rmig verbreiterten Enden,
die mit Saugn�pfen bewehrt sind und zum Fangen der Beute dienen. Acht kurze
Arme rund im die Mund�ffnung f�hren die Beute dem Mund zu.

Quelle: http://www.weichtiere.at/Kopffuesser/kalmar.html (Abruf 15.04.2006)


\subsection*{Aussprache}
Das Wort \emph{Architeuthis} ist aus dem Lateinischen abgeleitet, wird
deshalb auf der zweiten Silbe betont und sonst nach der im Deutschen
�blichen Sprechweise ausgesprochen. Insbesondere ist das ``ch'' kein K-Laut,
das ``eu'' wie in ``Europa'' und das zweite ``h'' stimmlos.

\section{�bersicht}

Die Architektur des \hyperref[compsys]{Compute-Systems} ist in drei Ebenen
und drei Stufen gegliedert, die im folgenden genauer erkl�rt werden:

\begin{center}
\begin{tabular}{l|c|c|c}

Stufe $\rightarrow$  &       Benutzer      & Dispatcher & Operative   \\
Ebene $\downarrow$   &        Stufe        &   Stufe    & Stufe       \\
\hline
 Anwendungs Ebene    &    Problem-Paket    &  Problem   & Teilproblem \\
                     &    Testumgebung     &  L�sung    & Teill�sung  \\
\hline
 Compute-System      & Problem-�bermittler & Dispatcher & Operative   \\
 Ebene               &  Statistik-Anzeige  &            &             \\
\hline
 Hardware Ebene & Benutzer-Rechner & Dispatcher-Rechner & Operative-Rechner \\
                &   Web-Server     &                    &                   \\
\end{tabular}
\end{center}


\section{Anwendungs Ebene}

F�r die Bedienung des Compute-Systems wird im Folgenden von mehreren 
Personen ausgegangen, die aber nicht notwendig verschieden sein m�ssen:

\begin{itemize}
    \item \hyperref[benutzer]{Benutzer}
    \item \hyperref[admin]{Dispatcher-Administrator}
    \item \hyperref[clber]{Operative-Bereitsteller}
\end{itemize}

\subsection{Benutzer Stufe}

\paragraph*{Problem-Paket}\label{probpak}
  steht f�r folgende Java-Klassen:
\begin{enumerate}
    \item {\em Problem}
    \item {\em PartialProblem (Teilproblem)}
    \item {\em PartialSolution (Teill�sung)}
\end{enumerate}

\paragraph*{Benutzer}\label{benutzer}
    startet ein Anwendungsprogramm des
    \hyperref[compsys]{Compute-Systems}. Er ist im Normalfall auch derjenige,
    der das {\em Problem-Paket} implementiert hat.

\paragraph*{Testumgebung}\label{testumg}
    stellt Tests zur �berpr�fung der Funktion und zur Messung der
    Performance des \hyperref[compsys]{Compute-Systems} bereit. F�r
    letzteres kann ein {\em Problem} jeweils auf dem
    \hyperref[benrech]{Benutzer-Rechner} und auf dem
    \hyperref[compsys]{Compute-System} berechnet und die jeweile
    Berechnungzeit verglichen werden.

\subsection{Dispatcher Stufe}

\paragraph*{Administrator}\label{admin}
    startet den Dispatcher mit der Java-Klasse {\tt ComputeManagerImpl}.

\paragraph*{Problem}\label{prob}
    bezeichnet eine Java-Klasse, die das Interface {\em Problem}
    implementiert und so mit Hilfe des \hyperref[compsys]{Compute-Systems}
    eine Berechnung vornimmt. Das Interface erfordert, dass die Java-Klasse
    dem \hyperref[compman]{Compute-Manager} {\em Teilprobleme} liefert und
    die zugeh�rigen {\em Teill�sungen} wieder zusammenfassen kann, um daraus
    die {\em Gesamtl�sung} zu erstellen.

\paragraph*{Gesamtl�sung}\label{gesl"os}
    wird vom {\em Problem} erzeugt,  nachdem alle notwendigen
    {\em Teilprobleme} berechnet worden sind. Die Gesamtl�sung muss ein
    {\em Serializable-Objekt} sein.


\subsection{Operative Stufe}

\paragraph*{Operative-Bereitsteller}\label{clber}
    startet einen Operative. Da mehere Operatives in einem Compute-System 
    arbeiten k�nnen, handelt es sich bei den Operative-Bereitstellern
    ebenfalls um mehrere Personen (die aber nicht notwendig verschieden sein
    m�ssen).

\paragraph*{Teilproblem (PartialProblem)}\label{teilprob}
    ist ein von einem {\em Problem} erzeugtes Objekt, das von einem
    \hyperref[compclie]{Operative} verarbeitet werden kann, indem es das
    Interface {\em PartialProblem} implementiert.

\paragraph*{Teill�sung (PartialSolution)}\label{teillos}
    ist ein von einem {\em Teilproblem} erzeugtes Objekt, das dem {\em
    Problem} zur�ckgegeben wird. Die Klasse muss {\em Serializable}
    implementieren.


\section{Compute-System Ebene}


\paragraph*{Compute-System}\label{compsys}
    bezeichnet die Gesamtheit aller Komponenten, die zur Berechnung eines
    parallelisierten \hyperref[prob]{Problems} ben�tigt werden, also den
    {\em Problem-�bermittler}, den {\em Dispatcher} und die {\em Operatives}.


\subsection{Benutzer Stufe}

\paragraph*{Problem-�bermittler (ProblemTransmitter)}\label{probueb}
\label{anwpr}
    ist die Software-Komponente, die Kontakt zum {\em Dispatcher} herstellt
    und das �bertragen des \hyperref[prob]{Problems} vom
    \hyperref[benutzer]{Benutzer} erm�glicht. Die zugeh�rigen Klassen sind
    \emph{ProblemTransmitterImpl} und \emph{ProblemComputation}.

\paragraph*{Statistik-Anzeige},
    stellt dem Benutzer aktuelle Informationen zum Compute-Systems insgesamt
    (\texttt{Systems-Statistik}) und zu seinem \hyperref[prob]{Problem}
    (\texttt{Problem-Statistik}) bereit. Die zugeh�rigen Klassen sind
    \emph{ProblemGUIStatisticsReader}, \emph{SystemGUIStatisticsReader} und
    \emph{SystemTextStatisticsReader}.


\subsection{Dispatcher Stufe}


\paragraph*{Dispatcher}\label{compserv}
    besteht aus drei Komponenten: Dem {\em Compute-Manager}, dem
    \emph{Prob\-lem-Manager} und der {\em Statistik}. Er ist das Herzst"uck
    des {\em Compute-Systems}. Er wird durch den
    \hyperref[admin]{\em Administrator} auf dem
    \hyperref[servrech]{\em Dispatcher-Rechner} gestartet.

\paragraph*{Problem-Manager}\label{problman}
    bezeichnet den Teil des {\em Dispatchers}, der sich um die Verwaltung
    der \hyperref[prob]{Probleme} und \hyperref[teilprob]{Teilprobleme}
    k"ummert, da dies bei mehreren {\em Problemen} notwendig ist.

\paragraph*{Compute-Manager}\label{compman}
    bezeichnet den Teil des {\em Dispatchers}, der sich um die Verwaltung
    der {\em Operatives} k"ummert und die Verteilung der
    \hyperref[teilprob]{Teilprobleme} "ubernimmt.

\paragraph*{Statistik}\label{stat}
    bezeichnet den Teil des {\em Dispatchers}, der die  statistischen Daten
    des {\em Compute-Systems} verwaltet. 


\subsection{Operative Stufe}


\paragraph*{Operative}\label{compclie}
    bekommt von dem {\em Compute-Manager} genau ein
    \hyperref[teilprob]{Teilproblem} und berechnet dieses. Die
    \hyperref[teillos]{Teill�sung} schickt er dann wieder an ihn zur�ck.


\section{Hardware Ebene}

Die Hardware unfasst die f�r das Gesamtsystem erforderlichen Rechner.
Einzelne konkrete Rechner k�nnen auch mehrere Funktionen ausf�hren (z.B.
gleichzeitig {\em Dispatcher-Rechner} und {\em Operative-Rechner} sein).


\subsection{Benutzer Stufe}

\paragraph*{Benutzer Rechner}\label{benrech}
  startet die Kommunikation zum \hyperref[compsys]{Compute-System}.

\paragraph*{Web-Server}\label{weserver}
    ist n�tig, damit das \hyperref[compsys]{Compute-System} das
    \hyperref[probpak]{Problem-Paket}  laden kann.


\subsection{Dispatcher Stufe}

\paragraph*{Dispatcher Rechner}\label{servrech}
  stellt die Hardware f�r den \hyperref[compsys]{Dispatcher} zur Verf�gung.


\subsection{Operative Stufe}

\paragraph*{Operative Rechner}\label{clierech} 
  ist mit dem {\em Dispatcher-Rechner} verbunden. Auf ihm laufen die
  eigentlichen Berechnungen ab.


    %% Realease 1.0 dieser Software wurde am Institut f�r Intelligente Systeme der
%% Universit�t Stuttgart (http://www.informatik.uni-stuttgart.de/ifi/is/) unter
%% Leitung von Dietmar Lippold (dietmar.lippold@informatik.uni-stuttgart.de)
%% entwickelt.


\chapter[Benutzung als Dispatcher -Administrator]{Die Benutzung des Systems
  als Dispatcher-Administrator}

\section{Der Dispatcher}

Der Dispatcher besteht aus drei Komponenten: Dem Com\-pute-Manager,
dem Pro\-blem-Manager und der Statistik. Er ist das Herzst"uck des 
Compute-Systems. 


\subsection{Voraussetzungen}

Ben�tigt wird zun�chst ein funktionierendes Java Runtime Environment, Version
1.4 oder neuer. Es ist eine Netzwerkanbindung n"otig, damit eine Kommunikation
zu den Operatives geschaffen werden kann. Besondere Hardware-Anforderungen
werden nicht gestellt, ein Rechner mit 600 MHz und 256 MB RAM kann problemlos 
als Dispatcher f�r �ber 30 Operatives verwendet werden.


\subsection{Benutzung}

Um den Dispatcher benutzen zu k�nnen, muss er auf einem von allen Operatives
erreichbaren Computer gestartet werden. Dieser Computer muss auch f�r
Anfragen von den Computern offen sein, die sp�ter Probleme auf den
Dispatcher �bertragen wollen. Wie sp�ter beim Operative gilt auch hier: Am
besten l�uft der Dispatcher mit angepa�ter Priorit�t im Hintergrund, falls
auf dem Rechner zus�tzlich eine interaktive Arbeit erfolgt.

\subsection[Kommandozeilenparameter]{Aufruf und Kommandozeilenparameter}
Kommandozeile zum Starten des Dispatchers (alles in einer Zeile, die
Zeilenumbr�che wurden nur zur besseren �bersicht eingef�gt):

{\tt java -Djava.rmi.server.RMIClassLoaderSpi=\\
  de.unistuttgart.architeuthis.misc.CacheFlushingRMIClSpi\\
  -Djava.util.logging.config.file=logging.properties\\
  -Djava.security.policy=dispatcher.pol\\
  de.unistuttgart.architeuthis.dispatcher.DispatcherImpl\\
  -c <config-Datei> -port <Port-Nummer>\\
  -deadtime <Zeit> -deadtries <Anzahl> -d -t -help}

Dabei ist:

\begin{description}
\item[-Djava.rmi...CacheFlushingRMIClSpi] Optional. Dadurch werden die
  Klassen von Problemen nur gecached, solange das Problem in Berechnung
  ist. Sobald die L�sung dem Benutzer zur�ck�bermittelt wurde, werden die
  dazugeh�rigen Klassen aus dem Cache gel�scht. Damit k�nnen Probleme,
  deren Klassen modifiziert wurden, dem Compute-System wieder zum 
  Berechnen �bergeben werden. Normalerweise w�rde das Java-eigene
  Cachen von Klassen dies verhindern.

  {\bf Achtung:} Selbst durch diese Massnahme kann es zu merkw�rdigen
  Effekten kommen (z.B. ClassCast-Exception), wenn mehrere Probleme
  gleichzeitig in modifizierter Form berechnet werden, oder wenn mehrere
  Probleme im gleichen Verzeichnis des Webservers bereitgestellt werden.
  Daher sollte am Einfachsten f�r jedes neue Problem auch ein neues
  Verzeichnis auf dem Webserver angelegt werden. Dann treten die oben
  angedeuteten Probleme nicht auf.

\item[-Djava.util.logging.config.file=logging.properties]
  Optional. Gibt die Konfigurationsdatei f�r den im Dispatcher 
  verwendeten Logger an. Die Konfigurationsdatei selbst ist kommentiert,
  f�r die einzelnen Optionen bitte diese einsehen. Wird keine Angabe zur 
  Konfigurationsdatei gemacht, verwendet der Dispatcher die
  Standard-Konfiguration. Weitere Informationen finden sich au�erdem in der 
  Java API-Do\-ku\-men\-ta\-tion unter \texttt{java.util.logging}.
  Der Dispatcher arbeitet im Namespace ``de.unistuttgart.architeuthis''.

  {\bf Tipp:} Normalerweise sollten im Betrieb nur Informationen der
  Informationslevel WARNING oder SEVERE oder eventuell INFO geloggt werden,
  da sonst wichtige Ausgaben eventuell �bersehen werden. Zum Testen
  empfiehlt sich manchmal, einen niedrigeren Level zu w�hlen, beispielsweise
  CONFIG.

\item[-Djava.security.policy=dispatcher.pol]
  Die policy-Datei f�r den Dispatcher. Diese Datei darf sich nicht in
  einer jar-Datei befinden; der Pfad muss also relativ oder absolut angegeben
  werden. Siehe dazu das Kapitel \hyperref[sicherh]{Sicherheit}.  

\item[de.unistuttgart.architeuthis.dispatcher.DispatcherImpl]
  Die Klasse zum Starten des Dispatchers.

\item[-c <config-Datei>]
  Optional. Es kann eine Konfigurationsdatei angegeben werden, die die
  restlichen Parameter beinhaltet. Standardm��ig ist ``compserv.conf''
  eingestellt. Falls diese Datei nicht vorhanden ist, werden Standardwerte
  verwendet.

\item[-port <Port-Nummer>]
  Optional. Setzt die Port-Num\-mer, un\-ter der die RMI-Regi\-stry des
  Dispatchers zu erreichen sein wird. Standardm��ig wird 1099 angenommen
  (wie bei jeder RMI-Kommunikation).

\item[-deadtime <Zeit>]   
  Optional. Setzt die Zeit zwischen den Versuchen, Operatives zu finden, die
  nicht mehr antworten. Eine h�here Zeit verursacht weniger Rechenlast und
  Netzwerk-Verkehr auf dem Dispatcher, jedoch werden dann fehlerhafte
  Operatives sp�ter entdeckt.

\item[-deadtries <Anzahl>]
  Optional. Setzt die Anzahl der Fehler, die sich bei einer Verbindung zu
  einem Operative ereignen d�rfen, bis dieser entfernt wird.

\item[-d] Optional. Schaltet den Debug-Modus mit zus�tzlichen Meldungen
  ein.

\item[-t] Optional. Erzeugt f�r die �bergabe und f�r den Abbruch eines
  Teilproblems an die bzw.\ auf den Operatives einen neuen Thread.

\item[-help] Optional. Gibt eine Meldung zur Benutzung aus und beendet
  anschlie�end das Programm.
\end{description}


\chapter[Benutzung als Operative-Administrator]{Die Benutzung des Systems als
  Operative-Administrator}

\section{Der Operative}
Die Operatives stellen die Rechenleistung des Compute-Systems bereit und
f�hren die tats�chlichen Berechnungen aus.


\subsection{Voraussetzungen}
Die einzige Voraussetzung zum Betreiben eines Operatives ist ein
installiertes Java-Runtime-Environment der Version 1.3 oder neuer und eine
Netzwerkanbindung. Bei Benutzung von Version 1.3 ist jedoch zu beachten,
dass dann ein nicht-cachen\-der Betrieb nicht m�glich ist (siehe Option
\hyperref[cachen]{CacheFlushingRMIClSpi}).


\subsection{Benutzung}
Die Benutzung gestaltet sich sehr einfach. Der Operative muss nur gestartet
werden, wobei als Kommandozeilenparameter die RMI-Adresse des
Compute-Mana\-gers angegeben werden muss, bei dem sich der Operative anmelden
soll. Der Operative sollte am Besten mit nicht zu hoher Priorit�t im
Hintergrund laufen, falls auf dem Rechner zus�tzlich eine interaktive Arbeit
erfolgt.

{\bf Tipp:} Falls der Compute-Manager dauerhaft l�uft, kann der Operative
problemlos beim Rechnerstart im Hintergrund gestartet werden. Das 
Herunterfahren des Rechners beendet den Operative ord\-nungs\-ge\-m��.
Ebenfalls denkbar ist es, den Operative im Hintergrund des GDM laufen zu
lassen.

{\bf Achtung:} Der Dispatcher und der Operative m�ssen gegenseitig
erreichbar sein.


\subsection[Kommandozeilenparameter]{Aufruf und Kommandozeilenparameter}
Kommandozeile zum Aufruf des Operatives:

{\tt java -Djava.security.policy=operative.pol\\
  -Djava.rmi.server.RMIClassLoaderSpi=\\
  de.unistuttgart.architeuthis.misc.CacheFlushingRMIClSpi\\
  -Djava.util.logging.config.file=logging.properties\\
  de.unistuttgart.architeuthis.operative.OperativeImpl\\
  <Adresse> -d}

Dabei ist:

\begin{description}
\item[-Djava.security.policy=operative.pol]
  Gibt die Policy-Datei f�r den Operative an. Diese Datei darf sich nicht in
  einer jar-Datei befinden; der Pfad muss also relativ oder absolut angegeben
  werden. Siehe dazu das Kapitel
  \hyperref[sicherh]{Sicherheit}.

\item[-Djava.util.logging.config.file=logging.properties]
  Optional. Gibt die Konfigurationsdatei f�r den im Operative 
  verwendeten Logger an. Die Konfigurationsdatei selbst ist kommentiert,
  f�r die einzelnen Optionen bitte diese einsehen. Wird keine Angabe zur 
  Konfigurationsdatei gemacht, verwendet der Operative die
  Standard-Konfiguration. Weitere Informationen finden sich au�erdem in der 
  Java API-Do\-ku\-men\-ta\-tion unter \texttt{java.util.logging}.
  Der Dispatcher arbeitet im Namespace ``de.unistuttgart.architeuthis''.

  {\bf Tipp:} Normalerweise sollten im Betrieb nur Informationen der
  Informationslevel WARNING oder SEVERE oder eventuell INFO geloggt werden,
  da sonst wichtige Ausgaben eventuell �bersehen werden. Zum Testen
  empfiehlt sich manchmal, einen niedrigeren Level zu w�hlen, beispielsweise
  CONFIG.

\item[-Djava.rmi...CacheFlushingRMIClSpi]
\label{cachen}
  L�dt einen anderen Service Provider f�r den RMIClassLoader. Dies bewirkt,
  dass Klassen nicht gecached werden und somit Probleme, deren Klassen
  modifiziert wurden, wieder mit dem System berechnet werden k�nnen. Zur
  Minimierung des Netzwerk-Verkehrs kann diese Option auch entfallen.

  {\bf Achtung:} Falls diese Option nicht angegeben wird, werden
  Modifikationen in Teilproblem-Klassen nur dann erkannt, falls diese Klassen
  entweder in einem anderen Verzeichnis auf dem Webserver abgelegt werden oder
  der Operative neu gestartet wird. Wenn die Option angegeben wird, d�rfen
  die class-Dateien nicht ge�ndert werden, w�hrend das Problem berechnet wird.

\item[de.unistuttgart.architeuthis.operative.OperativeImpl]
  Die Klasse zum Starten des Operatives.

\item[Adresse] 
        Die Adresse der RMI-Registry, in der der Dispatcher eingetragen ist,
        bei dem sich der Operative anmelden soll. Exemplarisch:\\
        {\tt <Rechnername>:<Port>}

        Dabei ist:

        \begin{description}
        \item[Rechnername] Der Name oder die IP-Adresse des Rechners,
            auf dem der Dispatcher l�uft.
        \item[Port] Optional. Der Port, auf dem der Dispatcher
            auf Verbindungen h�rt. Falls nicht angegeben, wird 1099
            angenommen und au�erdem muss dann der : entfallen.
        \end{description}

\item[-d] Optional. Schaltet den Debug-Modus mit zus�tzlichen Meldungen
        ein.
\end{description}


\chapter[Benutzung als Endanwender]{Die Benutzung des Systems als Endanwender}

\section{Das eigentliche Problem}
Zu implementieren sind drei Java-Klassen, die im folgenden n"aher erkl"art
werden.

Alle nachfolgend genannten Interfaces befinden sich im Package
\texttt{de...userinterfaces.develop}.


\subsection[Problem]{Das Problem - Problem.java} 

Das Problem ist die organisierende Hauptklasse. Es generiert die
Teilprobleme, empf"angt die Teill"osungen und gibt die Gesamtl"osung
zur"uck, sobald diese erzeugt werden konnte. Das Interface \texttt{Problem}
besitzt dementsprechend drei Methoden. In der Regel sollte das konkrete
Problem serialisierbar sein und das Unter-Interface
\texttt{SerializableProblem} implementieren.

%{\scriptsize \verbatiminput{java/Problem.jav}}

%{\scriptsize \verbatiminput{java/SerializableProblem.jav}}

{\bf Hinweis:}
Falls das Erstellen von Teilproblemen oder das Verarbeiten von Teill�sungen
rechenintensiv ist, ist es ratsam, daf�r selbst Teilprobleme zu generieren.

\subsection[Teilproblem]{Das Teilproblem - PartialProblem.java}
Das Teilproblem ist das eigentliche Rechenprogramm, das auf dem Operative
ausgef�hrt wird. Das Interface selbst fungiert lediglich als Ober-Interface
ohne Methoden, erweitert aber das Interface \texttt{Serializable}.

%{\scriptsize \verbatiminput{java/PartialProblem.jav}}

Es besitzt zwei Unter-Interfaces, von denen eines von einem konkreten
Teilproblem zu implementieren ist. \texttt{NonCommPartialProblem} ist f�r
Teilprobleme, die keinen gemeinsamen Speichers benutzen
(nicht-kommunizierende Teilprobleme). \texttt{CommunicationPartialProblem}
ist f�r Teilprobleme, die einen gemeinsamen Speichers benutzen
(kommunizierende Teilprobleme). Beide Unter-Interfaces besitzen nur eine
Methode \texttt{compute} und unterscheiden sich nur darin, dass diese Methode
im ersten Fall keinen und im zweiten Fall einen Parameter besitzt.

%{\scriptsize \verbatiminput{java/NonCommPartialProblem.jav}}

%{\scriptsize \verbatiminput{java/CommunicationPartialProblem.jav}}


\subsection[Teill"osung]{Die Teill"osung - PartialSolution.java}
Eine Teill"osung ist das Ergebnis des berechneten Teilproblems und muss
das Interface \texttt{PartialSolution} implementieren, das wiederum
\texttt{Serializable} erweitert.

%{\scriptsize \verbatiminput{java/PartialSolution.jav}}

\subsection{Wichtige allgemeine Hinweise}

\begin{itemize}
\item Alle in einer serialisierbaren Klasse (die \texttt{Serializable}
      implementiert) enthaltenen Attribut-Klassen m�ssen serialisierbar
      sein.

\item Wenn eine serialisierbare Klasse eine nicht-serialisierbare Oberklasse
      besitzt, muss die Oberklasse einen parameterlosen Konstruktor
      besitzten. Au�erdem muss die Unterklasse daf�r Sorge tragen, dass nach
      der Deserialisierung  einer Instanz von ihr die Attribute der
      Oberklasse die erforderlichen Werte besitzen. Weitere Hinweise zur
      Serialisierung finden sich in der API-Beschreibung zum Interface
      \texttt{java.io.Serializable}.

\item In den serialisierbaren Objekten, insbesondere in den Klassen, die
      {\tt Partial\-Problem} und {\tt Partial\-Solution} implementieren,
      d�rfen keine sta\-ti\-schen Variablen
%     \marginpar{Statische Variablen}
      verwendet werden, da diese beim Versenden der Objekte via RMI nicht
      serialisiert werden. Konstanten (also {\tt static final}-Attribute)
      d�rfen jedoch verwendet werden.

\item Generell ist es empfehlenswert, jedes neue Problem, das auf dem
      Compute-System berechnet werden soll, in einem neuen Verzeichnis auf
      dem Webserver abzulegen.

\item Sobald das Ergebnis ermittelt werden kann, sollte das {\tt Problem}
      keine weiteren {\tt Partial\-Problem}-Objekte mehr erzeugen, da diese
      sonst auch berechnet werden, ohne nach der L�sung zu fragen. Bei der
      Ankunft einer berechneten {\tt Par\-tial\-So\-lu\-tion} wird jedoch
      nach der Gesamtl�sung gefragt.

\item Die Erzeugung und das Zusammensetzen der Teilprobleme sowie das
      Berechnen der Gesamtl"osung sollte, da es auf dem Dispatcher-Rechner
      ausgef"uhrt wird, nur geringen Rechenaufwand erfordern. Falls es mehr
      Rechenaufwand erfordert, sollten diese T"atigkeiten als eigene
      Teilprobleme vergeben werden.
\end{itemize}

\subsection[Hilfestellung]{Hilfestellung bei der Implementierung mittels
                           abstrakter Klassen}
Es werden abstrakte Klassen angeboten, die das Implementieren vereinfachen
sollen. Siehe dazu \hyperref[abstrakt]{Kapitel \ref{abstrakt}
\emph{Abstrakte Hilfsklassen}}.

\subsection{Bereitstellen der Klassen mittels des ClassFileServer}
\label{classfileserver}

Wie schon erw�hnt, m�ssen alle Klassen, die zur Verarbeitung des Problems
und der Teilprobleme erforderlich sind, auf einem Webserver f�r alle
Computer des Systems erreichbar sein (also sowohl f�r Operatives, wie auch
f�r den Dispatcher). Dabei muss die benutzte Package-Hierarchie wie bei Java
�blich als Verzeichnis-Hierarchie vorhanden sein. Die Klassen k�nnen
alternativ auch in einem jar-File liegen, das ebenfalls die
Package-Hierarchie nachbildet.

Falls kein Webserver zur Verf�gung steht, kann der sogenannte
{\tt Class\-File\-Server} benutzt werden. Dies ist ein von Sun geschriebener
Mini-Webserver, der nur dazu dient, Klassen an RMI-Applikationen
auszuliefern. Die Benutzung ist sehr einfach. Man starte den
ClassFileServer mit folgender Kommandozeile:

{\tt java de.unistuttgart.architeuthis.facade.ClassFileServer <port> <root>}

Dabei ist:

\begin{description}
\item[port] Der Port auf dem der Webserver ansprechbar sein soll.
\item[root] Das Verzeichnis, das als Wurzel-Verzeichnis f�r die
    auszuliefernden Dateien des Web\-servers dienen soll.
\end{description}

Danach sind unter {\tt http://<rechnername>:<port>/} alle Dateien und
Verzeichnisse unterhalb von {\tt<root>} erreichbar.

{\bf Achtung:}
Der ClassFileServer macht alle alle Dateien (nicht nur class-Dateien)
unterhalb des angegebenen Verzeichnisses zug�nglich. Mas sollte daher f�r
die class-Dateien ein eigenes Verzeichnis erzeugen.


\section{Die Problem-Verarbeitung}
Zur Verarbeitung eines (serialisierbaren) Problems steht als Schnittstelle
zwischen Benutzer und Compute-System die Klasse
{\tt de...facade.ProblemCom\-putation} zur Verf�gung. Diese bieten zwei
Arten von Methoden:

\begin{description}
\item[transmitProblem] Diese Methode �bertr�gt ein �bergebenes
  serialisierbares Problem zur Verarbeitung an den Dispatcher. Dabei ist
  zumindest der Rechnername des Dispatcher anzugeben. Au�erdem ist eine
  \emph{codebase}, d.h.\ ein URL, unter dem die vom Problem und von den
  Teilproblemen ben�tigten Klassen abrufbar sind, oder ein Array von
  codebases entweder zu �bergeben oder �ber das Property
   \texttt{java.rmi.server.codebase} beim Aufruf der JVM anzugeben. In
  jedem Fall kann zus�tzlich (optional) noch eine Instanz von
  {\tt RemoteStoreGenerator} �bergeben werden.

\item[computeProblem] Diese Methode berechnet ein �bergebenes
  serialisierbares Probem lokal, d.h.\ ohne �bertragung an den
  Dispatcher. Diese Methode ist besonders in der Phase der Entwicklung
  n�tzlich, da z.B.\ Testausgaben m�glich sind. Der Methode kann optional
  die vorzuschlagende Anzahl der Teilprobleme und au�erdem noch optional
  eine Instanz von {\tt RemoteStoreGenerator} �bergeben werden.
\end{description}

Die Verwendung der beiden Arten von Methoden ist ansonsten gleich und
unterscheidet sich nicht vom Aufruf von Methoden, die vom Benutzer selbst
implementiert wurden und lokal ausgef�hrt werden.

Sollen mehrere Probleme nacheinander verarbeitet werden, sollte von der
oben genannten Klasse aus Effizienzgr�nden trotzdem nur eine Instanz erzeugt
werden.


\section{Der Problem-�bermittler}
\label{ProblemTransmitterImpl}

F�r den Ausnahmefall, dass ein Problem nicht serialisierbar ist, kann das
Problem vom Compute-Manager geladen und vollst�ndig auf dem Compute-System
ausgef�hrt werden. Dazu steht die Klasse
\texttt{de...facade.ProblemTransmit\-terImpl} zur Verf�gung. Sie bietet
au�erdem, auch bei der �bertragung einer Instanz eines serialisierbaren
Problems, die M�glichkeit, w�hrend der Berechnung des Problems auf dem
Compute-System nebenl�ufig eine Problem-Statistik oder eine System-Statistik
abzufragen oder die Berechnung des Problems abzubrechen. Ihre Verwendung
hat jedoch einige Nachteile:

\begin{itemize}
\item  Sie bietet keine Methode zur lokalen Berechnung eines Problems.

\item  F�r die Berechnung eines nicht-serialisierbaren Problems kann die
       zu erzeugende Instanz nur durch den Konstruktor konfiguriert werden
       und als Parameter k�nnen nur Objekte angegeben werden (keine Werte
       elementaren Typs).

\item  F�r die Berechnung eines nicht-serialisierbaren Problems muss au�erdem
       der Klassenname als String angegeben werden, wodurch der Compiler
       keine entsprechenden Pr�fungen mehr vornehmen kann.
\end{itemize}

Zur �bertragung eines Problems mittels \texttt{ProblemTransmitterImpl} muss
von dieser Klasse unter Angabe des Rechnernames des Dispatchers eine
Instanz erzeugt werden. Die Klasse bietet mehrere Methoden mit dem Namen
{\tt transmitProblem}. Bei einem nicht-serialisierbaren Problem, von dem
eine Instanz auf dem Dispatcher erzeugt wird, ist ein Array mit den
(serialisierbaren) Parameter-Objekten f�r den Konstruktor anzugeben.

Als Beispiel f�r die Verwendung von {\tt ProblemTransmitterImpl}  kann die
Klasse \texttt{de...testenvironment.prime.advanced.PrimeNumbersParallel}
zusam\-men mit den anderen Klassen dieses Packages betrachtet werden.


\section{Ausf�hrung eines eigenst�ndigen Problems}

F�r den Sonderfall, dass ein Problem komplett eigenst�ndig ist und keine
Daten mit anderen Programmen austauscht und keine Eingaben vom Benutzer
ben�tigt, steht die Klasse \texttt{de...facade.ProblemTransmitterApp} zur
Verf�gung. Der Vorteil bei der Benutzung liegt darin, da� zu einem
Problem keine extra Klasse zur Ausf�hrung des Problems erstellt werden
braucht und da� automatisch eine Problem- und eine System-Statistik
angezeigt werden kann.

Die Kommandozeile zum Ausf�hren der Klasse ist:

{\tt java -Djava.security.policy=transmitter.pol\\
  de.unistuttgart.architeuthis.facade.ProblemTransmitterApp\\
  -r <ProblemManager> -u <packageURL> -c <klassenname>\\
  -f <dateiname> -s -d -n -p}

\begin{description}
\item[-Djava.security.policy=transmitter.pol]
  Gibt die Po\-li\-cy-Da\-tei f�r den Pro\-blem-�ber\-mit\-tler an. Diese
  Datei darf sich nicht in einer jar-Datei befinden; der Pfad muss also
  relativ oder absolut angegeben werden. Siehe dazu das Kapitel
  \hyperref[sicherh]{Sicherheit}.

\item[de.unistuttgart.architeuthis.facade.ProblemTransmitterApp]
  Die Haupt-Klasse der Kommandozeilen-Applikation.

\item[-r <ProblemManager>]
        Die Adresse der RMI-Registry, in der der
        Dispatcher eingetragen ist, an den sich die Kommandozeilen-Applikation
        wenden soll. Exemplarisch:

         {\tt <Rechnername>:<Port>}

        Dabei ist:

        \begin{description}
        \item[Rechnername] Der Name oder die IP-Adresse des Rechners,
            auf dem der Dispatcher l�uft.
        \item[Port] Optional. Der Port, auf dem der Dispatcher auf
            Verbindungen h�rt. Falls nicht angegeben, wird 1099 angenommen
            und au�erdem muss der Doppelpunkt dann entfallen.
        \end{description}

\item[-u <packageURL>] Der URL des �u�ersten Pakets der Klassen des
  Problems. Die Adresse muss entweder mit ``/'' enden, falls die Klassen
  ungepackt in der Package-Hierarchie vorliegen, oder mit dem Namen der
  jar-Datei, die die Klassen in der Package-Hierarchie enth�lt. Anstatt
  einen vorhandenen Web-Server kann man auch
  den \hyperref[classfileserver]{ClassFileServer} verwenden (s.\ Abschnitt
  \ref{classfileserver}).

\item[-c <klassenname>] Der vollst�ndige Klassenname (inklusive aller Packages)
  der Klasse des Problems, die das Interface {\tt Problem} implementiert.

{\bf Achtung:}
\begin{itemize} 
\item Da bei der �bermittlung durch die Kommandozeilen-Applikation dem Problem
      keine Parameter �bergeben werden k�nnen, ist es notwendig, dass das
      Problem einen parameterlosen Konstruktor implementiert, der m�gliche
      Attribute mit den ben�tigten Werten initialisiert.

\item Die Unterverzeichnisse, die durch die Package-Struktur entstehen,
      m�ssen auf dem Webserver ebenfalls vorhanden sein.\\
      Beispiel: Die Klasse {\tt MyProblem} aus dem Package {\tt mypackage}
      ist unter der URL\\ {\tt
      http://myserver/mydir/my\-pack\-age/My\-Prob\-lem\-.class} erreichbar.
      Dann muss der Parameter {\tt -u} den Wert {\tt
      http://my\-ser\-ver/my\-dir/} erhalten und der Parameter {\tt -c} den
      Wert {\tt my\-pack\-age.My\-Prob\-lem}.
\end{itemize}

\item[-f <dateiname>] Der Name der Datei, in der die L�sung des Problems
  gespeichert werden soll. Die L�sung wird dabei einfach in ihrer
  serialisierten Form in die Datei geschrieben. Wird kein Dateiname angegeben,
  wird die L�sung auf der Standard-Ausgabe ausgegeben.

\item[-s] Optional. �bertr�gt das Problem als serialisierbares Problem.

\item[-d] Optional. Schaltet zus�tliche Debug-Meldungen ein.

\item[-n] Optional. Schaltet die beiden graphischen Statistik--Fenster ab.
          Nach der Berechnung wird eine Statistik in Textform ausgegeben.

\item[-p] Optional. Startet nur das Fenster f�r die Problem--Statistik, nicht
          jedoch das f�r die System--Statistik.
\end{description}

Das Schlie�en der beiden graphischen Statistik--Fenster hat keinen Einflu�
auf die Berechnung des Problems. Die Berechnung des Problems kann jedoch
durch {\tt Strg-C} im Fenster, in dem das Problem gestartet wurde,
abgebrochen werden.


\section{Der Laufzeitvergleich}

Die Klasse \texttt{de...facade.RuntimeComparison} dient zum Vergleich
zwischen der Berechnung eines Problems auf dem lokalen Computer und der
Berechnung auf dem Compute-System. Sie gibt jeweils die ben�tigte Zeit und
die L"osung von beiden Berechnungen aus oder speichert die L"osungen in
Dateien.

\subsection{Durchf�hrung der Berechnungen}

Die Berechnung auf einem einzelnen Computer ist so realisiert, dass dem
Problem mitgeteilt wird, dass nur ein Operative zur Verf"ugung steht. Ob
dann nur ein Teilproblem generiert wird h"angt von der Implementierung des
Problems ab. Die Berechnung erfolgt nur durch den lokalen Computer.

Bei der verteilte Berechnung wird das Problem an einen Dispatcher �bergeben
und durch alle verf�gbaren Operatives berechnet.

\subsection[Kommandozeilenparameter]{Aufruf und Kommandozeilenparameter}

Die Kommandozeile zum Aufrufen der Testumgebung ist:

{\tt java -Djava.security.policy=transmitter.pol\\
  de.unistuttgart.architeuthis.facade.RuntimeComparison\\
  -r <ProblemManager> -u <packageURL>\\
  -c <klassenname> -f <dateiname> -d}

Dabei ist:

\begin{description}
\item[-Djava.security.policy=transmitter.pol]
  Die f�r die Testumgebung zu benutzende policy-Datei. Diese Datei darf
  sich nicht in einer jar-Datei befinden; der Pfad muss also relativ
  oder absolut angegeben werden. Siehe Kapitel \hyperref[sicherh]{Sicherheit}.

\item[de.unistuttgart.architeuthis.facade.RuntimeComparison]
  Die Haupt-Klasse des Laufzeitvergleichs.

\item[-r <computesystem>] 
    Die Adresse der RMI-Registry, in der der Dispatcher eingetragen ist,
    an die sich der Laufzeitvergleich wenden soll. Exemplarisch:

    {\tt <Rechnername>:<Port>}

    Dabei ist:

    \begin{description}
    \item[Rechnername] Der Name oder die IP-Adresse des Rechners,
        auf dem der Dispatcher l�uft.
    \item[Port] Optional. Der Port, auf dem der Dispatcher auf
        Verbindungen h�rt. Falls nicht angegeben, wird 1099 angenommen und
        au�erdem muss dann der Doppelpunkt entfallen.
    \end{description}

\item[-u <classURL>] Der URL des �u�ersten Pakets der Klassen des
  Problems. Die Adresse muss entweder mit ``/'' enden, falls die Klassen
  ungepackt in der Package-Hierarchie vorliegen, oder mit dem Namen der
  jar-Datei, die die Klassen in der Package-Hierarchie enth�lt. Anstatt
  einen vorhandenen Web-Server kann man auch
  den \hyperref[classfileserver]{ClassFileServer} verwenden (s.\ Abschnitt
  \ref{classfileserver}).

\item[-c <klassenname>] Der vollst�ndige Klassenname (inklusive aller Packages)
  der Klasse des Problems, die das Interface Problem implementiert.

  {\bf Achtung:} Die Unterverzeichnisse, die durch die Package-Struktur
  entstehen, m�ssen auf dem Webserver ebenfalls vorhanden sein.\\
  Beispiel: Die Klasse {\tt MyProblem} aus dem Package {\tt mypackage} ist
  unter der URL
  {\tt http://myserver/mydir/my\-pack\-age/My\-Prob\-lem\-.class} erreichbar.
  Dann muss der Parameter \texttt{-u} den Wert
  {\tt http://my\-ser\-ver/my\-dir/} erhalten und der Parameter \texttt{-c}
  den Wert {\tt my\-pack\-age.My\-Prob\-lem}.

\item[-f <dateiname>] Optional. Der Name der Datei, in der die L�sung des
  Problems gespeichert werden soll. Die L�sung wird dabei einfach in ihrer
  serialisierten Form in die Datei geschrieben. Wird kein Dateiname angegeben,
  wird die L�sung auf der Standard-Ausgabe ausgegeben.

\item[-d] Optional. Schaltet zus�tliche Debug-Meldungen ein.
\end{description}


\section{Die Statistik}

Mit der Statistik k�nnen Informationen �ber den Zustand des Compute-Systems
abgefragt werden. Diese werden entweder in einem eigenen Fenster oder
textuell ausgegeben.

\subsection{Voraussetzungen}
Die einzige Voraussetzung zum Anzeigen der Statistik ist ein installiertes
Java-Runtime-Environment der Version 1.3 oder neuer und eine
Netzwerkanbindung oder  ein lokales Arbeiten auf dem Dispatcher-Rechner.
F�r die graphische Statistik-Ausgabe ist eine funktionierende graphische
Oberfl�che erforderlich.

\subsection[Kommandozeilenparameter]{Aufruf und Kommandozeilenparameter}

Aufruf der Version mit eigenem Fenster:

{\tt java -Djava.security.policy=statisticreader.pol\\
     de.unistuttgart.architeuthis.facade.SystemGUIStatisticsReader\\
     <computeSystem>}

Aufruf der Konsolen-Version:

{\tt java -Djava.security.policy=statisticreader.pol\\
     de.unistuttgart.architeuthis.facade.SystemTextStatisticsReader\\
     <computeSystem>}

\begin{description}
\item[-Djava.security.policy=statisticreader.pol]
  Gibt die Po\-li\-cy-Da\-tei f�r die Sta\-tistik-Anzeige an. Diese Datei 
  darf sich nicht in einer jar-Datei befinden; der Pfad muss also relativ
  oder absolut angegeben werden. Siehe dazu das Kapitel
  \hyperref[sicherh]{Sicherheit}.

\item[de.unistuttgart.architeuthis.facade.SystemGUIStatisticsReader]
  Die Haupt-Klasse der Statistik-Anzeige in einem eigenen Fenster.

\item[de.unistuttgart.architeuthis.facade.SystemTextStatisticsReader]
  Die Haupt-Klasse der textuellen Statistik-Anzeige.

\item[<computeSystem>]
    Die Adresse der RMI-Registry, in der der Dispatcher eingetragen ist,
    an die sich die Statistik wenden soll. Exemplarisch:

     {\tt <Rechnername>:<Port>}

    Dabei ist:

    \begin{description}
    \item[Rechnername] Der Name oder die IP-Adresse des Rechners,
        auf dem der Dispatcher l�uft.
    \item[Port] Optional. Der Port, auf dem der Dispatcher auf
        Verbindungen h�rt. Falls nicht angegeben, wird 1099 angenommen
        und au�erdem muss dann der : entfallen.
    \end{description}
\end{description}


    %% Realease 1.0 dieser Software wurde am Institut f�r Intelligente Systeme der
%% Universit�t Stuttgart (http://www.informatik.uni-stuttgart.de/ifi/is/) unter
%% Leitung von Dietmar Lippold (dietmar.lippold@informatik.uni-stuttgart.de)
%% entwickelt.


\chapter{Gemeinsamer Speicher}

\section{Vorwort}

Dieses Kapitel beschreibt die M�glichkeit, f�r die Teilprobleme einen
gemeinsamen Speicher zu nutzen und neue Arten von gemeinsamen Speicher zu
entwickeln.


\subsection{Hinweise f�r den Benutzer}

F�r die Benutzung eines gemeinsamen Speichers sind drei Interfaces von
Bedeutung:

\begin{itemize}
\item  \texttt{de...userinterfaces.develop.RemoteStoreGenerator}
       definiert die Klasse, die die Instanzen vom gemeinsamen Speicher
       erzeugt.

\item  \texttt{de...userinterfaces.develop.RemoteStore}
       kennzeichnet den gemeinsamen Speicher.

\item  \texttt{de...userinterfaces.develop.CommunicationPartialProblem}
       definiert ein Teilproblem als eines, das den gemeinsamen Speicher
       benutzen kann.
\end{itemize}

Damit die Teilprobleme einen gemeinsamen Speicher verwenden k�nnen, ist
folgendes zu tun:

\begin{itemize}
\item  Die Teilprobleme m�ssen das Interface \texttt{CommunicationPartialProblem}
       implementieren.

\item  Es mu� eine konkrete Klasse erstellt werden, die das Interface
       \texttt{RemoteStore} implementiert.

\item  Es mu� eine konkrete Klasse erstellt werden, die das Interface\\
       \texttt{RemoteStoreGenerator} implementiert. In der Regel wird diese
       Klasse durch ein zus�tzliches Interface beschrieben.
\end{itemize}

Im Package \texttt{de.unistuttgart.architeuthis.remotestore} sind schon
Klassen, die die Interfaces \texttt{RemoteStore} und \texttt{RemoteStoreGenerator}
implementieren, vorhanden und k�nnen benutzt werden. Die Entwicklung eigener
Klassen wird im n�chsten Abschnitt beschrieben.

F�r die Kommunikation der Teilprobleme mit dem gemeinsamen Speicher scheinen
folgende drei M�glichkeiten sinnvoll:

\begin{itemize}
\item  Es ist f�r alle Teilprobleme nur ein zentraler Speicher vorhanden.
       Das Lesen aus diesem Speicher und das Schreiben in diesen Speicher
       erfolgt \emph{synchron}.

\item  Es ist ein zentraler Speicher und zus�tzlich auf jedem Operation ein
       Speicher vorhanden. Das Lesen der Teilprobleme aus dem Speicher
       erfolgt dabei \emph{asynchron}. F�r das Schreiben gibt es beide
       M�glicheiten.

       \begin{itemize}
       \item  Das Schreiben in den Speicher erfolgt \emph{synchron} (der
              Aufruf einer Methode zum Schreiben durch ein Teilproblem ist
              erst abgeschlossen, wenn die Methode auch bei allen anderen
              Speichern aufgerufen wurde).

       \item  Das Schreiben in den Speicher erfolgt \emph{asynchron} (der
              Aufruf einer Methode zum Schreiben durch ein Teilproblem ist
              sofort abgeschlossen und die Methode wird bei allen anderen
              Speichern anschlie�end. d.h.\ nebenl�ufig zur weiteren
              Berechnung des Teilproblems, aufgerufen).
       \end{itemize}
\end{itemize}

Ein synchrones Lesen bzw.\ Schreiben bedeutet, da� zum Beginn des Lesens
bzw.\ zum Abschlu� des Schreibens der Zustand des Speichers f�r alle
Teilprobleme identisch ist. Ein asynchrones Lesen bzw.\ Schreiben ist jedoch
in der Regel effizienter.

Es ist jedoch zu beachten, da� die Reihenfolge des Aufrufens von Methoden
des Speichers durch die Teilprobleme aber in allen F�llen unbestimmt ist (so
wie die Reihenfolge des Eintreffens von Teill�sungen beim Problem unbestimmt
ist).

In jedem Fall sollte ein gemeinsamer Speicher aber folgende Bedingungen
sicherstellen:

\begin{itemize}
\item  Wenn ein Teilproblem ein Objekt in den Speicher schreibt und
       dieses Teilproblem anschlie�end aus dem Speicher lie�t, sollte das
       vorherige Schreiben schon erfolgt sein.

\item  Mehrere Aufrufe zur Ver�nderung des Speichers durch ein Teilproblem
       sollten in den Speicher f�r alle anderen Teilprobleme in der gleichen
       Reihenfolge vorgenommen werden.
\end{itemize}

Die vorhandenen Implementierungen vom gemeinsamen Speicher erf�llen diese
Bedingungen und bieten alle drei Arten der Kommunikation. Die zu verwendende
Art der Kommunikation wird beim Aufruf des Konstruktors der Klasse, die das
Interface \texttt{RemoteStoreGenerator} implementiert, angegeben.


\subsection{Hinweise f�r den Entwickler}

Text zu erg�nzen.


    %% Realease 1.0 dieser Software wurde am Institut f�r Intelligente Systeme der
%% Universit�t Stuttgart (http://www.informatik.uni-stuttgart.de/ifi/is/) unter
%% Leitung von Dietmar Lippold (dietmar.lippold@informatik.uni-stuttgart.de)
%% entwickelt.


\chapter[Beispiel zur Parallelisierung]{Der Weg zum parallelisierten Programm
         anhand eines Beispiels}

Dieses Kapitel beschreibt die einfache Parallelisierung eines vorgegebenen
Programms.


\section{Aufgabenstellung}

Man betrachte zun�chst die folgende Methode {\tt primzahlTeilbereich} aus
\texttt{de.unistuttgart.architeuthis.testenvironment.PrimeNumbers}.

{\scriptsize \verbatiminput{java/primzahlTeilbereich.jav}}

Ermittelt werden sollen also alle Primzahlen, die in einem Intervall von
zwei Zahlen liegen. Da dies so geschieht, da� die Zahlen einzeln �berpr�ft
werden, ob sie jeweils eine Primzahl sind, sind die �berpr�fungen
unabh�ngig voneinander. Die Gesamtaufgabe kann also recht einfach in
Teilaufgaben zerlegt werden, indem das Gesamtintervall in Teilintervalle
zerlegt wird. Jedes Teilintervall bildet dann ein Teilproblem und eine
Teill�sung ist eine Liste der Primzahlen aus dem Teilintervall. Die
Gesamtl�sung ergibt sich dann durch hintereinander h�ngen der Listen
der Teill�sungen. Dabei ist zu beachten, da� die Primzahlen in der
Gesamtliste aufsteigend sortiert sein m�ssen.


\section{Implementierung}

Die Sortierung der Primzahlen in der Gesamtliste kann dadurch implizit
erfolgen, da� den Teilproblemen die Teilintervalle aufsteigend zugeordnet
werden (wie es naheliegend ist) und die Teill�sungen (die Listen der
Primzahlen) in der entsprechenden Reihenfolge aneinander geh�ngt werden.
Da die Berechnung der Teilprobleme unabh�ngig voneinander ist, k�nnen sie
au�erdem alle geleichzeitig erzeugt werden.

Die R�ckgabe der Teill�sungen in der Reihenfolge der Ausgabe der Teilprobleme
wird durch die zur Verf�gung stehenden abstrakten Hilfsklassen
(s.\ Kapitel \ref{abstrakt}) erledigt. Da die Teilprobleme au�erdem alle
gleichzeitig erzeugt werden k�nnen, kann die Klasse
\texttt{AbstractFixedSizeProblem} verwendet werden. F�r diese m�ssen
die folgenden beiden Methoden implementiert werden:

\begin{itemize}
\item protected PartialProblem[] createPartialProblems(int parProbsSuggested)
\item protected Serializable createSolution(PartialSolution[] partialSolutions)
\end{itemize}

Deren Implementierung erfolgt wie schon beschrieben. In
\texttt{createPartialProblems} werden die Teilprobleme dadurch erzeugt,
da� jedem Teilproblem ein Teilintervall gleicher Gr��e zugeordnet wird. In
\texttt{createSolution} werden dann alle Teill�sungen in ihrer Reihenfolge
im Array aneinandergeh�ngt (zum konkreten Typ der Elemente des Array s.u.).
Das Gesamtintervall wird durch die Unter- und Obergrenze dem Konstruktor
des Problems �bergeben. Nachfolgend die vollst�ndige Implementierung der
Problem-Klasse.

{\scriptsize \verbatiminput{java/basic/PrimeRangeProblemImpl.jav}}

Die Implementierung des Teilproblems erfolgt einfach dadurch, da� dem
Konstruktor das zu durchsuchende Teilintervall �bergeben wird und
in der Methode \texttt{compute()} zur Berechnung der Primzahlen die
vorgegebene Methode \texttt{primzahlTeilbereich} aufgerufen wird. Um die
Liste als Teill�sung zur�ckzugeben, wird die zur Verf�gung stehende
Hilfsklasse \texttt{ContainerPartialSolution} verwandt. Aus dieser mu�
die Liste dann mit der schon oben genannten Methode \texttt{createSolution}
abgerufen werden. Nachfolgend die vollst�ndige Implementierung der
Teilproblem-Klasse.

{\scriptsize \verbatiminput{java/basic/PrimePartialProblemImpl.jav}}


\section{Ausf�hrung}

Die Ausf�hrung bzw.\ Berechnung des Problems erfolgt in drei einfachen
Schritten:

\begin{enumerate}
\item  Es wird eine Instanz der Klasse \texttt{de...user.ProblemComputation}
       erzeugt.

\item  Es wird eine Instanz des Problems erzeugt. Diese wird in der Praxis
       in der Regel durch ein Programm, das die Parameter des Problems
       errechnet und das Ergebnis (die L�sung) des Problems weiterverwendet,
       erzeugt werden.

\item  Es wird die Methode \texttt{transmitProblem} der Instanz von
       \texttt{ProblemComputation} aufgerufen und ihr die Instanz vom
       Problem �bergeben. Au�erdem ist der Methode der Name des
       Dispatcher-Rechners und der URL, unter dem die class-Dateien
       abrufbar sind (z.B.\ mittels des ClassFileServer, s.\ Abschnitt
       \ref{classfileserver}), zu �bergeben. Insbesondere zu Testzwecken
       und zur Berechnung unabh�ngig von einem laufenden Compute-System
       kann alternativ auch die Methode \texttt{computeProblem} aufgerufen
       werden.
\end{enumerate}

Die nachfolgend angegebene Klasse
\texttt{de...testenvironment.prime.basic.GeneratePrimes} kann zur
Verdeutlichung des Vorgehens dienen.

{\scriptsize \verbatiminput{java/basic/GeneratePrimes.jav}}

Die Klasse besitzt als Beispiel lediglich eine einzelne
\texttt{main}-Methode, die die Parameter f�r das Problem als
Kommandozeilen-Argumente einliest. Der Name des Dispatchers und der URL
f�r die class-Dateien sind als Konstanten definiert. Sie h�tten nat�rlich
z.B.\ ebenfalls als Kommandozeilen-Argumente eingelesen werden k�nnen.
Von der Instanz der Klasse \texttt{ProblemComputation} h�tte abschlie�end
noch die finale Statistik zum Problem abgerufen (mittels der Methode
\texttt{getFinalProblemStat}) und ausgegeben werden k�nnen.

Wichtig ist, das f�r die JVM dem Property \texttt{java.security.policy}
die policy-Datei zugewiesen wird, indem auf der Kommandozeile f�r den
Befehl \texttt{java} die entsprechende policy-Datei angegeben wird. Ein
konkreter Aufruf des obigen Programms zur Ausf�hrung auf dem Compute-System
k�nnte unter Linux also folgenderma�en aussehen (alles in einer Zeile):

\texttt{java -cp ../deploy/User.jar:../deploy/Problems.jar\\
        -Djava.security.policy=../config/transmitter.pol\\
        de.unistuttgart.architeuthis.testenvironment.prime.basic.GeneratePrimes\\
        200000 201000 r}

Unter Windows m�ssen in den Pfadangaben die Slashes durch Backslashes
erseztzt und anstatt des Doppelpunktes ein Semikolon angegeben werden.


\section{Weitere Beispiele}

Weiter Beispiele sind im Package \texttt{de.unis.architeuthis.testumgebung}
zu finden. Hier eine kurze Auflistung der Probleme und jeweils eine kurze
Beschreibung dazu.

\begin{enumerate}
\item \texttt{de...testenvironment.prime.basic.PrimRangeProblemImpl}\\
	Das ist das obige Beispiel dieses Kapitels mittels der abstrakten
        Klasse \texttt{AbstractFixedSizeProblem}.

\item \texttt{de...testenvironment.prime.advanced.PrimRangeProblemImpl}\\
        Dieses Problem macht das gleiche wie das obige Problem, nur da�
        die Aufteilung in Teilprobleme geschickter so erfolgt, da� der
        Rechenaufwand f�r die einzelnen Teilprobleme noch gleicher ist.

\item \texttt{de...testenvironment.prime.advanced.PrimSequenceProblemImpl}\\
	Dies ist das Beispiel f"ur Fortgeschrittene im n"achsten Kapitel.

\item \texttt{de...testenvironment.montecarlo.MonteCarloProblemImpl}\\
        Hier wird das MonteCarlo-Verfahren zur Bestimmung der Zahl $Pi$
        parallelisert. Man bekommt somit bei einer festgelegten Rechenzeit
        eine bessere Genauigkeit f"ur $Pi$.

\item \texttt{de...testenvironment.random.RandomProblemImpl}\\
        Hier wartet jedes Teilproblem einfach eine zuf"allige Anzahl von
        Sekunden. Dies ist f"ur Testzwecke recht interessant.

\item \texttt{de...testenvironment.fail.FailProblemImpl}\\
        Ein Problem f�r Testzwecke, bei dem die Abfrage nach einer L�sung
        immer den Wert \texttt{null} liefert.

\item \texttt{de...testenvironment.caching.CachingTestProblem}\\
        Hier werden drei Dummy-Klassen in den Teilproblemen geladen. Mit
        diesem Problem wurde verglichen, inwiefern sich das Laden von
        Klassen zeitlich bemerkbar macht. Siehe dazu auch das Kapitel
        \hyperref[performance]{Performance}.

\item \texttt{de...testenvironment.hashstore.HashStoreProblemImpl}\\
        Bei diesem Problem benutzen die Teilprobleme das Package
        \texttt{de...remotestore.hashmap} als gemeinsamen Speicher.
\end{enumerate}


    %% Realease 1.0 dieser Software wurde am Institut f�r Intelligente Systeme der
%% Universit�t Stuttgart (http://www.informatik.uni-stuttgart.de/ifi/is/) unter
%% Leitung von Dietmar Lippold (dietmar.lippold@informatik.uni-stuttgart.de)
%% entwickelt.


\chapter[Beispiel f�r Fortgeschrittene]{Beispiel f�r Fortgeschrittene}

Dieses Kapitel beschreibt die Parallelisierung eines Problems, das sich nicht
ganz einfach wie im vorherigen Kapitel parallelisieren l�st. Als Alternative
zum vorherigen Kapitel soll das Problem au�erdem als nicht-serialisiertes
Problem verarbeitet werden.


\section{Aufgabenstellung}

Es soll die folgende Methode \texttt{primzahlTeilfolge} der Klasse
\texttt{de.unistuttgart.architeuthis.testenvironment.PrimeNumbers}
parallelisiert werden.

{\scriptsize \verbatiminput{java/primzahlTeilfolge.jav}}

Im Gegensatz zur Methode \texttt{primzahlTeilbereich} aus dem vorherigen
Kapitel kann das Intervall der Methode nicht einfach in Teilintervalle
aufgeteilt und Teilproblemen zugeordnet werden. Welche Zahlen zu einem
Teilintervall in Bezug auf die Primzahl-Eigenschaft zu untersuchen w�ren
w�rde n�mlich von der Anzahl der Primzahlen in den vorhergehenden
Teilintervallen abh�ngen. Die Teilprobleme k�nnten nur noch sequentiell
erzeugt und verarbeitet werden.

Daher wird das gegebene Problem in ein anderes Problem transformiert, das
sich gut parallelisieren l�st, konkret in das Problem aus dem vorhergehenden
Kapitel. Dazu werden alle Primzahlen bis zum einem Wert ermittelt, der
gr��er (aber m�glichst nicht viel gr��er) ist als die gr��te Primzahl aus
dem vorgegebenen Intervall. Die Absch�tzung dieser Zahl erfolgt nach der
Formel von Rosser und Schoenfeld (siehe: J.\ B.\ Rosser and L.\ Schoenfeld.
\emph{Approximate formulas for some functions of prime numbers}. Illinois
Journal of Mathematics, 6:64--94, 1962), auf die hier nicht n�her
eingegangen wird. Die Primzahlen aus den ermittelten Listen zu den
Teilintervalle werden dann der Reihe nach gez�hlt und die mit den Nummern
aus dem vorgegebenen Intervall werden als Ergebnis geliefert.


\section{Implementierung}

Um die Problem-Klasse (\texttt{PrimeSequenceProblemImpl} genannt) zu
implementieren, k�nnte man sich wie beschrieben auf die Berechnung einer
Obergrenze des Intervalls, das der Klasse \emph{PrimeRangeProblemImpl}
�bergeben wird, beschr�nken (als Untergrenze ist Null, Eins oder Zwei
zu �bergeben) und aus der von dieser Klasse erzeugten Liste
von Primzahlen die ben�tigten extrahieren. In der erzeugten Liste von
Primzahlen k�nnten aber extrem viel mehr Primzahlen enthalten sein, als
anschlie�end ben�tigt w�rden, wenn die Gr��e des Intervalls klein ist (im
Extremfall z.B.\ nur aus einer Zahl besteht) gegen�ber den absoluten Werten
der Intervallgrenzen. Daher sollen von den Teilintervallen die Primzahlen
nur gespeichert werden, wenn ihre Nummern im vorgegebenen Intervall liegen.
Ansonsten soll nur die Anzahl der Primzahlen gespeichert werden, die bereits
berechnet wurden. Um dies zu erreichen m�ssen die einzelnen Teill�sungen so
bald wie m�glich (unter Beachtung der Reihenfolge der Teill�sungen)
geliefert werden. Dies tut gerade die abstrakte Klasse
\texttt{AbstractOrderedProblem} (s.\ Abschnitt
\ref{AbstractOrderedProblem}), von der \texttt{PrimeSequenceProblemImpl}
also erben sollte.

Dabei sind folgende beiden Methoden zu implementieren:

\begin{itemize}
\item protected PartialProblem createPartialProblem(int parProbsSuggested)
\item protected Serializable receivePartialSolution(PartialSolution parSol)
\end{itemize}

In \texttt{createPartialProblem} wird beim ersten Aufruf die Obergrenze des
Intervalls ermittelt, das anschlie�end zur Erzeugung einer Instanz von
\texttt{PrimeRangeProblemImpl} (einer etwas verbesserten Version der Klasse
aus dem vorherigen Kapitel) verwendet wird. Mit dieser wird dann ein
Array von Teilproblemen erzeugt. Anschlie�end und bei jedem
weiteren Aufruf liefert die Methode dann das jeweils n�chste Teilproblem.

In \texttt{receivePartialSolution} werden die Teill�sungen entgegengenommen
und verarbeitet. Bei den Teill�sungen handelt es sich wie im vorhergehenden
Kapitel um Instanzen von \texttt{ContainerPartialSolution}, in denen eine
Liste der Primzahlen (der Teilintervalle) gespeichert ist, da ja wie im
vorhergehenden Kapitel die gleichen Teilprobleme verwendet werden. Die
Primzahlen der Teilintervalle (die entsprechend den Teilintervallen ja
aufsteigend geordnet geliefert werden) werden gez�hlt und die Primzahlen aus
dem zu ermittelnden Bereich werden gespeichert. Wenn alle Primzahlen aus dem
vorgegebenen Intervall berechnet wurden, liefert die Methode die Liste
dieser Primzahlen, ansonsten liefert sie den Wert \texttt{null}.

Die vollst�ndige Klasse ist nachfolgend angegeben.

{\scriptsize \verbatiminput{java/advanced/PrimeSequenceProblemImpl.jav}}


\section{Ausf�hrung}

Zur Ausf�hrung bzw.\ Berechnung des Problems kann wie im vorherigen Kapitel
beschrieben die Klasse \texttt{ProblemComputation} benutzt werden. Als
Alternative soll hier aber die Benutzung der Klasse
\texttt{de...user.ProblemTransmitterImpl} dargestellt werden (s.a.\ Abschnitt
\ref{ProblemTransmitterImpl}).

Von dieser Klasse ist unter Angabe des Namens des Dispatchers eine Instanz
zu erzegen. Diese bieten dann  u.a.\ eine Methode \texttt{transmitProblem},
der ein URL, unter dem die class-Dateien abrufbar sind, der Name der
Problem-Klasse und die Parameter f�r den Konstruktor der Problem-Klasse
�bergeben werden. Als Ergebnis liefert die Methode die Gesamtl�sung. Zur
Verdeutlichung kann die nachfolgend zusammen mit einigen Konstanten
angegebene Methode \texttt{primzahlTeilfolge} dienen, die bis auf die
zus�tzlichen Exceptions die gleiche Signatur wie die am Anfang des Kapitels
angegebene Methode hat.

{\scriptsize \verbatiminput{java/advanced/primzahlTeilfolgeRemote.jav}}

Zum Aufruf dieser Methode kann wie im vorhergehenden Kapitel eine Klasse
\texttt{GeneratePrimes} verwendet werden. Sie ist in gleicher Weise (mit
den gleichen Parametern, bis auf den anzugebenden Buchstaben) aufzurufen.
Auch diese ist zur Verdeutlichung nachfolgend angegeben.

{\scriptsize \verbatiminput{java/advanced/GeneratePrimes.jav}}


    %% Realease 1.0 dieser Software wurde am Institut f�r Intelligente Systeme der
%% Universit�t Stuttgart (http://www.informatik.uni-stuttgart.de/ifi/is/) unter
%% Leitung von Dietmar Lippold (dietmar.lippold@informatik.uni-stuttgart.de)
%% entwickelt.


\chapter{Performance}
\label{performance}

Die folgenden Werte beziehen sich auf die Version 0.9 des Systems und sind
f�r die aktuelle Version nur noch begrenzt aussagekr�ftig.

\section{Einf�hrung}
In diesem Kapitel sind die Ergebnisse einiger Untersuchungen festgehalten,
denen das Compute-System unterzogen wurde, um herauszufinden, wie hoch der
Geschwindigkeitszuwachs bei mehreren Operatives ist, aber auch, wie hoch der
Zeitverlust ist, der sich durch das Verteilen der Teilprobleme ergibt.

\section{Testkonfiguration}
Die Tests wurden in einem Computer-Pool durchgef�hrt, der w�hrend der
Durchf�hrung der Tests auch von anderen Leuten benutzt wurde. Alle genannten
Zeiten verstehen sich, falls nicht anders angegeben, in Sekunden und sind
gerundete Mittelwerte mehrerer Messreihen eines bereits benutzten
Compute-Systems (siehe dazu ``Der erste Start'').

\subsection{Dispatcher}

Hardware des Dispatchers:

  \begin{itemize}
  \item Dual-Athlon 1800+
  \item 1 GB RAM
  \item Red Hat Linux 9
  \item Sun JRE Version 1.4.2
  \end{itemize}

\subsection{Operatives}

Hardware der Operatives:

  \begin{itemize}
  \item Pentium III 600 MHz
  \item 256 MB RAM
  \item Red Hat Linux 7.3
  \item Sun JRE Version 1.4.0
  \end{itemize}

\subsection{Testprobleme}
\subsubsection{Prim-Problem}
F�r das Prim-Problem ist die Aufgabenstellung: Berechne die 2.000.000te
Primzahl. Verwendet wurde die Klasse {\tt GeneratePrimes} aus dem Package
\texttt{de.unistuttgart.architeuthis.testenvironment.prime}.

\subsubsection{Caching-Problem}
Hier ist die Aufgabe ebenfalls: Berechne die 2.000.000te Primzahl. Au�erdem
werden jedoch drei Klassen zus�tzlich vom Webserver geladen, um zu testen,
wie stark sich das Caching der Klassen auf die Performance auswirkt.
Verwendet wurde die Klasse {\tt CachingTestProblem} in\\
{\tt de.unistuttgart.architeuthis.testenvironment.caching}.

\subsection{Referenzwerte}
Folgender Wert ist als Referenzwert zu betrachten: Die lokale Berechnung des
Prim-Problems auf einem Operative-Rechner ben�tigt 307,6s. Dies definiert
einem Speed-Up von 1. Wenn man das Prim-Problem auf einem Rechner verteilt
berechnet (also mit einem Dispatcher und einem Operative) berechnet, dann
ergibt sich eine Dauer von 391,2s, also ein ``Speed-Up'' von 0,79.

\begin{table*}[htbp]
  \centering
  \begin{tabular}{l l l}
  Beschreibung                  & Dauer in Sekunden  & Speed-Up\\ \hline
  Berechnung mit lokaler Klasse &      307,6         & 1       \\
  Verteilte Berechnung auf nur einem Rechner & 391,2 & 0,79    \\
  \end{tabular}
  \caption{Referenzwerte}
\end{table*}


\section{Testergebnisse}

\subsection{Verschiedene Anzahlen von Operatives}
Dieser Test soll Aufschluss �ber die effektive Geschwindigkeitssteigerung
geben, die bei Verwendung mehrerer Operatives zu erwarten ist. Verwendet
wurde das Prim-Problem.

\begin{table*}[htbp]
  \centering
  \begin{tabular}{l l l}
  Anzahl Operatives & Gesamtdauer & Speed-Up\\
  \hline
  25 & 20,9 & 14,7 \\
  20 & 25,0 & 12,3\\
  15 & 33,2 & 9,3\\
  10 & 44,7 & 6,9\\
  5  & 84,2 & 3,6\\
  3  &133,6 & 2,3\\
  1  &349,2 & 0,88\\
  \end{tabular}
  \caption{Prim-Problem ohne Caching auf Operativeseite}
  
\end{table*}

\begin{table*}[htbp]
  \centering
  \begin{tabular}{l l l}
  Anzahl Operatives & Gesamtdauer & Speed-Up\\
  \hline
  25 & 20,9 & 14,7 \\
  20 & 24,8 & 12,4\\
  15 & 32,6 & 9,4\\
  10 & 46,6 & 6,6\\
  5  & 86,4 & 3,6\\
  3  &138,2 & 2,2\\
  1  &360,9 & 0,85\\
  \end{tabular}
  \caption{Prim-Problem mit aktiviertem Caching auf Operativeseite}
\end{table*}

Aus den Messwerten ist zu entnehmen, dass die Geschwindigkeit nahezu linear
mit der Anzahl der Operatives skaliert.

\subsection{Einfluss des Caching}

\subsubsection{Test mit Caching von Teilproblem-Klassen}

Wie oben bereits ersichtlich ist, beeinflusst das Caching bei gro�en
Rechenzeiten pro Teilproblem und wenigen zu ladenden Klassen die
Geschwindikeit kaum. Um einen Unterschied genauer zu untersuchen, wurde das
Caching-Problem verwandt. Bei diesem Test wurden 15 Operatives verwendet.

\begin{table*}[htbp]
  \centering
  \begin{tabular}{l l}
  Status & Dauer in Millisekunden\\ \hline
  Caching aktiviert & 1182\\
  Caching deaktiviert & 2239\\
  \end{tabular}
  \caption{Berechnung des Caching-Problems mit 15 Operatives}
\end{table*}

Man sieht, dass bei kurzen Berechnungszeiten der Teilprobleme und vielen zu
ladenden Klassen das Caching einen Geschwindigkeitsvorteil bringt, da dann
das Laden der Klassen �berproportional viel Zeit ben�tigt.

\subsubsection{Test mit Caching von Teill�sungs-Klasse}

Ein weiterer Test zum Caching bestand darin, 3 Probleme, die ebenfalls die 
2.000.000te Primzahl berechnen, zu vergleichen. Zus�tzlich zum Prim- und zum
Caching-Problem wurde ein MyInteger-Problem geschrieben, dass die L�sung in
eine eigene Klasse, die ebenfalls vom Webserver geladen werden muss,
schreibt. Die Ergebnisse sind in den Tabellen \ref{Tab-Caching-2.1} und
\ref{Tab-Caching-2.2} angegeben.

\begin{table*}[htbp]
\centering
\begin{tabular}{l l l l}
Anzahl der Operatives & Prim-Problem & Caching-Problem & MyInteger-Problem \\
\hline
25 & 17  & 18  & 17  \\
20 & 21  & 22  & 23  \\
15 & 27  & 27  & 27  \\
10 & 41  & 41  & 42  \\
5  & 82  & 82  & 82  \\
3  & 133 & 132 & 131 \\
1  & 346 & 346 & 346 \\
\end{tabular}
\caption{Unterschiede zwischen den Problemen mit Cache}
\label{Tab-Caching-2.1}
\end{table*}

\begin{table*}[htbp]
\centering
\begin{tabular}{l l l l}
Anzahl der Operatives & Prim-Problem & Caching-Problem & MyInteger-Problem \\
\hline
25 & 22  & 25  & 22  \\
20 & 23  & 26  & 27  \\
15 & 31  & 32  & 32  \\
10 & 43  & 43  & 43  \\
5  & 85  & 82  & 81  \\
3  & 133 & 133 & 133 \\
1  & 350 & 345 & 346 \\
\end{tabular}
\caption{Unterschiede zwischen den Problemen ohne Cache}
\label{Tab-Caching-2.2}
\end{table*}

\newpage

\subsubsection{Speedup}

Als Referenzwerte dienten die Werte aus Tabelle \ref{Tab-Caching-3.1}.

\begin{table*}[htbp]
\centering
\begin{tabular}{l | l}
Prim-Problem & 304,6 \\
Caching-Problem & 305,2 \\
MyInteger-Problem & 310,3 \\
\end{tabular}
\caption{Referenzwerte zur Berechnung des Speedup}
\label{Tab-Caching-3.1}
\end{table*}

Der Speedup ist in den Tabellen \ref{Tab-Caching-3.2}, \ref{Tab-Caching-3.3}
und \ref{Tab-Caching-3.4} angegeben.

\begin{table*}[htbp]
\centering
\begin{tabular}{l l l}
Anzahl der Operatives & Prim-Problem & Prim-Problem \\
& mit Cache & ohne Cache \\
\hline
25 & 17,92 & 13,85 \\
20 & 14,51 & 13,24 \\
15 & 11,28 & 9,83  \\
10 & 7,43  & 7,08  \\
5  & 3,72  & 3,58  \\
3  & 2,29  & 2,29  \\
1  & 0,88  & 0,87  \\
\end{tabular}
\caption{Speedup beim Prim-Problem}
\label{Tab-Caching-3.2}
\end{table*}

\begin{table*}[htbp]
\centering
\begin{tabular}{l l l}
Anzahl der Operatives& Caching-Problem & 
Caching-Problem \\
& mit Cache & ohne Cache \\
\hline
25 & 17,24 & 12,41\\
20 & 14,10 & 11,93\\
15 & 11,49 & 9,70 \\
10 & 7,57  & 7,22 \\
5  & 3,78  & 3,78 \\
3  & 2,35  & 2,33 \\
1  & 0,90  & 0,90 \\
\end{tabular}
\caption{Speedup bez�glich Caching von Teilproblemen}
\label{Tab-Caching-3.3}
\end{table*}

\begin{table*}[htbp]
\centering
\begin{tabular}{l l l}
Anzahl der Operatives & MyInteger-Problem & MyInteger-Problem \\
& mit Cache & ohne Cache\\
\hline
25 & 17,95 & 13,87 \\
20 & 13,27 & 11,30 \\
15 & 11,30 & 9,54  \\
10 & 7,27  & 7,10  \\
5  & 3,72  & 3,77  \\
3  & 2,33  & 2,29  \\
1  & 0,88  & 0,88  \\
\end{tabular}
\caption{Speedup bez�glich Caching von Teill�sungen}
\label{Tab-Caching-3.4}
\end{table*}

\subsection{Der erste Start}

W�hrend der Tests stellten wir fest, dass aufgrund einiger Eigenarten des
RMI-Systems die erste Eingabe eines Problems in das Compute-Systems stets l�nger
dauert als die folgenden Eingaben (selbst bei abgeschaltetem Caching). Dies
h�ngt vor allem mit der Benutzung der Socket-Verbindungen zusammen, da diese
unter RMI wiederverwendet werden, sobald einmal eine Verbindung hergestellt
wurde. Um diesen Einfluss zu testen, wurden die folgenden Testreihen
durchgef�hrt.

\begin{table*}[htbp]
  \centering
  \begin{tabular}{l l}
  Status                              & Dauer in Millisekunden\\
  \hline
  Erster Start, Caching aktiviert     & 7197\\
  Weitere Starts, Caching aktiviert   & 1166\\
  \hline
  Erster Start, Caching deaktiviert   & 7457\\
  Weitere Starts, Caching deaktiviert & 2239\\
  \end{tabular}
  \caption{Berechnung des Caching-Problems mit 15 Operatives}
\end{table*}

\begin{table*}[htbp]
  \centering
  \begin{tabular}{l l}
  Status                              & Dauer in Millisekunden\\
  \hline
  Erster Start, Caching aktiviert     & 31,0\\
  Weitere Starts, Caching aktiviert   & 20,4\\
  \hline
  Erster Start, Caching deaktiviert   & 29,3\\
  Weitere Starts, Caching deaktiviert & 21,2\\
  \end{tabular}
  \caption{Berechnung des Prim-Problems mit 25 Operatives}
\end{table*}


%    \printindex

\end{document}

