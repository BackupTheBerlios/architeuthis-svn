%% Realease 1.0 dieser Software wurde am Institut f�r Intelligente Systeme der
%% Universit�t Stuttgart (http://www.informatik.uni-stuttgart.de/ifi/is/) unter
%% Leitung von Dietmar Lippold (dietmar.lippold@informatik.uni-stuttgart.de)
%% entwickelt.


\chapter[Beispiel f�r Fortgeschrittene]{Beispiel f�r Fortgeschrittene}

Dieses Kapitel beschreibt die Parallelisierung eines Problems, das sich nicht
ganz einfach parallelisieren l�st. Als Alternative zum vorherigen Kapitel
soll das Problem au�erdem als nicht-serialisiertes Problem verarbeitet
werden.


\section{Aufgabenstellung}

Es soll die folgende Methode \texttt{primzahlTeilfolge} der Klasse
\texttt{de.unistuttgart.architeuthis.testenvironment.PrimeNumbers}
parallelisiert werden.

{\scriptsize \verbatiminput{java/primzahlTeilfolge.jav}}

Im Gegensatz zur Methode \texttt{primzahlTeilbereich} aus dem vorherigen
Kapitel kann das angegebene Intervall nicht einfach in Teilintervalle
aufgeteilt und Teilproblemen zugeordnet werden. Welche Zahlen zu einem
Teilintervall in Bezug auf die Primzahl-Eigenschaft zu untersuchen w�ren
w�rde n�mlich von der Anzahl der Primzahlen in den vorhergehenden
Teilintervallen abh�ngen. Die Teilprobleme k�nnten nur noch sequentiell
erzeugt und verarbeitet werden.

Daher wird das gegebene Problem in ein anderes Problem transformiert, das
sich gut parallelisieren l�st, konkret in das Problem aus dem vorhergehenden
Abschnitt. Dazu werden alle Primzahlen bis zum einem Wert ermittelt, der
gr��er (aber m�glichst nicht viel gr��er) ist als die gr��te Primzahl aus
dem vorgegebenen Intervall. Die Absch�tzung dieser Zahl erfolgt nach der
Formel von Rosser und Schoenfeld (siehe: J.\ B.\ Rosser and L.\ Schoenfeld.
\emph{Approximate formulas for some functions of prime numbers}. Illinois
Journal of Mathematics, 6:64--94, 1962), auf die hier nicht n�her
eingegangen wird. Die Primzahlen aus den ermittelten Listen der
Teilintervalle werden dann der Reihe nach gez�hlt und die mit den Nummern
aus dem vorgegebenen Intervall werden als Ergebnis geliefert.


\section{Implementierung}

Um die Problem-Klasse (\texttt{PrimeSequenceProblemImpl} genannt) zu
implementieren, k�nnte man sich wie beschrieben auf die Berechnung einer
Obergrenze des Intervalls, das der Klasse \emph{PrimeRangeProblemImpl}
�bergeben wird, beschr�nken und aus der von dieser Klasse erzeugten Liste
von Primzahlen die ben�tigten extrahieren. In der ereugten Liste von
Primzahlen k�nnten aber extrem viel mehr Primzahlen enthalten sein, als
anschlie�end ben�tigt w�rden, wenn die Gr��e des Intervalls klein ist (im
Extremfall z.B.\ nur aus einer Zahl besteht) gegen�ber den absoluten Werten
der Intervallgrenzen. Daher sollen von den Teilintervallen die Primzahlen
nur gespeichert werden, wenn ihre Nummern im vorgegebenen Intervall liegen.
Ansonsten soll nur die Anzahl der Primzahlen gespeichert werden, die bereits
berechnet wurden. Um dies zu erreichen m�ssen die einzelnen Teill�sungen so
bald wie m�glich (unter Beachtung der Reihenfolge der Teill�sungen)
geliefert werden. Dies tut gerade die abstrakte Klasse
\texttt{AbstractOrderedProblem} (s.\ Abschnitt
\ref{AbstractOrderedProblem}), von der \texttt{PrimeSequenceProblemImpl}
also erben sollte.

Dabei sind folgende beiden Methoden zu implementieren:

\begin{itemize}
\item protected PartialProblem createPartialProblem(int parProbsSuggested)
\item protected Serializable receivePartialSolution(PartialSolution parSol)
\end{itemize}

In \texttt{parProbsSuggested} wird beim ersten Aufruf die Obergrenze des
Intervalls ermittelt, das anschlie�end zur Erzeugung einer Instanz von
\texttt{PrimeRangeProblemImpl} (einer etwas verbesserten Version der Klasse
aus dem vorherigen Kapitel) verwendet wird. Anschlie�end und bei jedem
weiteren Aufruf liefert die Methode dann das n�chste Teilintervall.

In \texttt{receivePartialSolution} werden die Teill�sungen entgegengenommen
und verarbeitet. Bei den Teill�sungen handelt es sich wie im vorhergehenden
Kapitel um Instanzen von \texttt{ContainerPartialSolution}, in denen eine
Liste der Primzahlen (der Teilintervalle) gespeichert ist, da ja wie im
vorhergehenden Kapitel die gleichen Teilprobleme verwendet werden. Die
Primzahlen der Teilintervalle (die entsprechend den Teilintervallen ja
aufsteigend geordnet geliefert werden) werden gez�hlt und die Primzahlen aus
dem zu ermittelnden Bereich werden gespeichert. Wenn alle Primzahlen aus dem
vorgegebenen Intervall berechnet wurden, liefert die Methode die Liste
dieser Primzahlen, ansonsten liefert sie den Wert \texttt{null}.

Die vollst�ndige Klasse ist nachfolgend angegeben.

{\scriptsize \verbatiminput{java/advanced/PrimeSequenceProblemImpl.jav}}


\section{Ausf�hrung}

Zur Ausf�hrung bzw.\ Berechnung des Problems kann wie im vorherigen Kapitel
beschrieben die Klasse \texttt{ProblemComputation} benutzt werden. Als
Alternative soll hier aber die Benutzung der Klasse
\texttt{de...user.ProblemTransmitterImpl} dargestellt werden (s.a.\ Abschnitt
\ref{ProblemTransmitterImpl}).

Von dieser Klasse ist unter Angabe des Namens des Dispatchers eine Instanz
zu erzegen. Diese bieten dann  u.a.\ eine Methode \texttt{transmitProblem},
der ein URL, unter dem die class-Dateien abrufbar sind, der Name der
Problem-Klasse und die Parameter f�r den Konstruktor der Problem-Klasse
�bergeben werden. Als Ergebnis liefert die Methode die Gesamtl�sung. Zur
Verdeutlichung kann die nachfolgend zusammen mit einigen Konstanten
angegebene Methode \texttt{primzahlTeilfolge} dienen, die bis auf die
zus�tzlichen Exceptions die gleiche Signatur wie die am Anfang des Kapitels
angegebene Methode hat.

{\scriptsize \verbatiminput{java/advanced/primzahlTeilfolgeRemote.jav}}

Zum Aufruf dieser Methode kann wie im vorhergehenden Kapitel eine Klasse
\texttt{GeneratePrimes} verwendet werden. Sie ist in gleicher Weise (mit
den gleichen Parametern, bis auf den anzugebenden Buchstaben) aufzurufen.
Auch diese ist zur Verdeutlichung nachfolgend angegeben.

{\scriptsize \verbatiminput{java/advanced/GeneratePrimes.jav}}

