%% Realease 1.0 dieser Software wurde am Institut f�r Intelligente Systeme der
%% Universit�t Stuttgart (http://www.informatik.uni-stuttgart.de/ifi/is/) unter
%% Leitung von Dietmar Lippold (dietmar.lippold@informatik.uni-stuttgart.de)
%% entwickelt.


\chapter{Gemeinsamer Speicher}

\section{Vorwort}

Dieses Kapitel beschreibt die M�glichkeit, f�r die Teilprobleme einen
gemeinsamen Speicher zu nutzen und neue Arten von gemeinsamen Speicher zu
entwickeln.


\subsection{Hinweise f�r den Benutzer}

F�r die Benutzung eines gemeinsamen Speichers sind drei Interfaces von
Bedeutung:

\begin{itemize}
\item  \texttt{de.unistuttgart.architeuthis.userinterfaces.develop.RemoteStoreGenerator}
       definiert die Klasse, die die Instanzen vom gemeinsamen Speicher
       erzeugt.

\item  \texttt{de.unistuttgart.architeuthis.userinterfaces.develop.RemoteStore}
       kennzeichnet den gemeinsamen Speicher.

\item  \texttt{de.unistuttgart.architeuthis.userinterfaces.develop.CommunicationPartialProblem}
       definiert ein Teilproblem als eines, das den gemeinsamen Speicher
       benutzen kann.
\end{itemize}

Damit die Teilprobleme einen gemeinsamen Speicher verwenden k�nnen, ist
folgendes zu tun:

\begin{itemize}
\item  Die Teilprobleme m�ssen das Interface \texttt{CommunicationPartialProblem}
       implementieren.

\item  Es mu� eine konkrete Klasse erstellt werden, die das Interface
       \texttt{RemoteStore} implementiert.

\item  Es mu� eine konkrete Klasse erstellt werden, die das Interface
       \texttt{RemoteStoreGenerator} implementiert. In der Regel wird diese
       Klasse durch ein zus�tzliches Interface beschrieben.
\end{itemize}

Im Package \texttt{de.unistuttgart.architeuthis.remotestore} sind schon
Klassen, die die Interfaces \texttt{RemoteStore} und \texttt{RemoteStoreGenerator}
implementieren, vorhanden und k�nnen benutzt werden. Die Entwicklung eigener
Klassen wird im n�chsten Abschnitt beschrieben.

F�r die Kommunikation der Teilprobleme mit dem gemeinsamen Speicher scheinen
folgende drei M�glichkeiten sinnvoll:

\begin{itemize}
\item  Es ist f�r alle Teilprobleme nur ein zentraler Speicher vorhanden.
       Das Lesen aus diesem Speicher und das Schreiben in diesen Speicher
       erfolgt \emph{synchron}.

\item  Es ist ein zentraler Speicher und zus�tzlich auf jedem Operation ein
       Speicher vorhanden. Das Lesen der Teilprobleme aus dem Speicher
       erfolgt dabei \emph{asynchron}. F�r das Schreiben gibt es beide
       M�glicheiten.

       \begin{itemize}
       \item  Das Schreiben in den Speicher erfolgt \emph{synchron} (der
              Aufruf einer Methode zum Schreiben durch ein Teilproblem ist
              erst abgeschlossen, wenn die Methode auch bei allen anderen
              Speichern aufgerufen wurde).

       \item  Das Schreiben in den Speicher erfolgt \emph{asynchron} (der
              Aufruf einer Methode zum Schreiben durch ein Teilproblem ist
              sofort abgeschlossen und die Methode wird bei allen anderen
              Speichern anschlie�end. d.h.\ nebenl�ufig zur weiteren
              Berechnung des Teilproblems, aufgerufen).
       \end{itemize}
\end{itemize}

Ein synchrones Lesen bzw.\ Schreiben bedeutet, da� zum Beginn des Lesens
bzw.\ zum Abschlu� des Schreibens der Zustand des Speichers f�r alle
Teilprobleme identisch ist. Ein asynchrones Lesen bzw.\ Schreiben ist jedoch
in der Regel effizienter.

Es ist jedoch zu beachten, da� die Reihenfolge des Aufrufens von Methoden
des Speichers durch die Teilprobleme aber in allen F�llen unbestimmt ist (so
wie die Reihenfolge des Eintreffens von Teill�sungen beim Problem unbestimmt
ist).

In jedem Fall sollte ein gemeinsamer Speicher aber folgende Bedingungen
sicherstellen:

\begin{itemize}
\item  Wenn ein Teilproblem ein Objekt in den Speicher schreibt und
       dieses Teilproblem anschlie�end aus dem Speicher lie�t, sollte das
       vorherige Schreiben schon erfolgt sein.

\item  Mehrere Aufrufe zur Ver�nderung des Speichers durch ein Teilproblem
       sollten in den Speicher f�r alle anderen Teilprobleme in der gleichen
       Reihenfolge vorgenommen werden.
\end{itemize}

Die vorhandenen Implementierungen vom gemeinsamen Speicher erf�llen diese
Bedingungen und bieten alle drei Arten der Kommunikation. Die zu verwendende
Art der Kommunikation wird beim Aufruf des Konstruktors der Klasse, die das
Interface \texttt{RemoteStoreGenerator} implementiert, angegeben.


\subsection{Hinweise f�r den Entwickler}

Text zu erg�nzen.

