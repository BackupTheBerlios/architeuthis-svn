%% Realease 1.0 dieser Software wurde am Institut f�r Intelligente Systeme der
%% Universit�t Stuttgart (http://www.informatik.uni-stuttgart.de/ifi/is/) unter
%% Leitung von Dietmar Lippold (dietmar.lippold@informatik.uni-stuttgart.de)
%% entwickelt.


\chapter{Gemeinsamer Speicher}
\label{gemeinSpeicher}

\section{Vorwort}

Dieses Kapitel beschreibt die M�glichkeit, f�r die Teilprobleme einen
gemeinsamen Speicher zu nutzen und neue Arten von gemeinsamen Speicher zu
entwickeln.


\section{Hinweise f�r den Benutzer}

F�r die Benutzung eines gemeinsamen Speichers sind drei Interfaces von
Bedeutung:

\begin{itemize}
\item  \texttt{de...userinterfaces.develop.RemoteStoreGenerator}
       definiert die Klasse, die die Instanzen des gemeinsamen Speichers
       erzeugt.

\item  \texttt{de...userinterfaces.develop.RemoteStore}
       kennzeichnet den gemeinsamen Speicher.

\item  \texttt{de...userinterfaces.develop.CommunicationPartialProblem}
       definiert ein Teilproblem als eines, das den gemeinsamen Speicher
       benutzen kann.
\end{itemize}

Damit die Teilprobleme einen gemeinsamen Speicher verwenden k�nnen, ist
folgendes zu tun:

\begin{itemize}
\item  Die Teilprobleme m�ssen das Interface \texttt{CommunicationPartialProblem}
       implementieren.

\item  Es mu� eine konkrete Klasse erstellt werden, die das Interface
       \texttt{RemoteStore} implementiert. In der Regel wird diese Klasse
       durch ein zus�tzliches Interface beschrieben, �ber das ein Teilproblem
       auf den gemeinsamen Speicher zugreift.

\item  Es mu� eine konkrete Klasse erstellt werden, die das Interface\\
       \texttt{RemoteStoreGenerator} implementiert.
\end{itemize}

Im Package \texttt{de.unistuttgart.architeuthis.remotestore} sind schon
Klassen, die die Interfaces \texttt{RemoteStore} und
\texttt{RemoteStoreGenerator} implementieren, vorhanden und k�nnen benutzt
werden. Die Entwicklung eigener Klassen wird im n�chsten Abschnitt
beschrieben.

F�r die Kommunikation der Teilprobleme mit dem gemeinsamen Speicher scheinen
folgende drei M�glichkeiten sinnvoll:

\begin{itemize}
\item  Es ist f�r alle Teilprobleme nur ein zentraler Speicher vorhanden.
       Das Lesen aus diesem Speicher und das Schreiben in diesen Speicher
       erfolgt \emph{synchron}.

\item  Es ist ein zentraler Speicher und zus�tzlich auf jedem Operation ein
       dezentraler Speicher vorhanden. Das Lesen der Teilprobleme aus dem
       (dezentralen) Speicher erfolgt dabei \emph{asynchron}. F�r das
       Schreiben gibt es beide M�glicheiten.

       \begin{itemize}
       \item  Das Schreiben in den (zentralen) Speicher erfolgt
              \emph{synchron} (der Aufruf einer Methode zum Schreiben durch
              ein Teilproblem ist erst abgeschlossen, wenn die Methode auch
              bei allen anderen Speichern aufgerufen wurde).

       \item  Das Schreiben in den (zentralen) Speicher erfolgt
              \emph{asynchron} (der Aufruf einer Methode zum Schreiben durch
              ein Teilproblem ist sofort abgeschlossen und die Methode wird
              bei allen anderen Speichern anschlie�end. d.h.\ nebenl�ufig
              zur weiteren Berechnung des Teilproblems, aufgerufen).
       \end{itemize}
\end{itemize}

Ein synchrones Lesen bzw.\ Schreiben bedeutet, da� zum Beginn des Lesens
bzw.\ zum Abschlu� des Schreibens der Zustand des Speichers f�r alle
Teilprobleme identisch ist. Ein asynchrones Lesen bzw.\ Schreiben ist jedoch
in der Regel effizienter.

Es ist jedoch zu beachten, da� die Reihenfolge des Aufrufens von Methoden
des Speichers durch die Teilprobleme in allen F�llen unbestimmt ist (so
wie die Reihenfolge des Eintreffens von Teill�sungen beim Problem unbestimmt
ist).

In jedem Fall sollte ein gemeinsamer Speicher aber folgende Bedingungen
sicherstellen:

\begin{itemize}
\item  Wenn ein Teilproblem ein Objekt in den Speicher schreibt und dieses
       Teilproblem anschlie�end aus dem Speicher lie�t, sollte das vorherige
       Schreiben schon erfolgt sein.

\item  Mehrere Aufrufe zur Ver�nderung des Speichers durch ein Teilproblem
       sollten im Speicher f�r alle anderen Teilprobleme in der gleichen
       Reihenfolge vorgenommen werden.
\end{itemize}

Die vorhandenen Implementierungen vom gemeinsamen Speicher erf�llen diese
Bedingungen und bieten alle drei Arten der Kommunikation. Die zu verwendende
Art der Kommunikation wird beim Aufruf des Konstruktors der Klasse, die das
Interface \texttt{RemoteStoreGenerator} implementiert, angegeben.


\section{Hinweise f�r den Entwickler}

Die Entwicklung eines Remote-Store soll anhand des Beispiels vom Package
\texttt{de...remotestore.hashset} beschrieben werden. Die jeweiligen
Klassen aus diesem Package werden nachfolgend in Klammern angegeben.

Im Haupt-Package sollte nur ein Interface zur Benutzung durch das Teilproblem
(\texttt{UserRemoteHashSet}) und die Implementierung von
\texttt{RemoteStoreGenerator} (\texttt{RemoteHashSetGenerator}) enthalten
sein. Die weiteren Interfaces und die konkreten Klassen sollten sich in
Unter-Packages befinden.

Es sind zwei weitere Interfaces sinnvoll, eines f�r den zentralen Speicher
mit den Methoden zum Aufruf durch den dezentalen Speicher
(\texttt{interf.RelayHashSet}) und eines f�r den dezentralen Speicher mit den
Methoden zum Aufruf durch den zentalen Speicher (\texttt{interf.LocalHashSet}).
Entsprechend werden die Implementierungen der Interfaces im folgenden als
Relay-Store und Local-Store bezeichnet.

Zum asynchronen �bertragen von Daten vom Local-Store zum Relay-Store sind
einige Hilfsklassen vorhanden. Die �bertragung selbst kann durch die Klasse
\texttt{de...remotestore.Transmitter} erfolgen. F�r jede beim Relay-Store
aufzurufende Methode ist eine Unterklasse von
\texttt{de...remotestore.TransmitObject} zu erzeugen. Die zu �bertragenden
Objekte sind an den \texttt{Transmitter} zu �bergeben, der sie seinerseits
an eine Instanz von \texttt{de...remotestore.TransmitProcedure} (im Beispiel
\texttt{HashSetTransProc}) �bergibt. In dieser Prozedur kann f�r die Objekte
entsprechend ihrer Klasse die jeweils zugeh�rige Methode aufgerufen werden.
Wenn einer Methode mehrere Objekte zu �bergeben sind, sind diese vorher in
der Instanz einer neuen Klasse zusammenzufassen (wie in
\texttt{de...remotestore.hashmap.impl.MapEntry}).

In den Methoden vom Relay-Store (\texttt{RelayHashSetImpl}) werden die
entsprechenden Methoden aller Local-Stores aufgerufen. Bei der asynchronen
Kommunikation braucht jedoch die Methode desjenigen Local-Store, der die
Methode beim Relay-Store aufgerufen hat, nicht aufgerufen zu werden. Der
Local-Store mu� die Daten dann selbst lokal speichern.

Im Remote-Store (\texttt{RemoteHashSetImpl}), der sowohl das Interface f�r
das Teilproblem wie das Interface vom Local-Store implementiert, sind
entsprechend der beiden Aufgaben der Klasse (Interaktion mit dem Teilproblem
und Interaktion mit dem Relay-Store) zwei Arten der Synchronisation
erforderlich. Die eine dient zur Synchronisation des Zugriffs auf die
eigentlichen Daten der Datensturktur (im Beispiel auf das Attribut
\texttt{hashSet}), die andere dient zur Synchronisation des Zugriffs auf
den Relay-Store bzw.\ auf das Attribut, in dem er gespeichert ist
(\texttt{relayHashSet}). Die erste Art kann durch die Sychronisation der
jeweiligen Methode (also auf das Objekts \texttt{this}) geschehen,
f�r die andere sollte ein extra Objekt angelegt werden (im Beispiel
\texttt{relayStoreSyncObj}). Zur Vermeidung von dealocks ist wichtig, da�
die beiden Arten der Synchronisation nicht geschachtelt werden, d.h.\ an
keiner Stelle eine Synchronisation (ein Lock) sowohl auf \texttt{this} wie
auf das extra Objekt besteht.

