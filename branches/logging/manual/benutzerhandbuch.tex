%% Realease 1.0 dieser Software wurde am Institut f�r Intelligente Systeme der
%% Universit�t Stuttgart (http://www.informatik.uni-stuttgart.de/ifi/is/) unter
%% Leitung von Dietmar Lippold (dietmar.lippold@informatik.uni-stuttgart.de)
%% entwickelt.

\documentclass[a4,10pt]{report}
\usepackage[german]{babel} % Einbindung des deutschen Sprachpaketes
\usepackage[ansinew]{inputenc}
\usepackage{verbatim}
\usepackage[T1]{fontenc}
\usepackage{palatino}
\pagestyle{headings}
\usepackage[pdftex=true
,pdftitle={Glossar des Compute-Systems}
,pdfauthor={Achim Linke, Ralf Kible}
,pdfsubject={Hiermit sollen die Begriffe gekl�rt werden, die in den weiteren
    Dokumenten benutzt werden}
%,pdfkeywords={keyword 1, keyword 2, keyword 3, keyword 4}
%,plainpages=false
%,hypertexnames=false 
,pdfpagelabels=true
,pdfpagemode=UseOutlines
,hyperindex=true
,colorlinks=true]{hyperref} 
% Hier wird die Farbe der Links ein- und ausgeschaltet.

\usepackage{color}
\definecolor{darkred}{rgb}{0.5,0,0}
\definecolor{darkgreen}{rgb}{0,0.5,0}
\definecolor{darkblue}{rgb}{0,0,0.8}
\hypersetup{colorlinks
,linkcolor=darkblue
,filecolor=darkgreen
,urlcolor=darkred
,citecolor=darkblue}

%% M�gliche Werte: 
%% -1
%% keine �berschrift wird numeriert 
%% 0
%% Kapitel�berschriften werden numeriert 
%% 1
%% Kapitel- und Abschnitts�berschriften werden numeriert 
%% 2
%% Kapitel- bis Unterabschnitts�berschriften werden numeriert 
%% 3
%% Kapitel- bis Unterunterabschnitts�berschriften werden numeriert 
%% 4
%% Kapitel- bis Paragraphs�berschriften werden numeriert 
%% 5
%% alle �berschriften werden numeriert 
\setcounter{secnumdepth}{5}

\setcounter{tocdepth}{2}

\begin{document}
\title{Architeuthis - Benutzerhandbuch}
\author{J�rgen Heit, juergen.heit@gmx.de\\
Andreas Heydlauff, andiheydlauff@gmx.de\\
Ralf Kible, ralf\_kible@gmx.de\\
Achim Linke, achim81@gmx.net}

\date{Software Version: 0.9\\Stand: \today}
\maketitle
\tableofcontents
%\listoffigures
%\listoftables
\chapter{Allgemeine Hinweise}
Das Compue-System soll einem Benutzer die M"oglichkeit geben, ein von
ihm in Java implementiertes Problem auf mehreren Rechnern verteilt berechnen
zu lassen. Es besteht aus mehreren Komponenten, die hier erl"autert werden.


    %% Realease 1.0 dieser Software wurde am Institut f�r Intelligente Systeme der
%% Universit�t Stuttgart (http://www.informatik.uni-stuttgart.de/ifi/is/) unter
%% Leitung von Dietmar Lippold (dietmar.lippold@informatik.uni-stuttgart.de)
%% entwickelt.


\chapter{Funktionen der Komponenten des Systems}

\section{Vorwort}

Dieses Kapitel basiert auf der Spezifikation der hier dokumentierten
Software. Jedoch scheint es sich ebenfalls sehr gut als Kurz�bersicht �ber
die Funktionen und die Arbeitsweise des Compute-Systems zu eignen.

\section{Grunds�tzliches:}

\begin{itemize}
   \item Der Name des Systems ist Architeuthis.

   \item Zu erstellen ist ein Compute-System, das die Verwaltung f�r die
         parallele Berechnung einer beliebigen parallelisierbaren Aufgabe
         (im folgenden {\em Problem} genannt, siehe auch Glossar)
         �bernimmt.

   \item Die Implementierung erfolgt platformunabh�ngig in Java mit Hilfe von
         RMI.

   \item Die Implementierung besteht aus mehreren Programmteilen:
         Operative, Dispatcher, Problem-�bermittler,
           sowie einer Testumgebung (mit Beispielen von Problemen).
         F�r die korrekte Funktionalit�t muss eine RMI-f�hige Verbindung
         (Netzwerkverbindung) zwischen Problem-�bermittler und Dispatcher
         und ebenso zwischen Operative und Dispatcher bereitstehen.
\end{itemize}


\section{Dispatcher:}

Der Dispatcher ist ein ausf�hrbares Java-Programm, das pro Compute-System
genau einmal zur Verf�gung steht. Er besteht aus drei Komponenten:

\begin{enumerate}
    \item Dem Problem-Manager 
    \item Dem Compute-Manager
    \item Einer Statistik-Komponente
\end{enumerate}

\subsection{Problem-Manager}
\begin{itemize}
\item Die ben"otigten Klassen des Problems werden vom Problem-�bermittler an
      den Problem-Manager ``�bertragen''. Der Problem-�bermittler "ubergibt
      ihm die URL, an der das Problem-Paket inklusive ihrer
      Hauptproblemklasse liegt und entweder den Namen und die
      serialisierbaren Parameter des Konstruktors oder eine serialisierbare
      Instanz der Problemklasse. Bei der �bergabe eines Namens wird die
      Klasse vom Webserver geladen und das Problem wird initialisiert.

\item Nach der Problem�bergabe weist er das Problem an, sich in Teilprobleme
      zu unterteilen. Dabei wird vom Dispatcher eine Anzahl von Teilen 
      vorgeschlagen, die sich nach der Anzahl der momentan verf�gbaren
      Operatives richtet. Diese Teilprobleme werden einzeln von der
      Problemklasse geholt und �ber den Compute-Manager an Operatives
      verteilt.

\item Gibt die vom Compute-Manager erhaltene Teill�sung an das Problem
      weiter. Falls die Gesamtl�sung durch diese Teill�sung erzeugt werden
      kann, wird diese an den Problem-�bermittler gesendet.
\end{itemize}


\subsection{Compute-Manager}
\begin{itemize}
\item Operatives k�nnen sich bei ihm durch ein Interface registieren
      und abmelden.

\item Er erledigt die Verteilung der Teilprobleme eines oder mehrerer
      Probleme an die Operatives.

\item Falls Teilprobleme vorhanden sind, bekommen anmeldende 
      Operatives eines zum Berechnen.

\item Teilprobleme k�nnen auch mehrfach vergeben werden, sobald keine neuen
      Teilprobleme mehr vorhanden sind, aber noch welche in Berechnung sind.

\item Empf"angt die Teill"osungen der Operatives. Diese werden dann
      an den Problem-Manager weitergeleitet. Dadurch frei werdende Operatives
      bekommen ein neues Teilproblem zum Berechnen, falls eines vorhanden ist.

\item Wenn der Dispatcher keine Teilprobleme mehr hat und auch keine 
      mehr in Berechnung sind, wird jeder Operative, der eine Teill�sung 
      zur�ck gibt, als verf�gbar markiert, so dass ihm ein neues Teilproblem 
      geschickt wird, sobald welche vorhanden sind.

\item "Uberwacht die Operatives auf Erreichbarkeit.
\end{itemize}

\subsection{Statistik}
Es werden verschiedene Statistiken f�r den Benutzer gef�hrt. 
Sie enthalten Informationen �ber den aktuellen Zustand des 
Dispatchers oder �ber einzelne Probleme.

\begin{itemize}
\item  Eine allgemeine Statistik des Dispatchers enth�lt folgende Punkte:
       \begin{enumerate}
        \item Anzahl der angemeldeten Operatives
        \item Anzahl der Operatives, die gerade kein Teilproblem berechnen
        \item Anzahl der noch nicht fertig berechneten Probleme
        \item Anzahl der insgesamt erhaltenen Probleme
        \item Anzahl der berechnete Teilprobleme
        \item Anzahl der gerade in Berechnung befindlichen Teilprobleme
        \item Durchschnittliche Berechungsdauer der fertig berechneten
              Teilprobleme
        \item Berechungsdauer aller fertig berechneten Teilprobleme
        \item Alter des Dispatcher
       \end{enumerate}

\item  Folgende Punkte werden in der problemspezifischen Statistik f�r jedes
       einzelne Problem gef�hrt:
       \begin{enumerate}
        \item Anzahl der fertigen Teilprobleme
        \item Anzahl der gerade in Berechnung befindlichen Teilprobleme
        \item Durchschnittliche Berechungsdauer der fertig berechneten
              Teilprobleme
        \item Berechnungsdauer der fertig berechneten Teilprobleme
        \item Anzahl der vorgeschlagenen Teilprobleme vom Compute-Manager
              (entspricht der Anzahl der freien Operatives beim
              Initialisieren des Problems)   
        \item Alter des Problems auf dem Dispatcher
     \end{enumerate}
\end{itemize}

W�hrend des Betriebs kann eine Momentaufnahme jeder Statistik vom Benutzer 
in extra Programmen abgefragt werden und dort grafisch aufbereitet 
angezeigt werden. Eine problemspezifische Endstatistik wird mit der
Gesamtl�sung an  den Problem-�bermittler gesandt.

\section{Operative:}

\begin{itemize}

   \item Der Operative ist ein ausf�hrbares Java-Programm, das beliebig
         oft pro Compute-System existiert (Gr��enordung 10 -- 100 mal). Er
         berechnet ein Teilproblem.

   \item Ein Operative verbindet sich mit einem anzugebenden
         Dispatcher �ber RMI und registriert sich �ber das Interface,
         das der Dispatcher anbietet. Informationen �ber die Adresse des
         Dispatchers werden dem Operative �ber die Kommando-Zeile
         �bergeben.

   \item Der Verf�gbarkeitsstatus eines Operatives f�r neue Teilprobleme
         wird vom Compute-Manager verwaltet. Bei der Anmeldung des Operatives
           am Dispatcher berechnet der Operative noch keine Teilaufgabe.
         Sofern der Dispatcher noch Teilprobleme hat, wird dem
         Operative unmittelbar nach der Anmeldung ein Teilproblem
         "ubertragen.

   \item Sobald der Operative die Berechnung einer Teilaufgabe 
         abgeschlossen hat, schickt er seine Teill�sung an den Dispatcher. 
         Der Dispatcher wei� jetzt, dass der Operative wieder verf�gbar
         ist und kann ihm eine neue Teilaufgabe zuweisen.

   \item Vor Beenden meldet sich der Operative beim Dispatcher ab.

\end{itemize}


\section{Problem-�bermittler:}

\begin{itemize}

   \item Er ist die Schnittstelle zwischen Benutzer und Compute-Systems. "Uber
         ihn l"auft die Kommunikation.

   \item Der Problem-�bermittler ist ein lauff�higes Java-Programm, das URL
         und entweder ``Hauptklassenname'' und serislisierbare Parameter
         oder eine serialisierbare Instanz eines speziellen Problems an den
         Dispatcher, genauer an den Problem-Manager, �bermittelt.

   \item Er bekommt die L"osung des Problems vom Dispatcher, damit sie f"ur
         den Benutzer verf"ugbar ist. Der Problem-�bermittler wartet, bis der
	 Problem-Manager das Ergebnis zur�ckgeliefert hat.
\end{itemize}

\section{Problem}

\subsection{Anforderungen an das Problem:}

Das Problem muss ein Interface implementieren, das folgende Funktionen
bietet:

\begin{itemize}
   \item Teilprobleme bereitstellen, solange keine L�sung f�r das
	 Problem erstellt werden kann

   \item Teill�sung entgegen nehmen

   \item L�sung in endlicher Zeit erstellen
\end{itemize}

\subsection{Anforderungen an die L�sung:}

Eine L�sung wird von einem Problem in endlicher Zeit erzeugt und muss an
den Problem-"Ubermittler �bermittelbar (serialisierbar) sein.

\subsection{Anforderungen an das Teilproblem:}

Teilprobleme m�ssen von der Problemklasse bereitgestellt werden und vom
Dispatcher zu den Operatives �bermittelbar (serialisierbar) sein. Sie m�ssen
ein Interface implementieren, das die Funktion

\begin{itemize}
    \item Berechne das Teilproblem und erzeuge Teill�sung
\end{itemize}

bietet.

\subsection{Anforderungen an die Teill�sungen:}

Eine Teill�sung wird von einem Teilproblem in endlicher Zeit erzeugt und muss
von Operative �ber Dispatcher zur Problemklasse �bermittelbar sein.

\section{Laufzeit-Vergleich}

Programm dient zum Vergleich eines implementierten Problems zwischen der
Berechnung auf einem einzelnen Computer und der Berechnung auf dem
Com\-pute-System.

\begin{itemize}
    \item Bei der Berechnung auf einem einzelnen Computer wird \textit{ein}
          Teilproblem erzeugt und dieses berechnet.
    \item Bei der Berechnung auf dem Compute-System wird dem
          Problem-"Uber\-mitt\-ler die Adresse des Webservers, auf dem das
          Problem liegt, der Name des Problems und die Adresse des
          Compute-Systems �bergeben.
    \item Es wird die gebrauchte Zeit, sowie die L"osung von beiden
          Berechnungen ausgegeben.
\end{itemize}


    %% Realease 1.0 dieser Software wurde am Institut f�r Intelligente Systeme der
%% Universit�t Stuttgart (http://www.informatik.uni-stuttgart.de/ifi/is/) unter
%% Leitung von Dietmar Lippold (dietmar.lippold@informatik.uni-stuttgart.de)
%% entwickelt.


\chapter{�berblick �ber das Systems, Begriffskl�rung}

\section{Name: Architeuthis}

Der d�nische Wissenschaftler Japetus Steenstrup erhielt Schnabel, Schulp und
einige Saugn�pfe eines Riesenkalmars, der ein Jahr zuvor an der d�nischen
K�ste an Land gesp�lt worden war. Er verglich das Material mit
entsprechenden Organen bekannter kleinerer Kalmararten und schloss daraus,
dass es zu einem riesigen Kalmar geh�ren m�sse, den er Architeuthis, den
ersten oder gr��ten Kalmar, nannte. Die Gattung \emph{Architeuthis}
(Steenstrup 1856) bezeichnet heute noch die Atlantischen Riesenkalmare.

Riesenkalmare geh�ren zu den Kopff��ern (Cephalopoda). Sie haben im Ganzen
zehn Arme, davon zwei lange Tentakel mit keulenf�rmig verbreiterten Enden,
die mit Saugn�pfen bewehrt sind und zum Fangen der Beute dienen. Acht kurze
Arme rund im die Mund�ffnung f�hren die Beute dem Mund zu.

Quelle: http://www.weichtiere.at/Kopffuesser/kalmar.html (Abruf 15.04.2006)


\subsection*{Aussprache}
Das Wort \emph{Architeuthis} ist aus dem Lateinischen abgeleitet, wird
deshalb auf der zweiten Silbe betont und sonst nach der im Deutschen
�blichen Sprechweise ausgesprochen. Insbesondere ist das ``ch'' kein K-Laut,
das ``eu'' wie in ``Europa'' und das zweite ``h'' stimmlos.

\section{�bersicht}

Die Architektur des \hyperref[compsys]{Compute-Systems} ist in drei Ebenen
und drei Stufen gegliedert, die im folgenden genauer erkl�rt werden:

\begin{center}
\begin{tabular}{l|c|c|c}

Stufe $\rightarrow$  &       Benutzer      & Dispatcher & Operative   \\
Ebene $\downarrow$   &        Stufe        &   Stufe    & Stufe       \\
\hline
 Anwendungs Ebene    &    Problem-Paket    &  Problem   & Teilproblem \\
                     &    Testumgebung     &  L�sung    & Teill�sung  \\
\hline
 Compute-System      & Problem-�bermittler & Dispatcher & Operative   \\
 Ebene               &  Statistik-Anzeige  &            &             \\
\hline
 Hardware Ebene & Benutzer-Rechner & Dispatcher-Rechner & Operative-Rechner \\
                &   Web-Server     &                    &                   \\
\end{tabular}
\end{center}


\section{Anwendungs Ebene}

F�r die Bedienung des Compute-Systems wird im Folgenden von mehreren 
Personen ausgegangen, die aber nicht notwendig verschieden sein m�ssen:

\begin{itemize}
    \item \hyperref[benutzer]{Benutzer}
    \item \hyperref[admin]{Dispatcher-Administrator}
    \item \hyperref[clber]{Operative-Bereitsteller}
\end{itemize}

\subsection{Benutzer Stufe}

\paragraph*{Problem-Paket}\label{probpak}
  steht f�r folgende Java-Klassen:
\begin{enumerate}
    \item {\em Problem}
    \item {\em PartialProblem (Teilproblem)}
    \item {\em PartialSolution (Teill�sung)}
\end{enumerate}

\paragraph*{Benutzer}\label{benutzer}
    startet ein Anwendungsprogramm des
    \hyperref[compsys]{Compute-Systems}. Er ist im Normalfall auch derjenige,
    der das {\em Problem-Paket} implementiert hat.

\paragraph*{Testumgebung}\label{testumg}
    stellt Tests zur �berpr�fung der Funktion und zur Messung der
    Performance des \hyperref[compsys]{Compute-Systems} bereit. F�r
    letzteres kann ein {\em Problem} jeweils auf dem
    \hyperref[benrech]{Benutzer-Rechner} und auf dem
    \hyperref[compsys]{Compute-System} berechnet und die jeweile
    Berechnungzeit verglichen werden.

\subsection{Dispatcher Stufe}

\paragraph*{Administrator}\label{admin}
    startet den Dispatcher mit der Java-Klasse {\tt ComputeManagerImpl}.

\paragraph*{Problem}\label{prob}
    bezeichnet eine Java-Klasse, die das Interface {\em Problem}
    implementiert und so mit Hilfe des \hyperref[compsys]{Compute-Systems}
    eine Berechnung vornimmt. Das Interface erfordert, dass die Java-Klasse
    dem \hyperref[compman]{Compute-Manager} {\em Teilprobleme} liefert und
    die zugeh�rigen {\em Teill�sungen} wieder zusammenfassen kann, um daraus
    die {\em Gesamtl�sung} zu erstellen.

\paragraph*{Gesamtl�sung}\label{gesl"os}
    wird vom {\em Problem} erzeugt,  nachdem alle notwendigen
    {\em Teilprobleme} berechnet worden sind. Die Gesamtl�sung muss ein
    {\em Serializable-Objekt} sein.


\subsection{Operative Stufe}

\paragraph*{Operative-Bereitsteller}\label{clber}
    startet einen Operative. Da mehere Operatives in einem Compute-System 
    arbeiten k�nnen, handelt es sich bei den Operative-Bereitstellern
    ebenfalls um mehrere Personen (die aber nicht notwendig verschieden sein
    m�ssen).

\paragraph*{Teilproblem (PartialProblem)}\label{teilprob}
    ist ein von einem {\em Problem} erzeugtes Objekt, das von einem
    \hyperref[compclie]{Operative} verarbeitet werden kann, indem es das
    Interface {\em PartialProblem} implementiert.

\paragraph*{Teill�sung (PartialSolution)}\label{teillos}
    ist ein von einem {\em Teilproblem} erzeugtes Objekt, das dem {\em
    Problem} zur�ckgegeben wird. Die Klasse muss {\em Serializable}
    implementieren.


\section{Compute-System Ebene}


\paragraph*{Compute-System}\label{compsys}
    bezeichnet die Gesamtheit aller Komponenten, die zur Berechnung eines
    parallelisierten \hyperref[prob]{Problems} ben�tigt werden, also den
    {\em Problem-�bermittler}, den {\em Dispatcher} und die {\em Operatives}.


\subsection{Benutzer Stufe}

\paragraph*{Problem-�bermittler (ProblemTransmitter)}\label{probueb}
\label{anwpr}
    ist die Software-Komponente, die Kontakt zum {\em Dispatcher} herstellt
    und das �bertragen des \hyperref[prob]{Problems} vom
    \hyperref[benutzer]{Benutzer} erm�glicht. Die zugeh�rigen Klassen sind
    \emph{ProblemTransmitterImpl} und \emph{ProblemComputation}.

\paragraph*{Statistik-Anzeige},
    stellt dem Benutzer aktuelle Informationen zum Compute-Systems insgesamt
    (\texttt{Systems-Statistik}) und zu seinem \hyperref[prob]{Problem}
    (\texttt{Problem-Statistik}) bereit. Die zugeh�rigen Klassen sind
    \emph{ProblemGUIStatisticsReader}, \emph{SystemGUIStatisticsReader} und
    \emph{SystemTextStatisticsReader}.


\subsection{Dispatcher Stufe}


\paragraph*{Dispatcher}\label{compserv}
    besteht aus drei Komponenten: Dem {\em Compute-Manager}, dem
    \emph{Prob\-lem-Manager} und der {\em Statistik}. Er ist das Herzst"uck
    des {\em Compute-Systems}. Er wird durch den
    \hyperref[admin]{\em Administrator} auf dem
    \hyperref[servrech]{\em Dispatcher-Rechner} gestartet.

\paragraph*{Problem-Manager}\label{problman}
    bezeichnet den Teil des {\em Dispatchers}, der sich um die Verwaltung
    der \hyperref[prob]{Probleme} und \hyperref[teilprob]{Teilprobleme}
    k"ummert, da dies bei mehreren {\em Problemen} notwendig ist.

\paragraph*{Compute-Manager}\label{compman}
    bezeichnet den Teil des {\em Dispatchers}, der sich um die Verwaltung
    der {\em Operatives} k"ummert und die Verteilung der
    \hyperref[teilprob]{Teilprobleme} "ubernimmt.

\paragraph*{Statistik}\label{stat}
    bezeichnet den Teil des {\em Dispatchers}, der die  statistischen Daten
    des {\em Compute-Systems} verwaltet. 


\subsection{Operative Stufe}


\paragraph*{Operative}\label{compclie}
    bekommt von dem {\em Compute-Manager} genau ein
    \hyperref[teilprob]{Teilproblem} und berechnet dieses. Die
    \hyperref[teillos]{Teill�sung} schickt er dann wieder an ihn zur�ck.


\section{Hardware Ebene}

Die Hardware unfasst die f�r das Gesamtsystem erforderlichen Rechner.
Einzelne konkrete Rechner k�nnen auch mehrere Funktionen ausf�hren (z.B.
gleichzeitig {\em Dispatcher-Rechner} und {\em Operative-Rechner} sein).


\subsection{Benutzer Stufe}

\paragraph*{Benutzer Rechner}\label{benrech}
  startet die Kommunikation zum \hyperref[compsys]{Compute-System}.

\paragraph*{Web-Server}\label{weserver}
    ist n�tig, damit das \hyperref[compsys]{Compute-System} das
    \hyperref[probpak]{Problem-Paket}  laden kann.


\subsection{Dispatcher Stufe}

\paragraph*{Dispatcher Rechner}\label{servrech}
  stellt die Hardware f�r den \hyperref[compsys]{Dispatcher} zur Verf�gung.


\subsection{Operative Stufe}

\paragraph*{Operative Rechner}\label{clierech} 
  ist mit dem {\em Dispatcher-Rechner} verbunden. Auf ihm laufen die
  eigentlichen Berechnungen ab.


    %% Realease 1.0 dieser Software wurde am Institut f�r Intelligente Systeme der
%% Universit�t Stuttgart (http://www.informatik.uni-stuttgart.de/ifi/is/) unter
%% Leitung von Dietmar Lippold (dietmar.lippold@informatik.uni-stuttgart.de)
%% entwickelt.


\chapter[Benutzung als Dispatcher -Administrator]{Die Benutzung des Systems
  als Dispatcher-Administrator}

\section{Der Dispatcher}

Der Dispatcher besteht aus drei Komponenten: Dem Com\-pute-Manager,
dem Pro\-blem-Manager und der Statistik. Er ist das Herzst"uck des 
Compute-Systems. 


\subsection{Voraussetzungen}

Ben�tigt wird zun�chst ein funktionierendes Java Runtime Environment, Version
1.4 oder neuer. Es ist eine Netzwerkanbindung n"otig, damit eine Kommunikation
zu den Operatives geschaffen werden kann. Besondere Hardware-Anforderungen
werden nicht gestellt, ein Rechner mit 600 MHz und 256 MB RAM kann problemlos 
als Dispatcher f�r �ber 30 Operatives verwendet werden.


\subsection{Benutzung}

Um den Dispatcher benutzen zu k�nnen, muss er auf einem von allen Operatives
erreichbaren Computer gestartet werden. Dieser Computer muss auch f�r
Anfragen von den Computern offen sein, die sp�ter Probleme auf den
Dispatcher �bertragen wollen. Wie sp�ter beim Operative gilt auch hier: Am
besten l�uft der Dispatcher mit angepa�ter Priorit�t im Hintergrund, falls
auf dem Rechner zus�tzlich eine interaktive Arbeit erfolgt.

\subsection[Kommandozeilenparameter]{Aufruf und Kommandozeilenparameter}
Kommandozeile zum Starten des Dispatchers (alles in einer Zeile, die
Zeilenumbr�che wurden nur zur besseren �bersicht eingef�gt):

{\tt java -Djava.rmi.server.RMIClassLoaderSpi=\\
  de.unistuttgart.architeuthis.misc.CacheFlushingRMIClSpi\\
  -Djava.util.logging.config.file=logging.properties\\
  -Djava.security.policy=dispatcher.pol\\
  de.unistuttgart.architeuthis.dispatcher.DispatcherImpl\\
  -c <config-Datei> -port <Port-Nummer>\\
  -deadtime <Zeit> -deadtries <Anzahl> -d -t -help}

Dabei ist:

\begin{description}
\item[-Djava.rmi...CacheFlushingRMIClSpi] Optional. Dadurch werden die
  Klassen von Problemen nur gecached, solange das Problem in Berechnung
  ist. Sobald die L�sung dem Benutzer zur�ck�bermittelt wurde, werden die
  dazugeh�rigen Klassen aus dem Cache gel�scht. Damit k�nnen Probleme,
  deren Klassen modifiziert wurden, dem Compute-System wieder zum 
  Berechnen �bergeben werden. Normalerweise w�rde das Java-eigene
  Cachen von Klassen dies verhindern.

  {\bf Achtung:} Selbst durch diese Massnahme kann es zu merkw�rdigen
  Effekten kommen (z.B. ClassCast-Exception), wenn mehrere Probleme
  gleichzeitig in modifizierter Form berechnet werden, oder wenn mehrere
  Probleme im gleichen Verzeichnis des Webservers bereitgestellt werden.
  Daher sollte am Einfachsten f�r jedes neue Problem auch ein neues
  Verzeichnis auf dem Webserver angelegt werden. Dann treten die oben
  angedeuteten Probleme nicht auf.

\item[-Djava.util.logging.config.file=logging.properties]
  Optional. Gibt die Konfigurationsdatei f�r den im Dispatcher 
  verwendeten Logger an. Die Konfigurationsdatei selbst ist kommentiert,
  f�r die einzelnen Optionen bitte diese einsehen. Wird keine Angabe zur 
  Konfigurationsdatei gemacht, verwendet der Dispatcher die
  Standard-Konfiguration. Weitere Informationen finden sich au�erdem in der 
  Java API-Do\-ku\-men\-ta\-tion unter \texttt{java.util.logging}.
  Der Dispatcher arbeitet im Namespace ``de.unistuttgart.architeuthis''.

  {\bf Tipp:} Normalerweise sollten im Betrieb nur Informationen der
  Informationslevel WARNING oder SEVERE oder eventuell INFO geloggt werden,
  da sonst wichtige Ausgaben eventuell �bersehen werden. Zum Testen
  empfiehlt sich manchmal, einen niedrigeren Level zu w�hlen, beispielsweise
  CONFIG.

\item[-Djava.security.policy=dispatcher.pol]
  Die policy-Datei f�r den Dispatcher. Diese Datei darf sich nicht in
  einer jar-Datei befinden; der Pfad muss also relativ oder absolut angegeben
  werden. Siehe dazu das Kapitel \hyperref[sicherh]{Sicherheit}.  

\item[de.unistuttgart.architeuthis.dispatcher.DispatcherImpl]
  Die Klasse zum Starten des Dispatchers.

\item[-c <config-Datei>]
  Optional. Es kann eine Konfigurationsdatei angegeben werden, die die
  restlichen Parameter beinhaltet. Standardm��ig ist ``compserv.conf''
  eingestellt. Falls diese Datei nicht vorhanden ist, werden Standardwerte
  verwendet.

\item[-port <Port-Nummer>]
  Optional. Setzt die Port-Num\-mer, un\-ter der die RMI-Regi\-stry des
  Dispatchers zu erreichen sein wird. Standardm��ig wird 1099 angenommen
  (wie bei jeder RMI-Kommunikation).

\item[-deadtime <Zeit>]   
  Optional. Setzt die Zeit zwischen den Versuchen, Operatives zu finden, die
  nicht mehr antworten. Eine h�here Zeit verursacht weniger Rechenlast und
  Netzwerk-Verkehr auf dem Dispatcher, jedoch werden dann fehlerhafte
  Operatives sp�ter entdeckt.

\item[-deadtries <Anzahl>]
  Optional. Setzt die Anzahl der Fehler, die sich bei einer Verbindung zu
  einem Operative ereignen d�rfen, bis dieser entfernt wird.

\item[-d] Optional. Schaltet den Debug-Modus mit zus�tzlichen Meldungen
  ein.

\item[-t] Optional. Erzeugt f�r die �bergabe und f�r den Abbruch eines
  Teilproblems an die bzw.\ auf den Operatives einen neuen Thread.

\item[-help] Optional. Gibt eine Meldung zur Benutzung aus und beendet
  anschlie�end das Programm.
\end{description}


\chapter[Benutzung als Operative-Administrator]{Die Benutzung des Systems als
  Operative-Administrator}

\section{Der Operative}
Die Operatives stellen die Rechenleistung des Compute-Systems bereit und
f�hren die tats�chlichen Berechnungen aus.


\subsection{Voraussetzungen}
Die einzige Voraussetzung zum Betreiben eines Operatives ist ein
installiertes Java-Runtime-Environment der Version 1.3 oder neuer und eine
Netzwerkanbindung. Bei Benutzung von Version 1.3 ist jedoch zu beachten,
dass dann ein nicht-cachen\-der Betrieb nicht m�glich ist (siehe Option
\hyperref[cachen]{CacheFlushingRMIClSpi}).


\subsection{Benutzung}
Die Benutzung gestaltet sich sehr einfach. Der Operative muss nur gestartet
werden, wobei als Kommandozeilenparameter die RMI-Adresse des
Compute-Mana\-gers angegeben werden muss, bei dem sich der Operative anmelden
soll. Der Operative sollte am Besten mit nicht zu hoher Priorit�t im
Hintergrund laufen, falls auf dem Rechner zus�tzlich eine interaktive Arbeit
erfolgt.

{\bf Tipp:} Falls der Compute-Manager dauerhaft l�uft, kann der Operative
problemlos beim Rechnerstart im Hintergrund gestartet werden. Das 
Herunterfahren des Rechners beendet den Operative ord\-nungs\-ge\-m��.
Ebenfalls denkbar ist es, den Operative im Hintergrund des GDM laufen zu
lassen.

{\bf Achtung:} Der Dispatcher und der Operative m�ssen gegenseitig
erreichbar sein.


\subsection[Kommandozeilenparameter]{Aufruf und Kommandozeilenparameter}
Kommandozeile zum Aufruf des Operatives:

{\tt java -Djava.security.policy=operative.pol\\
  -Djava.rmi.server.RMIClassLoaderSpi=\\
  de.unistuttgart.architeuthis.misc.CacheFlushingRMIClSpi\\
  -Djava.util.logging.config.file=logging.properties\\
  de.unistuttgart.architeuthis.operative.OperativeImpl\\
  <Adresse> -d}

Dabei ist:

\begin{description}
\item[-Djava.security.policy=operative.pol]
  Gibt die Policy-Datei f�r den Operative an. Diese Datei darf sich nicht in
  einer jar-Datei befinden; der Pfad muss also relativ oder absolut angegeben
  werden. Siehe dazu das Kapitel
  \hyperref[sicherh]{Sicherheit}.

\item[-Djava.util.logging.config.file=logging.properties]
  Optional. Gibt die Konfigurationsdatei f�r den im Operative 
  verwendeten Logger an. Die Konfigurationsdatei selbst ist kommentiert,
  f�r die einzelnen Optionen bitte diese einsehen. Wird keine Angabe zur 
  Konfigurationsdatei gemacht, verwendet der Operative die
  Standard-Konfiguration. Weitere Informationen finden sich au�erdem in der 
  Java API-Do\-ku\-men\-ta\-tion unter \texttt{java.util.logging}.
  Der Dispatcher arbeitet im Namespace ``de.unistuttgart.architeuthis''.

  {\bf Tipp:} Normalerweise sollten im Betrieb nur Informationen der
  Informationslevel WARNING oder SEVERE oder eventuell INFO geloggt werden,
  da sonst wichtige Ausgaben eventuell �bersehen werden. Zum Testen
  empfiehlt sich manchmal, einen niedrigeren Level zu w�hlen, beispielsweise
  CONFIG.

\item[-Djava.rmi...CacheFlushingRMIClSpi]
\label{cachen}
  L�dt einen anderen Service Provider f�r den RMIClassLoader. Dies bewirkt,
  dass Klassen nicht gecached werden und somit Probleme, deren Klassen
  modifiziert wurden, wieder mit dem System berechnet werden k�nnen. Zur
  Minimierung des Netzwerk-Verkehrs kann diese Option auch entfallen.

  {\bf Achtung:} Falls diese Option nicht angegeben wird, werden
  Modifikationen in Teilproblem-Klassen nur dann erkannt, falls diese Klassen
  entweder in einem anderen Verzeichnis auf dem Webserver abgelegt werden oder
  der Operative neu gestartet wird. Wenn die Option angegeben wird, d�rfen
  die class-Dateien nicht ge�ndert werden, w�hrend das Problem berechnet wird.

\item[de.unistuttgart.architeuthis.operative.OperativeImpl]
  Die Klasse zum Starten des Operatives.

\item[Adresse] 
        Die Adresse der RMI-Registry, in der der Dispatcher eingetragen ist,
        bei dem sich der Operative anmelden soll. Exemplarisch:\\
        {\tt <Rechnername>:<Port>}

        Dabei ist:

        \begin{description}
        \item[Rechnername] Der Name oder die IP-Adresse des Rechners,
            auf dem der Dispatcher l�uft.
        \item[Port] Optional. Der Port, auf dem der Dispatcher
            auf Verbindungen h�rt. Falls nicht angegeben, wird 1099
            angenommen und au�erdem muss dann der : entfallen.
        \end{description}

\item[-d] Optional. Schaltet den Debug-Modus mit zus�tzlichen Meldungen
        ein.
\end{description}


\chapter[Benutzung als Endanwender]{Die Benutzung des Systems als Endanwender}

\section{Das eigentliche Problem}
Zu implementieren sind drei Java-Klassen, die im folgenden n"aher erkl"art
werden.

Alle nachfolgend genannten Interfaces befinden sich im Package
\texttt{de...userinterfaces.develop}.


\subsection[Problem]{Das Problem - Problem.java} 

Das Problem ist die organisierende Hauptklasse. Es generiert die
Teilprobleme, empf"angt die Teill"osungen und gibt die Gesamtl"osung
zur"uck, sobald diese erzeugt werden konnte. Das Interface \texttt{Problem}
besitzt dementsprechend drei Methoden. In der Regel sollte das konkrete
Problem serialisierbar sein und das Unter-Interface
\texttt{SerializableProblem} implementieren.

%{\scriptsize \verbatiminput{java/Problem.jav}}

%{\scriptsize \verbatiminput{java/SerializableProblem.jav}}

{\bf Hinweis:}
Falls das Erstellen von Teilproblemen oder das Verarbeiten von Teill�sungen
rechenintensiv ist, ist es ratsam, daf�r selbst Teilprobleme zu generieren.

\subsection[Teilproblem]{Das Teilproblem - PartialProblem.java}
Das Teilproblem ist das eigentliche Rechenprogramm, das auf dem Operative
ausgef�hrt wird. Das Interface selbst fungiert lediglich als Ober-Interface
ohne Methoden, erweitert aber das Interface \texttt{Serializable}.

%{\scriptsize \verbatiminput{java/PartialProblem.jav}}

Es besitzt zwei Unter-Interfaces, von denen eines von einem konkreten
Teilproblem zu implementieren ist. \texttt{NonCommPartialProblem} ist f�r
Teilprobleme, die keinen gemeinsamen Speichers benutzen
(nicht-kommunizierende Teilprobleme). \texttt{CommunicationPartialProblem}
ist f�r Teilprobleme, die einen gemeinsamen Speichers benutzen
(kommunizierende Teilprobleme). Beide Unter-Interfaces besitzen nur eine
Methode \texttt{compute} und unterscheiden sich nur darin, dass diese Methode
im ersten Fall keinen und im zweiten Fall einen Parameter besitzt.

%{\scriptsize \verbatiminput{java/NonCommPartialProblem.jav}}

%{\scriptsize \verbatiminput{java/CommunicationPartialProblem.jav}}


\subsection[Teill"osung]{Die Teill"osung - PartialSolution.java}
Eine Teill"osung ist das Ergebnis des berechneten Teilproblems und muss
das Interface \texttt{PartialSolution} implementieren, das wiederum
\texttt{Serializable} erweitert.

%{\scriptsize \verbatiminput{java/PartialSolution.jav}}

\subsection{Wichtige allgemeine Hinweise}

\begin{itemize}
\item Alle in einer serialisierbaren Klasse (die \texttt{Serializable}
      implementiert) enthaltenen Attribut-Klassen m�ssen serialisierbar
      sein.

\item Wenn eine serialisierbare Klasse eine nicht-serialisierbare Oberklasse
      besitzt, muss die Oberklasse einen parameterlosen Konstruktor
      besitzten. Au�erdem muss die Unterklasse daf�r Sorge tragen, dass nach
      der Deserialisierung  einer Instanz von ihr die Attribute der
      Oberklasse die erforderlichen Werte besitzen. Weitere Hinweise zur
      Serialisierung finden sich in der API-Beschreibung zum Interface
      \texttt{java.io.Serializable}.

\item In den serialisierbaren Objekten, insbesondere in den Klassen, die
      {\tt Partial\-Problem} und {\tt Partial\-Solution} implementieren,
      d�rfen keine sta\-ti\-schen Variablen
%     \marginpar{Statische Variablen}
      verwendet werden, da diese beim Versenden der Objekte via RMI nicht
      serialisiert werden. Konstanten (also {\tt static final}-Attribute)
      d�rfen jedoch verwendet werden.

\item Generell ist es empfehlenswert, jedes neue Problem, das auf dem
      Compute-System berechnet werden soll, in einem neuen Verzeichnis auf
      dem Webserver abzulegen.

\item Sobald das Ergebnis ermittelt werden kann, sollte das {\tt Problem}
      keine weiteren {\tt Partial\-Problem}-Objekte mehr erzeugen, da diese
      sonst auch berechnet werden, ohne nach der L�sung zu fragen. Bei der
      Ankunft einer berechneten {\tt Par\-tial\-So\-lu\-tion} wird jedoch
      nach der Gesamtl�sung gefragt.

\item Die Erzeugung und das Zusammensetzen der Teilprobleme sowie das
      Berechnen der Gesamtl"osung sollte, da es auf dem Dispatcher-Rechner
      ausgef"uhrt wird, nur geringen Rechenaufwand erfordern. Falls es mehr
      Rechenaufwand erfordert, sollten diese T"atigkeiten als eigene
      Teilprobleme vergeben werden.
\end{itemize}

\subsection[Hilfestellung]{Hilfestellung bei der Implementierung mittels
                           abstrakter Klassen}
Es werden abstrakte Klassen angeboten, die das Implementieren vereinfachen
sollen. Siehe dazu \hyperref[abstrakt]{Kapitel \ref{abstrakt}
\emph{Abstrakte Hilfsklassen}}.

\subsection{Bereitstellen der Klassen mittels des ClassFileServer}
\label{classfileserver}

Wie schon erw�hnt, m�ssen alle Klassen, die zur Verarbeitung des Problems
und der Teilprobleme erforderlich sind, auf einem Webserver f�r alle
Computer des Systems erreichbar sein (also sowohl f�r Operatives, wie auch
f�r den Dispatcher). Dabei muss die benutzte Package-Hierarchie wie bei Java
�blich als Verzeichnis-Hierarchie vorhanden sein. Die Klassen k�nnen
alternativ auch in einem jar-File liegen, das ebenfalls die
Package-Hierarchie nachbildet.

Falls kein Webserver zur Verf�gung steht, kann der sogenannte
{\tt Class\-File\-Server} benutzt werden. Dies ist ein von Sun geschriebener
Mini-Webserver, der nur dazu dient, Klassen an RMI-Applikationen
auszuliefern. Die Benutzung ist sehr einfach. Man starte den
ClassFileServer mit folgender Kommandozeile:

{\tt java de.unistuttgart.architeuthis.facade.ClassFileServer <port> <root>}

Dabei ist:

\begin{description}
\item[port] Der Port auf dem der Webserver ansprechbar sein soll.
\item[root] Das Verzeichnis, das als Wurzel-Verzeichnis f�r die
    auszuliefernden Dateien des Web\-servers dienen soll.
\end{description}

Danach sind unter {\tt http://<rechnername>:<port>/} alle Dateien und
Verzeichnisse unterhalb von {\tt<root>} erreichbar.

{\bf Achtung:}
Der ClassFileServer macht alle alle Dateien (nicht nur class-Dateien)
unterhalb des angegebenen Verzeichnisses zug�nglich. Mas sollte daher f�r
die class-Dateien ein eigenes Verzeichnis erzeugen.


\section{Die Problem-Verarbeitung}
Zur Verarbeitung eines (serialisierbaren) Problems steht als Schnittstelle
zwischen Benutzer und Compute-System die Klasse
{\tt de...facade.ProblemCom\-putation} zur Verf�gung. Diese bieten zwei
Arten von Methoden:

\begin{description}
\item[transmitProblem] Diese Methode �bertr�gt ein �bergebenes
  serialisierbares Problem zur Verarbeitung an den Dispatcher. Dabei ist
  zumindest der Rechnername des Dispatcher anzugeben. Au�erdem ist eine
  \emph{codebase}, d.h.\ ein URL, unter dem die vom Problem und von den
  Teilproblemen ben�tigten Klassen abrufbar sind, oder ein Array von
  codebases entweder zu �bergeben oder �ber das Property
   \texttt{java.rmi.server.codebase} beim Aufruf der JVM anzugeben. In
  jedem Fall kann zus�tzlich (optional) noch eine Instanz von
  {\tt RemoteStoreGenerator} �bergeben werden.

\item[computeProblem] Diese Methode berechnet ein �bergebenes
  serialisierbares Probem lokal, d.h.\ ohne �bertragung an den
  Dispatcher. Diese Methode ist besonders in der Phase der Entwicklung
  n�tzlich, da z.B.\ Testausgaben m�glich sind. Der Methode kann optional
  die vorzuschlagende Anzahl der Teilprobleme und au�erdem noch optional
  eine Instanz von {\tt RemoteStoreGenerator} �bergeben werden.
\end{description}

Die Verwendung der beiden Arten von Methoden ist ansonsten gleich und
unterscheidet sich nicht vom Aufruf von Methoden, die vom Benutzer selbst
implementiert wurden und lokal ausgef�hrt werden.

Sollen mehrere Probleme nacheinander verarbeitet werden, sollte von der
oben genannten Klasse aus Effizienzgr�nden trotzdem nur eine Instanz erzeugt
werden.


\section{Der Problem-�bermittler}
\label{ProblemTransmitterImpl}

F�r den Ausnahmefall, dass ein Problem nicht serialisierbar ist, kann das
Problem vom Compute-Manager geladen und vollst�ndig auf dem Compute-System
ausgef�hrt werden. Dazu steht die Klasse
\texttt{de...facade.ProblemTransmit\-terImpl} zur Verf�gung. Sie bietet
au�erdem, auch bei der �bertragung einer Instanz eines serialisierbaren
Problems, die M�glichkeit, w�hrend der Berechnung des Problems auf dem
Compute-System nebenl�ufig eine Problem-Statistik oder eine System-Statistik
abzufragen oder die Berechnung des Problems abzubrechen. Ihre Verwendung
hat jedoch einige Nachteile:

\begin{itemize}
\item  Sie bietet keine Methode zur lokalen Berechnung eines Problems.

\item  F�r die Berechnung eines nicht-serialisierbaren Problems kann die
       zu erzeugende Instanz nur durch den Konstruktor konfiguriert werden
       und als Parameter k�nnen nur Objekte angegeben werden (keine Werte
       elementaren Typs).

\item  F�r die Berechnung eines nicht-serialisierbaren Problems muss au�erdem
       der Klassenname als String angegeben werden, wodurch der Compiler
       keine entsprechenden Pr�fungen mehr vornehmen kann.
\end{itemize}

Zur �bertragung eines Problems mittels \texttt{ProblemTransmitterImpl} muss
von dieser Klasse unter Angabe des Rechnernames des Dispatchers eine
Instanz erzeugt werden. Die Klasse bietet mehrere Methoden mit dem Namen
{\tt transmitProblem}. Bei einem nicht-serialisierbaren Problem, von dem
eine Instanz auf dem Dispatcher erzeugt wird, ist ein Array mit den
(serialisierbaren) Parameter-Objekten f�r den Konstruktor anzugeben.

Als Beispiel f�r die Verwendung von {\tt ProblemTransmitterImpl}  kann die
Klasse \texttt{de...testenvironment.prime.advanced.PrimeNumbersParallel}
zusam\-men mit den anderen Klassen dieses Packages betrachtet werden.


\section{Ausf�hrung eines eigenst�ndigen Problems}

F�r den Sonderfall, dass ein Problem komplett eigenst�ndig ist und keine
Daten mit anderen Programmen austauscht und keine Eingaben vom Benutzer
ben�tigt, steht die Klasse \texttt{de...facade.ProblemTransmitterApp} zur
Verf�gung. Der Vorteil bei der Benutzung liegt darin, da� zu einem
Problem keine extra Klasse zur Ausf�hrung des Problems erstellt werden
braucht und da� automatisch eine Problem- und eine System-Statistik
angezeigt werden kann.

Die Kommandozeile zum Ausf�hren der Klasse ist:

{\tt java -Djava.security.policy=transmitter.pol\\
  de.unistuttgart.architeuthis.facade.ProblemTransmitterApp\\
  -r <ProblemManager> -u <packageURL> -c <klassenname>\\
  -f <dateiname> -s -d -n -p}

\begin{description}
\item[-Djava.security.policy=transmitter.pol]
  Gibt die Po\-li\-cy-Da\-tei f�r den Pro\-blem-�ber\-mit\-tler an. Diese
  Datei darf sich nicht in einer jar-Datei befinden; der Pfad muss also
  relativ oder absolut angegeben werden. Siehe dazu das Kapitel
  \hyperref[sicherh]{Sicherheit}.

\item[de.unistuttgart.architeuthis.facade.ProblemTransmitterApp]
  Die Haupt-Klasse der Kommandozeilen-Applikation.

\item[-r <ProblemManager>]
        Die Adresse der RMI-Registry, in der der
        Dispatcher eingetragen ist, an den sich die Kommandozeilen-Applikation
        wenden soll. Exemplarisch:

         {\tt <Rechnername>:<Port>}

        Dabei ist:

        \begin{description}
        \item[Rechnername] Der Name oder die IP-Adresse des Rechners,
            auf dem der Dispatcher l�uft.
        \item[Port] Optional. Der Port, auf dem der Dispatcher auf
            Verbindungen h�rt. Falls nicht angegeben, wird 1099 angenommen
            und au�erdem muss der Doppelpunkt dann entfallen.
        \end{description}

\item[-u <packageURL>] Der URL des �u�ersten Pakets der Klassen des
  Problems. Die Adresse muss entweder mit ``/'' enden, falls die Klassen
  ungepackt in der Package-Hierarchie vorliegen, oder mit dem Namen der
  jar-Datei, die die Klassen in der Package-Hierarchie enth�lt. Anstatt
  einen vorhandenen Web-Server kann man auch
  den \hyperref[classfileserver]{ClassFileServer} verwenden (s.\ Abschnitt
  \ref{classfileserver}).

\item[-c <klassenname>] Der vollst�ndige Klassenname (inklusive aller Packages)
  der Klasse des Problems, die das Interface {\tt Problem} implementiert.

{\bf Achtung:}
\begin{itemize} 
\item Da bei der �bermittlung durch die Kommandozeilen-Applikation dem Problem
      keine Parameter �bergeben werden k�nnen, ist es notwendig, dass das
      Problem einen parameterlosen Konstruktor implementiert, der m�gliche
      Attribute mit den ben�tigten Werten initialisiert.

\item Die Unterverzeichnisse, die durch die Package-Struktur entstehen,
      m�ssen auf dem Webserver ebenfalls vorhanden sein.\\
      Beispiel: Die Klasse {\tt MyProblem} aus dem Package {\tt mypackage}
      ist unter der URL\\ {\tt
      http://myserver/mydir/my\-pack\-age/My\-Prob\-lem\-.class} erreichbar.
      Dann muss der Parameter {\tt -u} den Wert {\tt
      http://my\-ser\-ver/my\-dir/} erhalten und der Parameter {\tt -c} den
      Wert {\tt my\-pack\-age.My\-Prob\-lem}.
\end{itemize}

\item[-f <dateiname>] Der Name der Datei, in der die L�sung des Problems
  gespeichert werden soll. Die L�sung wird dabei einfach in ihrer
  serialisierten Form in die Datei geschrieben. Wird kein Dateiname angegeben,
  wird die L�sung auf der Standard-Ausgabe ausgegeben.

\item[-s] Optional. �bertr�gt das Problem als serialisierbares Problem.

\item[-d] Optional. Schaltet zus�tliche Debug-Meldungen ein.

\item[-n] Optional. Schaltet die beiden graphischen Statistik--Fenster ab.
          Nach der Berechnung wird eine Statistik in Textform ausgegeben.

\item[-p] Optional. Startet nur das Fenster f�r die Problem--Statistik, nicht
          jedoch das f�r die System--Statistik.
\end{description}

Das Schlie�en der beiden graphischen Statistik--Fenster hat keinen Einflu�
auf die Berechnung des Problems. Die Berechnung des Problems kann jedoch
durch {\tt Strg-C} im Fenster, in dem das Problem gestartet wurde,
abgebrochen werden.


\section{Der Laufzeitvergleich}

Die Klasse \texttt{de...facade.RuntimeComparison} dient zum Vergleich
zwischen der Berechnung eines Problems auf dem lokalen Computer und der
Berechnung auf dem Compute-System. Sie gibt jeweils die ben�tigte Zeit und
die L"osung von beiden Berechnungen aus oder speichert die L"osungen in
Dateien.

\subsection{Durchf�hrung der Berechnungen}

Die Berechnung auf einem einzelnen Computer ist so realisiert, dass dem
Problem mitgeteilt wird, dass nur ein Operative zur Verf"ugung steht. Ob
dann nur ein Teilproblem generiert wird h"angt von der Implementierung des
Problems ab. Die Berechnung erfolgt nur durch den lokalen Computer.

Bei der verteilte Berechnung wird das Problem an einen Dispatcher �bergeben
und durch alle verf�gbaren Operatives berechnet.

\subsection[Kommandozeilenparameter]{Aufruf und Kommandozeilenparameter}

Die Kommandozeile zum Aufrufen der Testumgebung ist:

{\tt java -Djava.security.policy=transmitter.pol\\
  de.unistuttgart.architeuthis.facade.RuntimeComparison\\
  -r <ProblemManager> -u <packageURL>\\
  -c <klassenname> -f <dateiname> -d}

Dabei ist:

\begin{description}
\item[-Djava.security.policy=transmitter.pol]
  Die f�r die Testumgebung zu benutzende policy-Datei. Diese Datei darf
  sich nicht in einer jar-Datei befinden; der Pfad muss also relativ
  oder absolut angegeben werden. Siehe Kapitel \hyperref[sicherh]{Sicherheit}.

\item[de.unistuttgart.architeuthis.facade.RuntimeComparison]
  Die Haupt-Klasse des Laufzeitvergleichs.

\item[-r <computesystem>] 
    Die Adresse der RMI-Registry, in der der Dispatcher eingetragen ist,
    an die sich der Laufzeitvergleich wenden soll. Exemplarisch:

    {\tt <Rechnername>:<Port>}

    Dabei ist:

    \begin{description}
    \item[Rechnername] Der Name oder die IP-Adresse des Rechners,
        auf dem der Dispatcher l�uft.
    \item[Port] Optional. Der Port, auf dem der Dispatcher auf
        Verbindungen h�rt. Falls nicht angegeben, wird 1099 angenommen und
        au�erdem muss dann der Doppelpunkt entfallen.
    \end{description}

\item[-u <classURL>] Der URL des �u�ersten Pakets der Klassen des
  Problems. Die Adresse muss entweder mit ``/'' enden, falls die Klassen
  ungepackt in der Package-Hierarchie vorliegen, oder mit dem Namen der
  jar-Datei, die die Klassen in der Package-Hierarchie enth�lt. Anstatt
  einen vorhandenen Web-Server kann man auch
  den \hyperref[classfileserver]{ClassFileServer} verwenden (s.\ Abschnitt
  \ref{classfileserver}).

\item[-c <klassenname>] Der vollst�ndige Klassenname (inklusive aller Packages)
  der Klasse des Problems, die das Interface Problem implementiert.

  {\bf Achtung:} Die Unterverzeichnisse, die durch die Package-Struktur
  entstehen, m�ssen auf dem Webserver ebenfalls vorhanden sein.\\
  Beispiel: Die Klasse {\tt MyProblem} aus dem Package {\tt mypackage} ist
  unter der URL
  {\tt http://myserver/mydir/my\-pack\-age/My\-Prob\-lem\-.class} erreichbar.
  Dann muss der Parameter \texttt{-u} den Wert
  {\tt http://my\-ser\-ver/my\-dir/} erhalten und der Parameter \texttt{-c}
  den Wert {\tt my\-pack\-age.My\-Prob\-lem}.

\item[-f <dateiname>] Optional. Der Name der Datei, in der die L�sung des
  Problems gespeichert werden soll. Die L�sung wird dabei einfach in ihrer
  serialisierten Form in die Datei geschrieben. Wird kein Dateiname angegeben,
  wird die L�sung auf der Standard-Ausgabe ausgegeben.

\item[-d] Optional. Schaltet zus�tliche Debug-Meldungen ein.
\end{description}


\section{Die Statistik}

Mit der Statistik k�nnen Informationen �ber den Zustand des Compute-Systems
abgefragt werden. Diese werden entweder in einem eigenen Fenster oder
textuell ausgegeben.

\subsection{Voraussetzungen}
Die einzige Voraussetzung zum Anzeigen der Statistik ist ein installiertes
Java-Runtime-Environment der Version 1.3 oder neuer und eine
Netzwerkanbindung oder  ein lokales Arbeiten auf dem Dispatcher-Rechner.
F�r die graphische Statistik-Ausgabe ist eine funktionierende graphische
Oberfl�che erforderlich.

\subsection[Kommandozeilenparameter]{Aufruf und Kommandozeilenparameter}

Aufruf der Version mit eigenem Fenster:

{\tt java -Djava.security.policy=statisticreader.pol\\
     de.unistuttgart.architeuthis.facade.SystemGUIStatisticsReader\\
     <computeSystem>}

Aufruf der Konsolen-Version:

{\tt java -Djava.security.policy=statisticreader.pol\\
     de.unistuttgart.architeuthis.facade.SystemTextStatisticsReader\\
     <computeSystem>}

\begin{description}
\item[-Djava.security.policy=statisticreader.pol]
  Gibt die Po\-li\-cy-Da\-tei f�r die Sta\-tistik-Anzeige an. Diese Datei 
  darf sich nicht in einer jar-Datei befinden; der Pfad muss also relativ
  oder absolut angegeben werden. Siehe dazu das Kapitel
  \hyperref[sicherh]{Sicherheit}.

\item[de.unistuttgart.architeuthis.facade.SystemGUIStatisticsReader]
  Die Haupt-Klasse der Statistik-Anzeige in einem eigenen Fenster.

\item[de.unistuttgart.architeuthis.facade.SystemTextStatisticsReader]
  Die Haupt-Klasse der textuellen Statistik-Anzeige.

\item[<computeSystem>]
    Die Adresse der RMI-Registry, in der der Dispatcher eingetragen ist,
    an die sich die Statistik wenden soll. Exemplarisch:

     {\tt <Rechnername>:<Port>}

    Dabei ist:

    \begin{description}
    \item[Rechnername] Der Name oder die IP-Adresse des Rechners,
        auf dem der Dispatcher l�uft.
    \item[Port] Optional. Der Port, auf dem der Dispatcher auf
        Verbindungen h�rt. Falls nicht angegeben, wird 1099 angenommen
        und au�erdem muss dann der : entfallen.
    \end{description}
\end{description}


    %% Realease 1.0 dieser Software wurde am Institut f�r Intelligente Systeme der
%% Universit�t Stuttgart (http://www.informatik.uni-stuttgart.de/ifi/is/) unter
%% Leitung von Dietmar Lippold (dietmar.lippold@informatik.uni-stuttgart.de)
%% entwickelt.

\chapter[Parallelisierung anhand eines Beispiels]{Der Weg zum parallelisierten Programm anhand eines Beispiels}

\section{Ein einfaches Beispiel}
Man betrachte zun�chst die Methode {\tt primzahlTeilbereich} aus {\tt Primzahlen.java} :
{\scriptsize \verbatiminput{PrimzahlTeilbereich.jav}}
\par
Gesucht werden also Primzahlen, die zwischen einem Intervall von zwei Zahlen existieren.
{\tt Prim\-zahl\-Teil\-bereich} eignet sich sehr gut zum Verteilen, da hier jeder {\tt Operative} 
f�r sich alleine rechnen kann, ohne weitere Informationen
zu ben�tigen. Die Verwaltung wird von der abstrakten
Klasse {\tt AbstractOrderedProblem.java} "ubernommen, bei der auf die Reihenfolge der zur"uckkommenden 
{\tt Teill"osungen} geachtet wird. Siehe dazu auch Kapitel 9 - Abstrakte Hilfsklassen.

\par
Desweiteren muss das Interface {\tt PartialProblem} implementiert werden.

\par
Das Interface {\tt PartialSolution} ist durch die Verwendung der vorliegenden Klasse 
{\tt ContainerPartialSolution.java} schon fertig, in der die Teill"osung einfach in
ein {\tt Serializable-Objekt} gespeichert wird.


\subsection{Implementierung von \texttt{AbstractOrderedProblem}}
In der abstrakten Klasse, die das Interface {\tt Problem} implementiert, m"ussen die zwei Methoden 
\begin{itemize}
\item protected PartialProblem createPartialProblem(int problemsExpected)
\item protected Serializable receivePartialSolution(PartialSolution parSol)
\end{itemize}
noch implementiert werden.
\par
Es wird eine untere und obere Grenze f"ur das zu durchsuchende Intervall ben"otigt, sowie
eine Schrittweite f"ur die Teilprobleme. Desweiteren wird eine Liste f"ur die L"osung gebraucht und ein Z"ahler
f"ur die noch zur"uckkommenenden Teill"osungen. Als Beispiel: \\
Gesucht sind die Primzahlen zwischen 1 und 1000. Der erste {\tt Operative} berechnet die Primzahlen von 1-100, der
zweite von 101-200 usw. D.h. die Schrittweite ist 100 und der Z"ahler f"ur die Teill"osungen wird auf 10 gesetzt.\\
\par
In der {\tt createPartialProblem} Methode wird ein neues {\tt Teilproblem} zur"uckgegeben. Hierzu werden einfach
die Grenzen des Intervalls berechnet und die untere Grenze mit der Schrittweite nach oben gez"ahlt.
\par
In der {\tt receivePartialSolution} Methode wird die erhaltene Teill"osung zur Gesamtl"osung dazugef"ugt und
"uberpr"uft, ob dies die letzte Teill"osung war. Ist dies der Fall wird die Gesamtl"osung zur"uckgegeben. Wenn
dies nicht der Fall ist, muss {\tt null} zur"uckgegeben werden, da dies das Zeichen f"ur den {\tt Compute-Manager}
ist, dass das Problem noch nicht fertig berechnet ist.

{\scriptsize \verbatiminput{java/OrderedPrimeRangeProblem.java}}

\subsection{Implementierung von \texttt{PartialProblem}}
Klar ist, dass jedes Teilproblem den Bereich enthalten muss, in dem die Primzahlen bestimmt
werden sollen. Also legt man dazu Attribute an. Dann modifiziere man den Konstruktor, so dass diese
Grenzen des Bereichs festgelegt werden.
\par
Zu guter Letzt wird noch {\tt compute()} implementiert, indem man aus der Klasse
{\tt Primzahlen} die vorher ausgew�hlte Methode \\{\tt primzahlTeilberech} aufruft. 

{\scriptsize \verbatiminput{java/OrderedPrimeParProb.java}}


\section{Weitere Beispiele}
Weiter Beispiele kann man in dem Ordner {\tt de/unis/architeuthis/testumgebung} finden. Hier eine kurze Auflistung aller
Probleme und eine kurze Beschreibung dazu:

\begin{enumerate}
\item de.unistuttgart.architeuthis.testenvironment.montecarlo.MonteCarloProblemImpl\\
	Hier wird das MonteCarlo-Verfahren zur Bestimmung der Zahl Pi parallelisert. Man bekommt somit bei einer
	festgelegten Rechenzeit eine bessere Genauigkeit f"ur Pi.
\item de.unistuttgart.architeuthis.testenvironment.random.RandomProblemImpl\\
	Hier wartet jedes Teilproblem einfach eine zuf"allige Anzahl von Sekunden. Dies ist f"ur Testzwecke recht
	interessant gewesen.
\item de.unistuttgart.architeuthis.testenvironment.caching.CachingTestProblem\\
	Hier werden drei Dummy-Klassen in den Teilproblemen geladen. Mit diesem Problem wurde verglichen, inwiefern
	sich das Laden von Klassen zeitlich bemerkbar macht. Siehe dazu auch im Kapitel Performance.
\item de.unistuttgart.architeuthis.testenvironment.prim.example.OrderedPrimeRangeProblem\\
	Dies ist das Einf"uhrungsbeispiel mittels der abstrakten Klasse {\tt Abstract\-Ordered\-Problem.java}.
\item de.unistuttgart.architeuthis.testenvironment.prim.PrimRangeProblemImpl\\
	Dieses Problem macht das gleiche wie {\tt OrderedPrimeRangeProblem}, nur dass es keine abstrakten 	Hilfsklassen benutzt.
\item de.unistuttgart.architeuthis.testenvironment.prim.PrimSequenceProblemImpl\\
	Dies ist das Beispiel f"ur Fortgeschrittene im n"achsten Kapitel.

		
\end{enumerate}

    %% Realease 1.0 dieser Software wurde am Institut f�r Intelligente Systeme der
%% Universit�t Stuttgart (http://www.informatik.uni-stuttgart.de/ifi/is/) unter
%% Leitung von Dietmar Lippold (dietmar.lippold@informatik.uni-stuttgart.de)
%% entwickelt.


\chapter[Beispiel f"ur Fortgeschrittene]{Beispiel f"ur Fortgeschrittene}

Dieses Kapitel beschreibt die Parallelisierung eines nicht--serialisierten
Problems, das vollst�ndig auf dem Compute-System ausgef�hrt werden kann, das
also insbesondere keine Daten mit anderen Programmen austauschen mu� und
keine Eingaben vom Benutzer ben�tigt. Das Kapitel dient daher prim�r der
Darstellung der Parallelisierung selbst, nicht der Darstellung der Benutzung
des Compute-Systems.

In einer realen Anwendung wird das Problem das Interface
{\tt SerializableProblem} implementieren und der Benutzer wird mit dem
Compute-Systems nur �ber die Klasse
{\tt de.unistuttgart.architeuthis.user.ProblemComputation} interagieren.

\section{Die Problemstellung}
Aus dem vorigen Kapitel sind ja schon einige einfache Problem besprochen worden.
Nun soll ein Problem ohne abstrakte Hilfsklassen realisiert werden. \\
Man betrachte zun�chst die beiden folgenden Methoden:

\begin{enumerate}
\item primzahlTeilfolge
        {\scriptsize \verbatiminput{PrimzahlTeilfolge.jav}}
\item primzahlTeilbereich
        {\scriptsize \verbatiminput{PrimzahlTeilbereich.jav}}
\end{enumerate}
\par
Ziel: Parallelisierung von der Methode primzahlTeilfolge


\subsection[Strategie]{Beschreibung der Strategie}
Am Anfang steht nat�rlich die �berlegung, ob die Methode {\tt prim\-zahl\-Teil\-folge} direkt
f�r eine verteilte Berechnung geeignet ist. Hier muss man jedoch feststellen, dass sich die 
Methode selbst nicht eignet, da hier ein gro�er Aufwand in die Kommunikation zwischen den
berechnenden Systemen gesteckt werden m��te.

\par
Jedoch f�llt schnell auf, dass sich {\tt prim\-zahl\-Teil\-bereich} gut zum Verteilen
eignet, da hier jeder {\tt Operative} f�r sich alleine rechnen kann, ohne weitere Informationen
zu ben�tigen. Die Verwaltung (also das Aufstellen der tats�chlichen Teilfolge) kann von
der Klasse {\tt Problem} auf dem Dispatcher �bernommen werden, da dies keinen gro�en Aufwand mehr bedeutet,
falls bereits alle n�tigen Primzahlen vorliegen.


\par
Die grobe Strategie sieht also wie folgt aus:
\begin{itemize}
\item als {\tt Partial\-Problem} wird ein dynamischer {\tt Prim\-zahl\-Teil\-be\-reich} berechnet
\item das {\tt Problem} setzt diese Bereiche zusammen und rechnet die Anzahl der Primzahlen aus,
  um die gew�nschte Teilfolge zu extrahieren
\end{itemize}

\section[Vorgehen]{Das Vorgehen zur Realisierung dieser Strategie}
Um diese Strategie durchzuf�hren m�ssen nat�rlich zuerst die spezifizierten Interfaces implementiert 
werden. Wir beginnen mit den einfach zu implementierenden:

\subsection{Implementierung von \texttt{PartialProblem}}
Klar ist, dass jedes Teilproblem den Bereich enthalten muss, in dem die Primzahlen bestimmt
werden sollen. Also legt man dazu Attribute an. Ebenso ben�tigt man eine Identifikationsnummer,
die sp�ter die Teill�sung identifiziert, da hier die Reihenfolge wichtig ist, in der die 
Teill�sungen verarbeitet werden. Dann modifiziere man den Konstruktor, so dass alle diese
Daten gleich gespeichert werden k�nnen.
\par
Zu guter Letzt wird noch {\tt compute()} implementiert, und zwar indem man einfach aus der Klasse
{\tt Primzahlen} die vorher ausgew�hlte Methode \\{\tt PrimzahlTeilberech} aufruft. Dabei muss
noch beachtet werden, dass die Identifikationsnummer auch der Teill�sung �bergeben wird.

{\scriptsize \verbatiminput{java/PrimePartialProblemImpl.jav}}

\subsection{Implementierung von \texttt{PartialSolution}}
Diese Klasse muss eigentlich nur das Ergebnis der Berechnung von {\tt Prim\-zahl\-Teil\-bereich}
kapseln und au�erdem die Identifikationsnummer beinhalten.
Man erstellt also zwei {\tt get}-Methoden f�r die Identifikationsnummer und die {\tt Array\-List},
die als package-local definiert werden, und nat�rlich ebenfalls Attribute daf�r.
Abschlie�end erstellt man noch einen Konstruktor, der eben diese beiden Werte als Parameter erwartet und diese
sofort in die Attribute speichert.

{\scriptsize \verbatiminput{java/PrimePartialSolutionImpl.jav}}


\subsection{Implementierung von \texttt{Problem}}
Nun wird es etwas komplizierter, da einige �berlegungen zuvor get�tigt werden
m�ssen. Zun�chst scheint es sinnvoll, eine obere Absch�tzung f�r die gr��te
gesuchte Primzahl zu finden.
Wir benutzen hier eine Absch�tzung nach Rosser und Schoenfeld (siehe:
J. B. Rosser and L. Schoenfeld. Approximate formulas for some functions of prime
numbers. Illinois Journal of Mathematics, 6:64--94, 1962), auf die hier nicht
weiter eingegangen wird. Die Absch�tzung besagt, dass die n-te Primzahl mit
Sicherheit kleiner ist als n*(ln(n)+ln(ln(n))-1/2) f�r n > 15. Somit ist eine
einfache obere Grenze bekannt, bis zu der alle Zahlen untersucht werden m�ssen.
Nun muss man noch �berlegen, auf welche Art man das gesamte zu untersuchende
Intervall auf die verschiedenen {\tt Operatives} aufteilt. Der Einfachheit
halber w�hlen wir hier eine �quidistante Aufteilung des Intervalls.

\par
Au�erdem ist zu beachten, dass bei diesem Problem die Reihenfolge, in der die L�sungen eintreffen,
wichtig ist. So kann die L�sung Nr.5 nicht bearbeitet werden, bis L�sung Nr.4 eingetroffen ist. Um diesem
Problem zu begegnen, speichert man alle eintreffenden L�sungen zun�chst in einer {\tt HashMap}, als Schl�ssel
benutzt man die Identifikationsnummer. Anschlie�end kann man bequem in einer Schleife die aufeinanderfolgenden
Teill�sungen verarbeiten. Dazu muss nat�rlich noch die Identifikationsnummer der Teill�sung gespeichert werden,
die als n�chste bearbeitet werden muss.
\par
Nun k�nnte man alle L�sungen in der {\tt HashMap} behalten und warten, bis gen�gend Primzahlen ermittelt wurden.
Das scheint jedoch nicht zweckm��ig. Effizienter ist es, bis zum Erreichen der gesuchten Untergrenze nur die
Anzahl der bisher gefundenen Primzahlen zu speichern. Erst falls diese Grenze �berschritten wird, werden die
Primzahlen in die Gesamtl�sung �bernommen. Mit dieser Strategie k�nnen alle bearbeiteten Teill�sungen sofort
aus der {\tt HashMap} entfernt werden.
\par
Falls nun der {\tt Dispatcher} nach der Gesamtl�sung fragt, wird �berpr�ft, ob die Gesamtl�sung schon
die L�nge des gesuchten Intervalls hat. Falls dies nicht der Fall ist, wird {\tt null} zur�ckgegeben, ansonsten
die Gesamtl�sung.

{\scriptsize \verbatiminput{java/PrimeSequenceProblemImpl.jav}}


%  \printindex
\end{document}
