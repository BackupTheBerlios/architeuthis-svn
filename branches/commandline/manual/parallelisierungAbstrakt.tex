%% Realease 1.0 dieser Software wurde am Institut f�r Intelligente Systeme der
%% Universit�t Stuttgart (http://www.informatik.uni-stuttgart.de/ifi/is/) unter
%% Leitung von Dietmar Lippold (dietmar.lippold@informatik.uni-stuttgart.de)
%% entwickelt.

\chapter[Parallelisierung anhand eines Beispiels]{Der Weg zum parallelisierten Programm anhand eines Beispiels}

\section{Ein einfaches Beispiel}
Man betrachte zun�chst die Methode {\tt primzahlTeilbereich} aus {\tt Primzahlen.java} :
{\scriptsize \verbatiminput{PrimzahlTeilbereich.jav}}
\par
Gesucht werden also Primzahlen, die zwischen einem Intervall von zwei Zahlen existieren.
{\tt Prim\-zahl\-Teil\-bereich} eignet sich sehr gut zum Verteilen, da hier jeder {\tt Operative} 
f�r sich alleine rechnen kann, ohne weitere Informationen
zu ben�tigen. Die Verwaltung wird von der abstrakten
Klasse {\tt AbstractOrderedProblem.java} "ubernommen, bei der auf die Reihenfolge der zur"uckkommenden 
{\tt Teill"osungen} geachtet wird. Siehe dazu auch Kapitel 9 - Abstrakte Hilfsklassen.

\par
Desweiteren muss das Interface {\tt PartialProblem} implementiert werden.

\par
Das Interface {\tt PartialSolution} ist durch die Verwendung der vorliegenden Klasse 
{\tt ContainerPartialSolution.java} schon fertig, in der die Teill"osung einfach in
ein {\tt Serializable-Objekt} gespeichert wird.


\subsection{Implementierung von \texttt{AbstractOrderedProblem}}
In der abstrakten Klasse, die das Interface {\tt Problem} implementiert, m"ussen die zwei Methoden 
\begin{itemize}
\item protected PartialProblem createPartialProblem(int problemsExpected)
\item protected Serializable receivePartialSolution(PartialSolution parSol)
\end{itemize}
noch implementiert werden.
\par
Es wird eine untere und obere Grenze f"ur das zu durchsuchende Intervall ben"otigt, sowie
eine Schrittweite f"ur die Teilprobleme. Desweiteren wird eine Liste f"ur die L"osung gebraucht und ein Z"ahler
f"ur die noch zur"uckkommenenden Teill"osungen. Als Beispiel: \\
Gesucht sind die Primzahlen zwischen 1 und 1000. Der erste {\tt Operative} berechnet die Primzahlen von 1-100, der
zweite von 101-200 usw. D.h. die Schrittweite ist 100 und der Z"ahler f"ur die Teill"osungen wird auf 10 gesetzt.\\
\par
In der {\tt createPartialProblem} Methode wird ein neues {\tt Teilproblem} zur"uckgegeben. Hierzu werden einfach
die Grenzen des Intervalls berechnet und die untere Grenze mit der Schrittweite nach oben gez"ahlt.
\par
In der {\tt receivePartialSolution} Methode wird die erhaltene Teill"osung zur Gesamtl"osung dazugef"ugt und
"uberpr"uft, ob dies die letzte Teill"osung war. Ist dies der Fall wird die Gesamtl"osung zur"uckgegeben. Wenn
dies nicht der Fall ist, muss {\tt null} zur"uckgegeben werden, da dies das Zeichen f"ur den {\tt Compute-Manager}
ist, dass das Problem noch nicht fertig berechnet ist.

{\scriptsize \verbatiminput{java/OrderedPrimeRangeProblem.java}}

\subsection{Implementierung von \texttt{PartialProblem}}
Klar ist, dass jedes Teilproblem den Bereich enthalten muss, in dem die Primzahlen bestimmt
werden sollen. Also legt man dazu Attribute an. Dann modifiziere man den Konstruktor, so dass diese
Grenzen des Bereichs festgelegt werden.
\par
Zu guter Letzt wird noch {\tt compute()} implementiert, indem man aus der Klasse
{\tt Primzahlen} die vorher ausgew�hlte Methode \\{\tt primzahlTeilberech} aufruft. 

{\scriptsize \verbatiminput{java/OrderedPrimeParProb.java}}


\section{Weitere Beispiele}
Weiter Beispiele kann man in dem Ordner {\tt de/unis/architeuthis/testumgebung} finden. Hier eine kurze Auflistung aller
Probleme und eine kurze Beschreibung dazu:

\begin{enumerate}
\item de.unistuttgart.architeuthis.testenvironment.montecarlo.MonteCarloProblemImpl\\
	Hier wird das MonteCarlo-Verfahren zur Bestimmung der Zahl Pi parallelisert. Man bekommt somit bei einer
	festgelegten Rechenzeit eine bessere Genauigkeit f"ur Pi.
\item de.unistuttgart.architeuthis.testenvironment.random.RandomProblemImpl\\
	Hier wartet jedes Teilproblem einfach eine zuf"allige Anzahl von Sekunden. Dies ist f"ur Testzwecke recht
	interessant gewesen.
\item de.unistuttgart.architeuthis.testenvironment.caching.CachingTestProblem\\
	Hier werden drei Dummy-Klassen in den Teilproblemen geladen. Mit diesem Problem wurde verglichen, inwiefern
	sich das Laden von Klassen zeitlich bemerkbar macht. Siehe dazu auch im Kapitel Performance.
\item de.unistuttgart.architeuthis.testenvironment.prim.example.OrderedPrimeRangeProblem\\
	Dies ist das Einf"uhrungsbeispiel mittels der abstrakten Klasse {\tt Abstract\-Ordered\-Problem.java}.
\item de.unistuttgart.architeuthis.testenvironment.prim.PrimRangeProblemImpl\\
	Dieses Problem macht das gleiche wie {\tt OrderedPrimeRangeProblem}, nur dass es keine abstrakten 	Hilfsklassen benutzt.
\item de.unistuttgart.architeuthis.testenvironment.prim.PrimSequenceProblemImpl\\
	Dies ist das Beispiel f"ur Fortgeschrittene im n"achsten Kapitel.

		
\end{enumerate}
